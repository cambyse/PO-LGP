\documentclass[10pt,fleqn,twoside]{article}
\usepackage{palatino}
\usepackage{amsmath}
\usepackage{amssymb}
\usepackage{amsfonts}
\usepackage{amsthm}
\usepackage{eucal}
\usepackage{graphicx}
\usepackage{color}

\usepackage[round]{natbib}
\bibliographystyle{abbrvnat}
%\usepackage[german]{babel}
%\usepackage[utf8]{inputenc}

\graphicspath{{pics/}{figs/}{~/write/tex/pics/}{~/write/tex/figs/}{~/teaching/pics-all/}}
\usepackage{geometry}
\geometry{a4paper,hdivide={35mm,*,35mm},vdivide={35mm,*,35mm}}
\renewcommand{\baselinestretch}{1.1}

\newenvironment{items}{
\par\small
\begin{list}{--}{\leftmargin4ex \rightmargin0ex \labelsep1ex \labelwidth2ex
\topsep0pt \parsep0ex \itemsep3pt}
}{
\end{list}
}


  \renewcommand{\a}{\ensuremath\alpha}
  \renewcommand{\b}{\beta}
  \renewcommand{\c}{\gamma}
  \renewcommand{\d}{\delta}
    \newcommand{\D}{\Delta}
    \newcommand{\e}{\epsilon}
    \newcommand{\g}{\gamma}
    \newcommand{\G}{\Gamma}
  \renewcommand{\l}{\lambda}
  \renewcommand{\L}{\Lambda}
    \newcommand{\m}{\mu}
    \newcommand{\n}{\nu}
    \newcommand{\N}{\nabla}
  \renewcommand{\k}{\kappa}
  \renewcommand{\o}{\omega}
  \renewcommand{\O}{\Omega}
    \newcommand{\p}{\phi}
    \newcommand{\ph}{\varphi}
  \renewcommand{\P}{\Phi}
  \renewcommand{\r}{\varrho}
    \newcommand{\s}{\sigma}
  \renewcommand{\S}{\Sigma}
  \renewcommand{\t}{\theta}
    \newcommand{\T}{\Theta}
  %\renewcommand{\v}{\vartheta}
    \newcommand{\x}{\xi}
    \newcommand{\X}{\Xi}
    \newcommand{\Y}{\Upsilon}

  \renewcommand{\AA}{{\cal A}}
    \newcommand{\BB}{{\cal B}}
    \newcommand{\CC}{{\cal C}}
    \newcommand{\EE}{{\cal E}}
    \newcommand{\FF}{{\cal F}}
    \newcommand{\GG}{{\cal G}}
    \newcommand{\HH}{{\cal H}}
    \newcommand{\II}{{\cal I}}
    \newcommand{\KK}{{\cal K}}
    \newcommand{\LL}{{\cal L}}
    \newcommand{\MM}{{\cal M}}
    \newcommand{\NN}{{\cal N}}
    \newcommand{\oNN}{\overline\NN}
    \newcommand{\OO}{{\cal O}}
    \newcommand{\PP}{{\cal P}}
    \newcommand{\QQ}{{\cal Q}}
    \newcommand{\RR}{{\cal R}}
  \renewcommand{\SS}{{\cal S}}
    \newcommand{\TT}{{\cal T}}
    \newcommand{\uu}{{\cal u}}
    \newcommand{\UU}{{\cal U}}
    \newcommand{\VV}{{\cal V}}
    \newcommand{\XX}{{\cal X}}
    \newcommand{\YY}{{\cal Y}}
    \newcommand{\SOSO}{{\cal SO}}
    \newcommand{\GLGL}{{\cal GL}}

    \newcommand{\Ee}{{\rm E}}

  \newcommand{\NNN}{{\mathbb{N}}}
  \newcommand{\ZZZ}{{\mathbb{Z}}}
  %\newcommand{\RRR}{{\mathrm{I\!R}}}
  \newcommand{\RRR}{{\mathbb{R}}}
  \newcommand{\CCC}{{\mathbb{C}}}
  \newcommand{\one}{{{\bf 1}}}
  \newcommand{\eee}{\text{e}}

  \newcommand{\NNNN}{{\overline{\cal N}}}

  \renewcommand{\[}{\Big[}
  \renewcommand{\]}{\Big]}
  \renewcommand{\(}{\Big(}
  \renewcommand{\)}{\Big)}
  \renewcommand{\|}{\,|\,}
  \renewcommand{\=}{\!=\!}
    \newcommand{\<}{{\ensuremath\langle}}
  \renewcommand{\>}{{\ensuremath\rangle}}

  \newcommand{\Prob}{{\rm Prob}}
  \newcommand{\Dir}{{\rm Dir}}
  \newcommand{\Aut}{{\rm Aut}}
  \newcommand{\cor}{{\rm cor}}
  \newcommand{\corr}{{\rm corr}}
  \newcommand{\cov}{{\rm cov}}
  \newcommand{\sd}{{\rm sd}}
  \newcommand{\tr}{{\rm tr}}
  \newcommand{\Tr}{{\rm Tr}}
  \newcommand{\rank}{{\rm rank}}
  \newcommand{\diag}{{\rm diag}}
  \newcommand{\id}{{\rm id}}
  \newcommand{\Id}{{\rm\bf I}}
  \newcommand{\Gl}{{\rm Gl}}
  \renewcommand{\th}{\ensuremath{{}^\text{th}} }
  \newcommand{\lag}{\mathcal{L}}
  \newcommand{\inn}{\rfloor}
  \newcommand{\lie}{\pounds}
  \newcommand{\longto}{\longrightarrow}
  \newcommand{\speer}{\parbox{0.4ex}{\raisebox{0.8ex}{$\nearrow$}}}
  \renewcommand{\dag}{ {}^\dagger }
  \newcommand{\Ji}{J^\sharp}
  \newcommand{\h}{{}^\star}
  \newcommand{\w}{\wedge}
  \newcommand{\too}{\longrightarrow}
  \newcommand{\oot}{\longleftarrow}
  \newcommand{\To}{\Rightarrow}
  \newcommand{\oT}{\Leftarrow}
  \newcommand{\oTo}{\Leftrightarrow}
  \newcommand{\Too}{\;\Longrightarrow\;}
  \newcommand{\oto}{\leftrightarrow}
  \newcommand{\ot}{\leftarrow}
  \newcommand{\ootoo}{\longleftrightarrow}
  \newcommand{\ow}{\stackrel{\circ}\wedge}
  \newcommand{\feed}{\nonumber \\}
  \newcommand{\comma}{~,\quad}
  \newcommand{\period}{~.\quad}
  \newcommand{\del}{\partial}
%  \newcommand{\quabla}{\Delta}
  \newcommand{\point}{$\bullet~~$}
  \newcommand{\doubletilde}{ ~ \raisebox{0.3ex}{$\widetilde {}$} \raisebox{0.6ex}{$\widetilde {}$} \!\! }
  \newcommand{\topcirc}{\parbox{0ex}{~\raisebox{2.5ex}{${}^\circ$}}}
  \newcommand{\topdot} {\parbox{0ex}{~\raisebox{2.5ex}{$\cdot$}}}
  \newcommand{\topddot} {\parbox{0ex}{~\raisebox{1.3ex}{$\ddot{~}$}}}
  \newcommand{\sym}{\topcirc}
  \newcommand{\tsum}{\textstyle\sum}

  \newcommand{\half}{\ensuremath{\frac{1}{2}}}
  \newcommand{\third}{\ensuremath{\frac{1}{3}}}
  \newcommand{\fourth}{\ensuremath{\frac{1}{4}}}

  \newcommand{\ubar}{\underline}
  %\renewcommand{\vec}{\underline}
  \renewcommand{\vec}{\boldsymbol}
  %\renewcommand{\_}{\underset}
  %\renewcommand{\^}{\overset}
  %\renewcommand{\*}{{\rm\raisebox{-.6ex}{\text{*}}{}}}
  \renewcommand{\*}{\text{\footnotesize\raisebox{-.4ex}{*}{}}}

  \newcommand{\gto}{{\raisebox{.5ex}{${}_\rightarrow$}}}
  \newcommand{\gfrom}{{\raisebox{.5ex}{${}_\leftarrow$}}}
  \newcommand{\gnto}{{\raisebox{.5ex}{${}_\nrightarrow$}}}
  \newcommand{\gnfrom}{{\raisebox{.5ex}{${}_\nleftarrow$}}}

  %\newcommand{\RND}{{\SS}}
  %\newcommand{\IF}{\text{if }}
  %\newcommand{\AND}{\textsc{and }}
  %\newcommand{\OR}{\textsc{or }}
  %\newcommand{\XOR}{\textsc{xor }}
  %\newcommand{\NOT}{\textsc{not }}

  %\newcommand{\argmax}[1]{{\rm arg}\!\max_{#1}}
  %\newcommand{\argmin}[1]{{\rm arg}\!\min_{#1}}
  \DeclareMathOperator*{\argmax}{argmax}
  \DeclareMathOperator*{\argmin}{argmin}
  \DeclareMathOperator{\sign}{sign}
  \DeclareMathOperator{\acos}{acos}
  %\newcommand{\argmax}[1]{\underset{~#1}{\text{argmax}}\;}
  %\newcommand{\argmin}[1]{\underset{~#1}{\text{argmin}}\;}
  \newcommand{\ee}[1]{\ensuremath{\cdot10^{#1}}}
  \newcommand{\sub}[1]{\ensuremath{_{\text{#1}}}}
  \newcommand{\up}[1]{\ensuremath{^{\text{#1}}}}
  \newcommand{\kld}[3][{}]{D_{#1}\big(#2\,\big|\!\big|\,#3\big)}
  %\newcommand{\kld}[2]{D\big(#1:#2\big)}
  \newcommand{\sprod}[2]{\big<#1\,,\,#2\big>}
  \newcommand{\End}{\text{End}}
  \newcommand{\txt}[1]{\quad\text{#1}\quad}
  \newcommand{\Over}[2]{\genfrac{}{}{0pt}{0}{#1}{#2}}
  %\newcommand{\mat}[1]{{\bf #1}}
  \newcommand{\arr}[2]{\hspace*{-.5ex}\begin{array}{#1}#2\end{array}\hspace*{-.5ex}}
  \newcommand{\mat}[3][.9]{
    \renewcommand{\arraystretch}{#1}{\scriptscriptstyle{\left(
      \hspace*{-1ex}\begin{array}{#2}#3\end{array}\hspace*{-1ex}
    \right)}}\renewcommand{\arraystretch}{1.2}
  }
  %\newcommand{\case}[1]{\left\{\arr{ll}{#1}\right.}
  \newcommand{\seq}[1]{\textsf{\<#1\>}}
  \newcommand{\seqq}[1]{\textsf{#1}}
  \newcommand{\floor}[1]{\lfloor#1\rfloor}
  \newcommand{\Exp}[2]{{\rm E}_{#1}\{#2\}} \newcommand{\E}{{\rm E}}

  \newcommand{\href}[1]{}
  \newcommand{\url}[2]{\href{#1}{{\color{blue}#2}}}
  \newcommand{\anchor}[2]{\begin{picture}(0,0)\put(#1){#2}\end{picture}}
  \newcommand{\pagebox}{\begin{picture}(0,0)\put(-3,-23){
    \textcolor[rgb]{.5,1,.5}{\framebox[\textwidth]{\rule[-\textheight]{0pt}{0pt}}}}
    \end{picture}}

  \newcommand{\hide}[1]{
    \begin{list}{}{\leftmargin0ex \rightmargin0ex \topsep0ex \parsep0ex}
       \helvetica{5}{1}{m}{n}
       \renewcommand{\section}{\par SECTION: }
       \renewcommand{\subsection}{\par SUBSECTION: }
       \item[$~~\blacktriangleright$]
       #1%$\blacktriangleleft~~$
       \message{^^JHIDE--Warning!^^J}
    \end{list}
  }
  %\newcommand{\hide}[1]{{\tt[hide:~}{\footnotesize\sf #1}{\tt]}\message{^^JHIDE--Warning!^^J}}
  \newcommand{\Hide}{\renewcommand{\hide}[1]{\message{^^JHIDE--Warning (hidden)!^^J}}}
  \newcommand{\HIDE}{\renewcommand{\hide}[1]{}}
  \newcommand{\fullhide}[1]{}
  \newcommand{\todo}[1]{{\tt[TODO: #1]}\message{^^JTODO--Warning: #1^^J}}
  \newcommand{\Todo}{\renewcommand{\todo}[1]{\message{^^JTODO--Warning (hidden)!^^J}}}
  %\renewcommand{\title}[1]{\renewcommand{\thetitle}{#1}}
  \newcommand{\header}{\begin{document}\mytitle\cleardefs}
  \newcommand{\contents}{{\tableofcontents}\renewcommand{\contents}{}}
  \newcommand{\footer}{\small\bibliography{bibs}\end{document}}
  \newcommand{\widepaper}{\usepackage{geometry}\geometry{a4paper,hdivide={25mm,*,25mm},vdivide={25mm,*,25mm}}}
  \newcommand{\moviex}[2]{\movie[externalviewer]{#1}{#2}} %\pdflatex\usepackage{multimedia}
  \newcommand{\rbox}[1]{\fboxrule2mm\fcolorbox[rgb]{1,.85,.85}{1,.85,.85}{#1}}
  \newcommand{\redbox}[2]{\fboxrule2mm\fcolorbox[rgb]{1,.7,.7}{1,.7,.7}{\begin{minipage}{#1}\center#2\end{minipage}}}
  \newcommand{\twocol}[5][c]{\begin{minipage}[#1]{#2\columnwidth}#4\end{minipage}\begin{minipage}[#1]{#3\columnwidth}#5\end{minipage}}
  \newcommand{\threecoltext}[7][c]{\begin{minipage}[#1]{#2\textwidth}#5\end{minipage}\begin{minipage}[#1]{#3\textwidth}#6\end{minipage}\begin{minipage}[#1]{#4\textwidth}#7\end{minipage}}
  \newcommand{\threecol}[7][c]{\begin{minipage}[#1]{#2\columnwidth}#5\end{minipage}\begin{minipage}[#1]{#3\columnwidth}#6\end{minipage}\begin{minipage}[#1]{#4\columnwidth}#7\end{minipage}}
  \newcommand{\helvetica}[4]{\setlength{\unitlength}{1pt}\fontsize{#1}{#1}\linespread{#2}\usefont{OT1}{phv}{#3}{#4}}
  \newcommand{\helve}[1]{\helvetica{#1}{1.5}{m}{n}}
  \newcommand{\german}{\usepackage[german]{babel}\usepackage[latin1]{inputenc}}

\newcommand{\norm}[2]{|\!|#1|\!|_{{}_{#2}}}
\newcommand{\expr}[1]{[\hspace{-.2ex}[#1]\hspace{-.2ex}]}

\newcommand{\Jwi}{J^\sharp_W}
\newcommand{\Twi}{T^\sharp_W}
\newcommand{\Jci}{J^\natural_C}
\newcommand{\hJi}{{\bar J}^\sharp}
\renewcommand{\|}{\,|\,}
\renewcommand{\=}{\!=\!}
\newcommand{\myminus}{{\hspace*{-.3pt}\text{\rm -}\hspace*{-.5pt}}}
\newcommand{\myplus}{{\hspace*{-.3pt}\text{\rm +}\hspace*{-.5pt}}}
\newcommand{\1}{{\myminus1}}
\newcommand{\2}{{\myminus2}}
\newcommand{\3}{{\myminus3}}
\newcommand{\mT}{{\myminus{}\top}}
\newcommand{\po}{{\myplus1}}
\newcommand{\pt}{{\myplus2}}
%\renewcommand{\-}{\myminus}
%\newcommand{\+}{\myplus}
\renewcommand{\T}{{\!\top\!}}
\renewcommand{\mT}{{\myminus{}\top}}

\newenvironment{centy}{
\vspace{15mm}
\large
\hspace*{5mm}
\begin{minipage}{8cm}\it\color{blue}
}{
\end{minipage}
}

\newcommand{\old}{{\text{old}}}
\newcommand{\new}{{\text{new}}}


%%%%%%%%%%%%%%%%%%%%%%%%%%%%%%%%%%%%%%%%%%%%%%%%%%%%%%%%%%%%%%%%%%%%%%%%%%%%%%%%
\pdflatex
\usepackage{framed}
  \definecolor{shadecolor}{gray}{0.9}
  \setlength{\FrameSep}{3pt}
\usepackage{fancyvrb}
\DefineShortVerb{\@}

\title{FOL\\A graph implementation of First Order Logic}
\author{M Toussaint}

\begin{document}
\maketitle

\section{Graph basics}

Our graph syntax is a bit different to standard conventions. Actually,
our graph is a hyper graph: nodes can play the role of normal nodes,
or hypernodes (=edges or factors/cliques) that connect other nodes. At
the same time our graph is a typed dictionary: every node has a set of
keys (or tags, to retrieve the node by name) and a typed value (every
node can be of a different type).
\begin{itemize}
\item A graph is a set of nodes
\item Every node has three properties:
\begin{items}
\item A tuple of \textbf{keys} (=strings)
\item A tuple of \textbf{parents} (=references to other nodes)
\item A typed \textbf{value} (the type may differ for every node)
\end{items}
\end{itemize}

The ascii file format of a graph clarifies this:
\begin{shaded}
\begin{Verbatim}[fontfamily=courier,fontsize=\tiny]
## a trivial graph
x    	     # key=x, value=true, parents=none
y	     # key=y, value=true, parents=none
(x y)	     # key=none, value=true, parents=x y
(-1 -2)	     # key=none, value=true, parents=the previous and the y-node

## a bit more verbose graph
node A{ color=blue }		# keys=node A, value=<Graph>, parents=none
node B{ color=red, value=5 }	# keys=node B, value=<Graph>, parents=none
edge C(A,B){ width=2 }	   	# keys=edge C, value=<Graph>, parents=A B
hyperedge(B C) = 5   		# keys=hyperedge, value=5, parents=B C

## standard value types
a=string	# MT::String (except for keywords 'true' and 'false' and 'Mod' and 'Include')
b="STRING"	# MT::String (does not require a '=')
c='file.txt'	# MT::FileToken (does not require a '=')
d=-0.1234	# double
e=[1 2 3 0.5]	# MT::arr (does not require a '=')
f=(c d e)	# MT::Array<*Node> (list of other nodes in the Graph)
g      		# bool (default: true)
h=true		# bool
i=false		# bool
j={ a=0 }	# sub-Graph (special: does not require a '=')

## parsing: = {..} (..) , and \n are separators for parsing key-value-pairs
b0=false b1 b2, b3() b4   # 4 booleans with keys 'b0', 'b1 b2', 'b3', 'b4'
k={ a, b=0.2 x="hallo"	  # sub-Graph with 6 nodes
  y
  z()=filename.org x }

## special Node Keys

## merging: after reading all nodes, the Graph takes all Merge nodes, deletes the Merge tag, and calls a merge()
## this example will modify/append the respective attributes of k
Merge k { y=false, z=otherString, b=7, c=newAttribute }

## including
Include = 'example_include.kvg'   # first creates a normal FileToken node then opens and includes the file directly

## referring to nodes (constants/macros)
macro = 5
val=(macro) # *G["val"]->getValue<double>() will return 5

## any types
trans=<T t(10 0 0)>  # 'T' is the tag for an arbitrary type (here an ors::Transformation) which was registered somewhere in the code using the registry()
                     # (does not require a '=')

## strange notations
a()	   # key=a, value=true, parents=none
()	   # key=none, value=true, parents=none
[1 2 3 4]  # key=none, value=MT::arr, parents=none
[2 3 4]
[4 6]
\end{Verbatim}
\end{shaded}

A special case is when a node has a Graph-typed value. This is
considered a \textbf{subgraph}. Subgraphs are sometimes handled
special: their nodes can have parents from the containing graph, or
from other subgraphs of the containing graph. (When deep copying
graphs they requre special care.)

\section{Representing KB as a Graph}

We represent everything, a full knowledge base (KB), as a graph:
\begin{itemize}
\item Symbols (both, constants and predicate symbols) are nil-valued
  nodes. We assume that they are declared in the root scope of the
  graph
\item A grounded literal is a tuple of symbols, for instance
% 
  @(on box1 box2)@. Note that we can equally write this as
%
  @(box1 on box2)@. There is no need to have the 'predicate' first. In
  fact, the basic methods do not distinguish between predicate and
  contant symbols.
\item A universal quantification $\forall X$ is represented as a scope
  (=subgraph) which first declares the logic variables as nil-valued
  nodes as the subgraph, then the rest. The rest is typically an
  implication, i.e., a rule. For instance
$$\forall X Y~ p(X, Y) q(Y) \To q(X)$$
 is represented as @{X, Y, { (p X Y) (q Y) } { (q X) }@
where the precondition and postconditions are subgraphs of the
rule-subgraph.
\end{itemize}
Here is how the standard FOL example from Stuart Russell's lecture is represented:
\begin{shaded}
\begin{Verbatim}[fontfamily=courier,fontsize=\tiny]
Constant M1
Constant M2
Constant Nono
Constant America
Constant West

American
Weapon
Sells
Hostile
Criminal
Missile
Owns
Enemy

STATE {
(Owns Nono M1),
(Missile M1),
(Missile M2),
(American West),
(Enemy Nono America)
}

Query { (Criminal West) }

Rule {
x, y, z,
{ (American x) (Weapon y) (Sells x y z) (Hostile z) }
{ (Criminal x) }
}

Rule {
x
{ (Missile x) (Owns Nono x) }
{ (Sells West x Nono) }
}

Rule {
x
{ (Missile x) }
{ (Weapon x) }
}

Rule {
x
{ (Enemy x America) }
{ (Hostile x) }
}
\end{Verbatim}
\end{shaded}

\subsection{Valued predicates}

By default all tuples in the graph are boolean-valued with default
value true. In the above example all literals are actually
true-valued. A rule @{X, Y, { (p X Y) (q Y) } { (q X)! }@
means $\forall X Y~ p(X, Y) q(Y) \To \neg q(X)$. If in the KB we only
store true facts, this would 'delete' the fact @(q X)!@ from the KB
(for some $X$).

As nodes of our graph can be of any type, we can represent predicates
of any type, for instance @{X, Y, { (p X Y) (q Y)=3 } { (q X)=4 }@
would let $p(X)$ be double-typed.


\section{Methods}

The most important methods are the following:
\begin{itemize}
\item Checking whether \textbf{two facts are equal}. Facts are
  grounded literals. Equality is simply checked by checking if all
  symbols (predicate or constant) in the tuples are equal. Optionally
  (by default), it is also checked if the fact values are equal.
\item Checking whether \textbf{a fact equals a
  literal+substitution}. The literal is a tuple of symbols, some of
  which may be first order variables. All variables must be of the
  same scope (=declared in the same subgraph, in the same rule). The
  substitution is a mapping of these variables to root-level symbols
  (predicate of constant symbols). The methods loops through the
  literal's symbols; whenever a symbol is in the substitution scope it
  replaces it by the substitution; then compares to the fact
  symbol. Optionally (by default) also the value is checked for equality.
\item Check whether \textbf{a fact is directly true in a KB (or
  scope)} (without inference). This uses the graph connectivity to
  quickly find any KB-fact that shares the same symbols; then checks
  these for exact equality.
\item Check whether \textbf{a literal+substitution is directly true in
  a KB} (without inference).
\item Given a single literal with only ONE logic variable, and a KB of facts,
  \textbf{compute the domain} (=possible values of the variable) for the
  literal to be true. If the literal is negated the $D \gets
  D\setminus d$, otherwise $D \gets D \cup d$ if the $d$ is the domain
  for true facts in the KB. [TODO: do this also for multi-variable literals]
\item \textbf{Compute the set of possible substitutions for a
  conjunction of literals} (typically precondition of a rule) to be
  true in a KB.
\item \textbf{Apply a set of 'effect literals'} (RHS of a rule): generate facts
  that are substituted literals
\end{itemize}

Given these methods, forward chaining, or forward simulation (for
MCTS) is straight-forward. 

\end{document}

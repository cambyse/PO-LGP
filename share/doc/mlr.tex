
\documentclass[10pt,fleqn,twoside]{article}
\usepackage{palatino}
\usepackage{amsmath}
\usepackage{amssymb}
\usepackage{amsfonts}
\usepackage{amsthm}
\usepackage{eucal}
\usepackage{graphicx}
\usepackage{color}
\usepackage{framed}
  \definecolor{shadecolor}{gray}{0.9}
  \setlength{\FrameSep}{3pt}

\usepackage[round]{natbib}
\bibliographystyle{abbrvnat}
%\usepackage[german]{babel}
%\usepackage[utf8]{inputenc}

\graphicspath{{pics/}{figs/}{~/write/tex/pics/}{~/write/tex/figs/}{~/teaching/pics-all/}}
\usepackage{geometry}
\geometry{a4paper,hdivide={35mm,*,35mm},vdivide={35mm,*,35mm}}
\renewcommand{\baselinestretch}{1.1}

\newenvironment{items}{
\par\small
\begin{list}{--}{\leftmargin4ex \rightmargin0ex \labelsep1ex \labelwidth2ex
\topsep0pt \parsep0ex \itemsep3pt}
}{
\end{list}
}


  \renewcommand{\a}{\ensuremath\alpha}
  \renewcommand{\b}{\beta}
  \renewcommand{\c}{\gamma}
  \renewcommand{\d}{\delta}
    \newcommand{\D}{\Delta}
    \newcommand{\e}{\epsilon}
    \newcommand{\g}{\gamma}
    \newcommand{\G}{\Gamma}
  \renewcommand{\l}{\lambda}
  \renewcommand{\L}{\Lambda}
    \newcommand{\m}{\mu}
    \newcommand{\n}{\nu}
    \newcommand{\N}{\nabla}
  \renewcommand{\k}{\kappa}
  \renewcommand{\o}{\omega}
  \renewcommand{\O}{\Omega}
    \newcommand{\p}{\phi}
    \newcommand{\ph}{\varphi}
  \renewcommand{\P}{\Phi}
  \renewcommand{\r}{\varrho}
    \newcommand{\s}{\sigma}
  \renewcommand{\S}{\Sigma}
  \renewcommand{\t}{\theta}
    \newcommand{\T}{\Theta}
  %\renewcommand{\v}{\vartheta}
    \newcommand{\x}{\xi}
    \newcommand{\X}{\Xi}
    \newcommand{\Y}{\Upsilon}

  \renewcommand{\AA}{{\cal A}}
    \newcommand{\BB}{{\cal B}}
    \newcommand{\CC}{{\cal C}}
    \newcommand{\EE}{{\cal E}}
    \newcommand{\FF}{{\cal F}}
    \newcommand{\GG}{{\cal G}}
    \newcommand{\HH}{{\cal H}}
    \newcommand{\II}{{\cal I}}
    \newcommand{\KK}{{\cal K}}
    \newcommand{\LL}{{\cal L}}
    \newcommand{\MM}{{\cal M}}
    \newcommand{\NN}{{\cal N}}
    \newcommand{\oNN}{\overline\NN}
    \newcommand{\OO}{{\cal O}}
    \newcommand{\PP}{{\cal P}}
    \newcommand{\QQ}{{\cal Q}}
    \newcommand{\RR}{{\cal R}}
  \renewcommand{\SS}{{\cal S}}
    \newcommand{\TT}{{\cal T}}
    \newcommand{\uu}{{\cal u}}
    \newcommand{\UU}{{\cal U}}
    \newcommand{\VV}{{\cal V}}
    \newcommand{\XX}{{\cal X}}
    \newcommand{\YY}{{\cal Y}}
    \newcommand{\SOSO}{{\cal SO}}
    \newcommand{\GLGL}{{\cal GL}}

    \newcommand{\Ee}{{\rm E}}

  \newcommand{\NNN}{{\mathbb{N}}}
  \newcommand{\ZZZ}{{\mathbb{Z}}}
  %\newcommand{\RRR}{{\mathrm{I\!R}}}
  \newcommand{\RRR}{{\mathbb{R}}}
  \newcommand{\CCC}{{\mathbb{C}}}
  \newcommand{\one}{{{\bf 1}}}
  \newcommand{\eee}{\text{e}}

  \newcommand{\NNNN}{{\overline{\cal N}}}

  \renewcommand{\[}{\Big[}
  \renewcommand{\]}{\Big]}
  \renewcommand{\(}{\Big(}
  \renewcommand{\)}{\Big)}
  \renewcommand{\|}{\,|\,}
  \renewcommand{\=}{\!=\!}
    \newcommand{\<}{{\ensuremath\langle}}
  \renewcommand{\>}{{\ensuremath\rangle}}

  \newcommand{\Prob}{{\rm Prob}}
  \newcommand{\Dir}{{\rm Dir}}
  \newcommand{\Aut}{{\rm Aut}}
  \newcommand{\cor}{{\rm cor}}
  \newcommand{\corr}{{\rm corr}}
  \newcommand{\cov}{{\rm cov}}
  \newcommand{\sd}{{\rm sd}}
  \newcommand{\tr}{{\rm tr}}
  \newcommand{\Tr}{{\rm Tr}}
  \newcommand{\rank}{{\rm rank}}
  \newcommand{\diag}{{\rm diag}}
  \newcommand{\id}{{\rm id}}
  \newcommand{\Id}{{\rm\bf I}}
  \newcommand{\Gl}{{\rm Gl}}
  \renewcommand{\th}{\ensuremath{{}^\text{th}} }
  \newcommand{\lag}{\mathcal{L}}
  \newcommand{\inn}{\rfloor}
  \newcommand{\lie}{\pounds}
  \newcommand{\longto}{\longrightarrow}
  \newcommand{\speer}{\parbox{0.4ex}{\raisebox{0.8ex}{$\nearrow$}}}
  \renewcommand{\dag}{ {}^\dagger }
  \newcommand{\Ji}{J^\sharp}
  \newcommand{\h}{{}^\star}
  \newcommand{\w}{\wedge}
  \newcommand{\too}{\longrightarrow}
  \newcommand{\oot}{\longleftarrow}
  \newcommand{\To}{\Rightarrow}
  \newcommand{\oT}{\Leftarrow}
  \newcommand{\oTo}{\Leftrightarrow}
  \newcommand{\Too}{\;\Longrightarrow\;}
  \newcommand{\oto}{\leftrightarrow}
  \newcommand{\ot}{\leftarrow}
  \newcommand{\ootoo}{\longleftrightarrow}
  \newcommand{\ow}{\stackrel{\circ}\wedge}
  \newcommand{\feed}{\nonumber \\}
  \newcommand{\comma}{~,\quad}
  \newcommand{\period}{~.\quad}
  \newcommand{\del}{\partial}
%  \newcommand{\quabla}{\Delta}
  \newcommand{\point}{$\bullet~~$}
  \newcommand{\doubletilde}{ ~ \raisebox{0.3ex}{$\widetilde {}$} \raisebox{0.6ex}{$\widetilde {}$} \!\! }
  \newcommand{\topcirc}{\parbox{0ex}{~\raisebox{2.5ex}{${}^\circ$}}}
  \newcommand{\topdot} {\parbox{0ex}{~\raisebox{2.5ex}{$\cdot$}}}
  \newcommand{\topddot} {\parbox{0ex}{~\raisebox{1.3ex}{$\ddot{~}$}}}
  \newcommand{\sym}{\topcirc}
  \newcommand{\tsum}{\textstyle\sum}

  \newcommand{\half}{\ensuremath{\frac{1}{2}}}
  \newcommand{\third}{\ensuremath{\frac{1}{3}}}
  \newcommand{\fourth}{\ensuremath{\frac{1}{4}}}

  \newcommand{\ubar}{\underline}
  %\renewcommand{\vec}{\underline}
  \renewcommand{\vec}{\boldsymbol}
  %\renewcommand{\_}{\underset}
  %\renewcommand{\^}{\overset}
  %\renewcommand{\*}{{\rm\raisebox{-.6ex}{\text{*}}{}}}
  \renewcommand{\*}{\text{\footnotesize\raisebox{-.4ex}{*}{}}}

  \newcommand{\gto}{{\raisebox{.5ex}{${}_\rightarrow$}}}
  \newcommand{\gfrom}{{\raisebox{.5ex}{${}_\leftarrow$}}}
  \newcommand{\gnto}{{\raisebox{.5ex}{${}_\nrightarrow$}}}
  \newcommand{\gnfrom}{{\raisebox{.5ex}{${}_\nleftarrow$}}}

  %\newcommand{\RND}{{\SS}}
  %\newcommand{\IF}{\text{if }}
  %\newcommand{\AND}{\textsc{and }}
  %\newcommand{\OR}{\textsc{or }}
  %\newcommand{\XOR}{\textsc{xor }}
  %\newcommand{\NOT}{\textsc{not }}

  %\newcommand{\argmax}[1]{{\rm arg}\!\max_{#1}}
  %\newcommand{\argmin}[1]{{\rm arg}\!\min_{#1}}
  \DeclareMathOperator*{\argmax}{argmax}
  \DeclareMathOperator*{\argmin}{argmin}
  \DeclareMathOperator{\sign}{sign}
  \DeclareMathOperator{\acos}{acos}
  %\newcommand{\argmax}[1]{\underset{~#1}{\text{argmax}}\;}
  %\newcommand{\argmin}[1]{\underset{~#1}{\text{argmin}}\;}
  \newcommand{\ee}[1]{\ensuremath{\cdot10^{#1}}}
  \newcommand{\sub}[1]{\ensuremath{_{\text{#1}}}}
  \newcommand{\up}[1]{\ensuremath{^{\text{#1}}}}
  \newcommand{\kld}[3][{}]{D_{#1}\big(#2\,\big|\!\big|\,#3\big)}
  %\newcommand{\kld}[2]{D\big(#1:#2\big)}
  \newcommand{\sprod}[2]{\big<#1\,,\,#2\big>}
  \newcommand{\End}{\text{End}}
  \newcommand{\txt}[1]{\quad\text{#1}\quad}
  \newcommand{\Over}[2]{\genfrac{}{}{0pt}{0}{#1}{#2}}
  %\newcommand{\mat}[1]{{\bf #1}}
  \newcommand{\arr}[2]{\hspace*{-.5ex}\begin{array}{#1}#2\end{array}\hspace*{-.5ex}}
  \newcommand{\mat}[3][.9]{
    \renewcommand{\arraystretch}{#1}{\scriptscriptstyle{\left(
      \hspace*{-1ex}\begin{array}{#2}#3\end{array}\hspace*{-1ex}
    \right)}}\renewcommand{\arraystretch}{1.2}
  }
  %\newcommand{\case}[1]{\left\{\arr{ll}{#1}\right.}
  \newcommand{\seq}[1]{\textsf{\<#1\>}}
  \newcommand{\seqq}[1]{\textsf{#1}}
  \newcommand{\floor}[1]{\lfloor#1\rfloor}
  \newcommand{\Exp}[2]{{\rm E}_{#1}\{#2\}} \newcommand{\E}{{\rm E}}

  \newcommand{\href}[1]{}
  \newcommand{\url}[2]{\href{#1}{{\color{blue}#2}}}
  \newcommand{\anchor}[2]{\begin{picture}(0,0)\put(#1){#2}\end{picture}}
  \newcommand{\pagebox}{\begin{picture}(0,0)\put(-3,-23){
    \textcolor[rgb]{.5,1,.5}{\framebox[\textwidth]{\rule[-\textheight]{0pt}{0pt}}}}
    \end{picture}}

  \newcommand{\hide}[1]{
    \begin{list}{}{\leftmargin0ex \rightmargin0ex \topsep0ex \parsep0ex}
       \helvetica{5}{1}{m}{n}
       \renewcommand{\section}{\par SECTION: }
       \renewcommand{\subsection}{\par SUBSECTION: }
       \item[$~~\blacktriangleright$]
       #1%$\blacktriangleleft~~$
       \message{^^JHIDE--Warning!^^J}
    \end{list}
  }
  %\newcommand{\hide}[1]{{\tt[hide:~}{\footnotesize\sf #1}{\tt]}\message{^^JHIDE--Warning!^^J}}
  \newcommand{\Hide}{\renewcommand{\hide}[1]{\message{^^JHIDE--Warning (hidden)!^^J}}}
  \newcommand{\HIDE}{\renewcommand{\hide}[1]{}}
  \newcommand{\fullhide}[1]{}
  \newcommand{\todo}[1]{{\tt[TODO: #1]}\message{^^JTODO--Warning: #1^^J}}
  \newcommand{\Todo}{\renewcommand{\todo}[1]{\message{^^JTODO--Warning (hidden)!^^J}}}
  %\renewcommand{\title}[1]{\renewcommand{\thetitle}{#1}}
  \newcommand{\header}{\begin{document}\mytitle\cleardefs}
  \newcommand{\contents}{{\tableofcontents}\renewcommand{\contents}{}}
  \newcommand{\footer}{\small\bibliography{bibs}\end{document}}
  \newcommand{\widepaper}{\usepackage{geometry}\geometry{a4paper,hdivide={25mm,*,25mm},vdivide={25mm,*,25mm}}}
  \newcommand{\moviex}[2]{\movie[externalviewer]{#1}{#2}} %\pdflatex\usepackage{multimedia}
  \newcommand{\rbox}[1]{\fboxrule2mm\fcolorbox[rgb]{1,.85,.85}{1,.85,.85}{#1}}
  \newcommand{\redbox}[2]{\fboxrule2mm\fcolorbox[rgb]{1,.7,.7}{1,.7,.7}{\begin{minipage}{#1}\center#2\end{minipage}}}
  \newcommand{\twocol}[5][c]{\begin{minipage}[#1]{#2\columnwidth}#4\end{minipage}\begin{minipage}[#1]{#3\columnwidth}#5\end{minipage}}
  \newcommand{\threecoltext}[7][c]{\begin{minipage}[#1]{#2\textwidth}#5\end{minipage}\begin{minipage}[#1]{#3\textwidth}#6\end{minipage}\begin{minipage}[#1]{#4\textwidth}#7\end{minipage}}
  \newcommand{\threecol}[7][c]{\begin{minipage}[#1]{#2\columnwidth}#5\end{minipage}\begin{minipage}[#1]{#3\columnwidth}#6\end{minipage}\begin{minipage}[#1]{#4\columnwidth}#7\end{minipage}}
  \newcommand{\helvetica}[4]{\setlength{\unitlength}{1pt}\fontsize{#1}{#1}\linespread{#2}\usefont{OT1}{phv}{#3}{#4}}
  \newcommand{\helve}[1]{\helvetica{#1}{1.5}{m}{n}}
  \newcommand{\german}{\usepackage[german]{babel}\usepackage[latin1]{inputenc}}

\newcommand{\norm}[2]{|\!|#1|\!|_{{}_{#2}}}
\newcommand{\expr}[1]{[\hspace{-.2ex}[#1]\hspace{-.2ex}]}

\newcommand{\Jwi}{J^\sharp_W}
\newcommand{\Twi}{T^\sharp_W}
\newcommand{\Jci}{J^\natural_C}
\newcommand{\hJi}{{\bar J}^\sharp}
\renewcommand{\|}{\,|\,}
\renewcommand{\=}{\!=\!}
\newcommand{\myminus}{{\hspace*{-.3pt}\text{\rm -}\hspace*{-.5pt}}}
\newcommand{\myplus}{{\hspace*{-.3pt}\text{\rm +}\hspace*{-.5pt}}}
\newcommand{\1}{{\myminus1}}
\newcommand{\2}{{\myminus2}}
\newcommand{\3}{{\myminus3}}
\newcommand{\mT}{{\myminus{}\top}}
\newcommand{\po}{{\myplus1}}
\newcommand{\pt}{{\myplus2}}
%\renewcommand{\-}{\myminus}
%\newcommand{\+}{\myplus}
\renewcommand{\T}{{\!\top\!}}
\renewcommand{\mT}{{\myminus{}\top}}

\newenvironment{centy}{
\vspace{15mm}
\large
\hspace*{5mm}
\begin{minipage}{8cm}\it\color{blue}
}{
\end{minipage}
}

\newcommand{\old}{{\text{old}}}
\newcommand{\new}{{\text{new}}}


\newcommand{\rf}{{\text{ref}}}
\newcommand{\eig}{{\text{eig}}}
\newcommand{\bJ}{{\mathbf{J}}}
\newcommand{\bh}{{\mathbf{h}}}
\newcommand{\bH}{{\mathbf{H}}}
\newcommand{\ft}{\text{ft}}

%%%%%%%%%%%%%%%%%%%%%%%%%%%%%%%%%%%%%%%%%%%%%%%%%%%%%%%%%%%%%%%%%%%%%%%%%%%%%%%%
\pdflatex
\usepackage{fancyvrb}
\DefineShortVerb{\@}
\fvset{numbers=none,xleftmargin=5ex,fontsize=\footnotesize}

\title{Guide to the MLR Code}
\author{M Toussaint}



\begin{document}
\maketitle

{\footnotesize\tableofcontents

}

%%%%%%%%%%%%%%%%%%%%%%%%%%%%%%%%%%%%%%%%%%%%%%%%%%%%%%%%%%%%%%%%%%%%%%%%%%%%%%%%

\section{Linear Algebra}

I spare the description of the @mlr::Array@ class -- it is a standard
tensor storage. It's implementation and interface is very similar to
Octave's Array class.

Instead I just provide my recommentation for C++ operator overloading
for easy linear algebra syntax.

\noindent
\begin{tabular}{|p{.25\columnwidth}|p{.4\columnwidth}|p{.4\columnwidth}|}
\hline
& Tensor/Matlab notation
& C++
\\
\hline
``inner'' product\footnotemark[1]
& $C_{ijl} = \sum_k A_{ijk} B_{kl}$
& @A * B@
\\
index-wise product\footnotemark[2]
& $c_i = a_i b_i$ ~or~ $c=a\circ b$
& @a % b@
\\
& $C_{ijkl} = A_{ijk} B_{kl}$
& @A % B@
\\
diag
& $\diag(a)$
& @diag(a)@
\\
& $\diag(a) B$ ~or~ $c_i = a_i B_{ij}$
& @a % B@ ~or~ @diag(a) * B@
\\
element-wise product
& $c_i = a_i b_i$ ~or~ $c=a\circ b$
& @a % b@
\\
& $C_{ij} = A_{ij} B_{ij}$  ~or~ $C=A\circ B$
& no operator-overload!\footnotemark[3]\newline @elemWiseProduct(A,B)@
\\
outer product
& $C_{ijklm} = A_{ijk} B_{lm}$
& @A ^ B@
\\
& $a b^\T$ ~ (vectors)
& @a ^ b@
\\
transpose
& $A_{ij} = B_{ji}$
& @A = ~B@
\\
inverse
& $A B^\1 C$
& @A*(1/B)*C@
\\
& $A^\1 b$ ~ (or @A\b@ in Matlab)
& @A|b@ %(\footnotemark[4])
\\
\hline
element \emph{reference}\footnotemark[5]
& $A_{103}$
& @A(1,0,3)@
\\
& $A_{(n-2)03}$
& @A(-2,0,3)@
\\
sub-\emph{references!}\footnotemark[6]
& $x_i=A_{2i}$
& @A(2,{})@ ~or~ @A[2]@
\\
& $C_{i} = A_{20i}$ ~or~ @C=A[2,0,:]@
& @A(2,0,{})@ ~ (\footnotemark[7])
\\
&  $C_{ijk} = A_{20ijk}$
& @A(2,0,{})@ ~ {\tiny (trailing @{},{}..@ are implicit)}
\\
sub-refercing ranges\footnotemark[8]
& @C=A[2:4,:,:]@
& @A({2,4})@ ~ {\tiny (trailing @{},{}..@ are implicit)}
\\
& @A[2,1:3,:]@
& @A(2,{1,3})@
\\
\hline
sub-\emph{copies}\footnotemark[9]
& @A(1:3, :, 5:)@
& @A.sub(1,3, 0,-1, 5,-1)@
\\
sub-selected-copies
& @A[{1,3,4},:,{2,3},2:5]@
& @A.sub({1,3,4}, 0,-1, {2,3}, 2,5)@
\\
\hline
sub-assignment\footnotemark[10]
& @A[4:6, 2:5] = B@ ~ ($B\in\RRR^{3\times 4}$)
& @A.setBlock(B, 4, 2)@
\\
& @x[4:6] = b@ ~ ($b\in\RRR^3$)
& @x.setBlock(b, 4)@ ~or~ @x({4,6}) = b@
\\
\hline
initialization
& @A=[1 2 3]'@
& @arr A={1.,2,3}@ ~or~ @arr A(3, {1.,2,3})@
\\
& @A=[1 2 3]@
& @arr A=~arr({1.,2,3})@ ~or~ @arr A(1, 3, {1.,2,3})@
\\
& @A=[1 2; 3 4]@
& @arr A(2,2, {1.,2,3,4})@
\\
\hline
concatenation
& $(x^\T,y^\T,z^\T)^\T$ ~ (stacked vectors)
& @(x,y,z)@
\\
& @cat(1,A,B)@ ~ (stacked matrices)
& @(A,B)@ ~ (memory serial)
\\
\hline
\end{tabular}
\footnotetext[1]{The word ``inner'' product should,
  strictly, be used to refer to a general 2-form $\<\cdot,\cdot\>$
  which, depending on coordinates, may have a non-Euclidean metric
  tensor. However, here we use it in the sense of ``assuming Euclidean
  metric''.}
\footnotetext[2]{For matrices or tensors, this is \emph{not} the
  element-wise (Hadamard) product!}
\footnotetext[3]{The elem-wize product for
  matricies/tensors is much less used within equations than what I call 'index-wise product'.}
%% \footnotetext[4]{Caution! This is notation is somewhat awkward -- it calls a special solver for $A^\1 b$
%%   instead of computing $A^\1$ separately.}
\footnotetext[5]{Negative indices are always interpreted as $n-index$. As
we start indexing from 0, the index $n$ is already out of range. $n-1$
is the last entry. An index of -1 therefore means 'last'.}
\footnotetext[6]{In C++, assuming the last index to be
  memory aligned, sub-referencing is only efficient w.r.t.\ major
  indices: the reference then points to the same memory as the parent
  tensor.}
\footnotetext[7]{As a counter example, \texttt{A[:,0,2]} could be referenced in a memory aligned manner. It can only be copied with \texttt{A.sub(0,-1,0,0,2,2)}}
\footnotetext[8]{again, this can only be ranges
  w.r.t.\ the major index, to ensure memory alignment}
%% \footnotetext[1]{Note: The same for \emph{copying} would be
%%   @A.sub(2,4,0,-1,0,-1)@ resp.\ @A.sub(2,2,1,3,0,-1)@ }
\footnotetext[9]{this is a $\RRR^{3\times ...}$
  matrix: The '3' is \emph{included} in the range.}
\footnotetext[10]{As the assignments are not memory-aligned,
  they can't be done with returned references.}

\section{Graph}

Our graph syntax is a bit different to standard conventions. Actually,
our graph could be called a \emph{key-value hierarchical hyper graph}:
nodes can play the role of normal nodes, or hypernodes (=edges or
factors/cliques) that connect other nodes. Every node also has a set
of keys (or tags, to retrieve the node by name) and a typed value
(every node can be of a different type). This value can also be a
graph, allowing to represent hierarchies of graphs and subgraphs.
\begin{itemize}
\item A graph is a set of nodes
\item Every node has three properties:
\begin{items}
\item A tuple of \textbf{keys} (=strings)
\item A tuple of \textbf{parents} (=references to other nodes)
\item A typed \textbf{value} (the type may differ for every node)
\end{items}
\end{itemize}
Therefore, depending on the use case, such a graph could represent
just a key-value list, an 'any-type' container (container of things of
varying types), a normal graph, a hierarchical graph, or an xml data
structure.

We use the graph in particular also to define a generic file format,
which we use for configuration (parameter) files, files that define
robot kinematic and geometry, or any other structured data. This ascii
file format of a graph helps to also understand the data
structure. Here is the @example.g@ from @test/Core/graph@:
\begin{shaded}
\begin{Verbatim}[fontfamily=courier,fontsize=\tiny]
## a trivial graph
x            # a 'vertex': key=x, value=true, parents=none
y            # another 'vertex': key=y, value=true, parents=none
(x y)        # an 'edge': key=none, value=true, parents=x y
x            # key=x, value=true, parents=none
y            # key=y, value=true, parents=none
(x y)        # key=none, value=true, parents=x y
(-1 -2)      # key=none, value=true, parents=the previous and the y-node

## a bit more verbose graph
node A{ color=blue }         # keys=node A, value=<Graph>, parents=none
node B{ color=red, value=5 } # keys=node B, value=<Graph>, parents=none
edge C(A,B){ width=2 }       # keys=edge C, value=<Graph>, parents=A B
hyperedge(A B C) = 5         # keys=hyperedge, value=5, parents=A B C

## standard value types
a=string      # MT::String (except for keywords 'true' and 'false' and 'Mod' and 'Include')
b="STRING"    # MT::String (does not require a '=')
c='file.txt'  # MT::FileToken (does not require a '=')
d=-0.1234     # double
e=[1 2 3 0.5] # MT::arr (does not require a '=')
#f=(c d e)    # DEPRECATED!! MT::Array<*Node> (list of other nodes in the Graph)
g!            # bool (default: true, !means false)
h=true        # bool
i=false       # bool
j={ a=0 }     # sub-Graph (special: does not require a '=')

## parsing: = {..} (..) , and \n are separators for parsing key-value-pairs
b0=false b1 b2, b3() b4   # 4 boolean nodes with keys 'b0', 'b1 b2', 'b3', 'b4'
k={ a, b=0.2 x="hallo"     # sub-Graph with 6 nodes
  y
  z()=filename.org x }

## special Node Keys

## editing: after reading all nodes, the Graph takes all Edit nodes, deletes the Edit tag, and calls a edit()
## this example will modify/append the respective attributes of k
Edit k { y=false, z=otherString, b=7, c=newAttribute }

## including
Include = 'example_include.g'   # first creates a normal FileToken node then opens and includes the file directly

## any types
#trans=<T t(10 0 0)>  # 'T' is the tag for an arbitrary type (here an ors::Transformation)
                      # which was registered somewhere in the code using the registry()
                      # (does not require a '=')

## strange notations
a()       # key=a, value=true, parents=none
()        # key=none, value=true, parents=none
[1 2 3 4] # key=none, value=MT::arr, parents=none
\end{Verbatim}
\end{shaded}

A special case is when a node has a Graph-typed value. This is
considered a \textbf{subgraph}. Subgraphs are sometimes handled
special: their nodes can have parents from the containing graph, or
from other subgraphs of the containing graph. Some methods of the
@Graph@ class (to find nodes by key or type) allow to specify whether
also nodes in subgraphs or parentgraphs are to be searched.

\section{Logic}

We represent everything, a full knowledge base (KB), as a graph:
\begin{itemize}
\item Symbols (both, constants and predicate symbols) are nil-valued
  nodes. We assume that they are declared in the root scope of the
  graph
\item A grounded literal is a tuple of symbols, for instance
  @(on box1 box2)@. Note that we can equally write this as
  @(box1 on box2)@. There is no need to have the 'predicate' first. In
  fact, the basic methods do not distinguish between predicate and
  contant symbols.
\item A universal quantification $\forall X$ is represented as a scope
  (=subgraph) which first declares the logic variables as nil-valued
  nodes as the subgraph, then the rest. The rest is typically an
  implication, i.e., a rule. For instance
$$\forall X Y~ p(X, Y) q(Y) \To q(X)$$
 is represented as @{X, Y, { (p X Y) (q Y) } { (q X) }@
where the precondition and postconditions are subgraphs of the
rule-subgraph.
\end{itemize}
Here is how the standard FOL example from Stuart Russell's lecture is represented:
\begin{shaded}
\begin{Verbatim}[fontfamily=courier,fontsize=\tiny]
Constant M1
Constant M2
Constant Nono
Constant America
Constant West

American
Weapon
Sells
Hostile
Criminal
Missile
Owns
Enemy

STATE {
(Owns Nono M1),
(Missile M1),
(Missile M2),
(American West),
(Enemy Nono America)
}

Query { (Criminal West) }

Rule {
x, y, z,
{ (American x) (Weapon y) (Sells x y z) (Hostile z) }
{ (Criminal x) }
}

Rule {
x
{ (Missile x) (Owns Nono x) }
{ (Sells West x Nono) }
}

Rule {
x
{ (Missile x) }
{ (Weapon x) }
}

Rule {
x
{ (Enemy x America) }
{ (Hostile x) }
}
\end{Verbatim}
\end{shaded}

By default all tuples in the graph are boolean-valued with default
value true. In the above example all literals are actually
true-valued. A rule @{X, Y, { (p X Y) (q Y) } { (q X)! }@
means $\forall X Y~ p(X, Y) q(Y) \To \neg q(X)$. If in the KB we only
store true facts, this would 'delete' the fact @(q X)!@ from the KB
(for some $X$).

As nodes of our graph can be of any type, we can represent predicates
of any type, for instance @{X, Y, { (p X Y) (q Y)=3 } { (q X)=4 }@
would let $p(X)$ be double-typed.


\subsection{Methods}

The most important methods are the following:
\begin{itemize}
\item Checking whether \textbf{two facts are equal}. Facts are
  grounded literals. Equality is simply checked by checking if all
  symbols (predicate or constant) in the tuples are equal. Optionally
  (by default), it is also checked if the fact values are equal.
\item Checking whether \textbf{a fact equals a
  literal+substitution}. The literal is a tuple of symbols, some of
  which may be first order variables. All variables must be of the
  same scope (=declared in the same subgraph, in the same rule). The
  substitution is a mapping of these variables to root-level symbols
  (predicate of constant symbols). The methods loops through the
  literal's symbols; whenever a symbol is in the substitution scope it
  replaces it by the substitution; then compares to the fact
  symbol. Optionally (by default) also the value is checked for equality.
\item Check whether \textbf{a fact is directly true in a KB (or
  scope)} (without inference). This uses the graph connectivity to
  quickly find any KB-fact that shares the same symbols; then checks
  these for exact equality.
\item Check whether \textbf{a literal+substitution is directly true in
  a KB} (without inference).
\item Given a single literal with only ONE logic variable, and a KB of facts,
  \textbf{compute the domain} (=possible values of the variable) for the
  literal to be true. If the literal is negated the $D \gets
  D\setminus d$, otherwise $D \gets D \cup d$ if the $d$ is the domain
  for true facts in the KB. [TODO: do this also for multi-variable literals]
\item \textbf{Compute the set of possible substitutions for a
  conjunction of literals} (typically precondition of a rule) to be
  true in a KB.
\item \textbf{Apply a set of 'effect literals'} (RHS of a rule): generate facts
  that are substituted literals
\end{itemize}

Given these methods, forward chaining, or forward simulation (for
MCTS) is straight-forward. 

%%%%%%%%%%%%%%%%%%%%%%%%%%%%%%%%%%%%%%%%%%%%%%%%%%%%%%%%%%%%%%%%%%%%%%%%%%%%%%%%

\section{Optim}

The optimization code contributes a nice way to represent (structured)
optimization problems, and a few basic solvers for constrained and
unconstrained, black-box, gradient-, and hessian-available problems.

\subsection{Representing Optimization Problems}

The data structures to represent optimization problems are

A \textbf{standard unconstrained problem}
\begin{align}
\min_x f(x)
\end{align}
\begin{shaded}
\begin{Verbatim}
typedef std::function<double(arr& df, arr& Hf, const arr& x)> ScalarFunction;
\end{Verbatim}
\end{shaded}
The double return value is $f(x)$. The gradient @df@ is returned only
on request (for @df!=NoArr@). The hessian @Hf@ is returned only on
request. If the caller requests @df@ or @Hf@ but the implementation
cannot compute gradient or hessian, the implementation should @HALT@.

These can be implemented using a lambda expression or setting it equal
to a C-function. See the examples. 

A \textbf{sum-of-squares problem} (Gauss-Newton type)
\begin{align}
\min_x y(x)^\T y(x)
\end{align}
\begin{shaded}
\begin{Verbatim}
typedef std::function<void(arr& y, arr& Jy, const arr& x)> VectorFunction;
\end{Verbatim}
\end{shaded}
where the vector @y@ must always be returned. The Jacobian @Jy@ is
returned on request (@Jy!=NoArr@).

A \textbf{constrained problem} (for vector valued functions $f,y,g,h$)
\begin{align}\label{eqConstrained}
\min_x \sum_i f_i(x) + y(x)^\T y(x) \st g(x) \le 0\comma h(x) = 0
\end{align}
Note that we can rewrite this as
\begin{align}
\min_x \sum_{t\in F} \phi_t(x) + \sum_{t\in Y} \phi_t(x)^\T \phi_t(x)
\st \forall_{t\in G}:~ \phi_t(x)\le 0\comma \forall_{t\in H}:~ \phi_t(x)=0 ~,
\end{align}
where the vector-valued \emph{feature} function $\phi$ contains all $f_i, y_i, g_i,
h_i$, and the disjoint partition $F \cup Y \cup G \cup H = \{1,..,T\}$ indicates whether the
$t$-th feature contributes a scalar objective, sum-of-square objective,
inequality constraint or equality constraint. We represent this as
\begin{shaded}
\begin{Verbatim}
enum TermType { noTT=0, fTT, sumOfSqrTT, ineqTT, eqTT };
typedef mlr::Array<TermType> TermTypeA;
struct ConstrainedProblem{
  virtual ~ConstrainedProblem() = default;
  virtual void phi(arr& phi, arr& J, arr& H, TermTypeA& tt, const arr& x) = 0;
};
\end{Verbatim}
\end{shaded}
Here, the returned @phi@ is the feature vector and the returned @tt@
indicates for every @phi@-entry its type
%
(@fTT, sumOfSqrTT, ineqTT, eqTT@). The (on request) returned @J@ is the Jacobian of @phi@. The
(on request) returned @H@ is the Hessian \emph{of the scalar features
  only}, that is, the Hessian of $\sum_i f_i(x)$.

A \textbf{structured constrained problem}: Assume we have $N$ decision
variables $x_i \in \RRR^{d_i}$, each with its own dimensionality
$d_i$. Assume we have $J$ features $\phi_{j=1,..,J}$, but each feature $\phi_j$
depends on only a tuples $X_j \subseteq \{x_1,..,x_N\}$ of variables. We minimize
\begin{align}\label{eqGraphOpt}
\min_{x_{1:N}} \sum_{j\in F} \phi_j(X_j) + \sum_{j\in Y} \phi_j(X_j)^\T \phi_j(X_j)
\st \forall_{j\in G}:~ \phi_j(X_j)\le 0\comma \forall_{j\in H}:~ \phi_j(X_j)=0 ~.
\end{align}
\begin{shaded}
\begin{Verbatim}
struct GraphProblem {
  virtual void getStructure(uintA& variableDimensions, uintAA& featureVariables, TermTypeA& featureTypes) = 0;
  virtual void phi(arr& phi, arrA& J, arrA& H, const arr& x) = 0;
};
\end{Verbatim}
\end{shaded}
Here we decided to provide a method that first allows the optimizer to
query the structure of the problem: return $N$, $d_{i=1,..,N}$, $J$,
$X_{t=1,..,J}$, and @tt@$_{i=1,..,J}$. This allows the optimizer to
setup its own data structures or so. Then, in each iteration the
optimizer only queries @phi(...)@. This always returns the
$J$-dimensional feature vector @phi@, which contains an $f_i$, $y_i$,
$g_i$ or $h_i$-value, depending on @tt(j)@. This @phi(j)@ may only
depend on the decision variables $X_j$. On request it returns the
gradient @J(j)@ of @phi(j)@ w.r.t.\ $X_j$. Note that the
dimensionality of $X_j$ may vary---therefore we return an array of
gradients instead of a Jacobian. On request also a hessian @H(j)@ is
returned for the scalar objectives (when @tt(j)==fTT@).

A \textbf{k-order Markov Optimization} problem. We have $T$ decision
variables $x_{1,..,T}$, each with potentially different dimensionality
$d_{1,..,T}$. We have $J$ features $\phi_{1,..,J}$, each of which may
only depend on $k+1$ consecutive variables $X_j=(x_{t_j-k},..,x_{t_j})$, where
$t_j$ tells us which $k+1$-tuple feature $\phi_j$ depends on. We
minimize again \refeq{eqGraphOpt}. For easier readibility, this is equivalent to a problem of the form:
\begin{align}\label{eqKOMO}
\min_{x_{1:T}}&
\sum_{t=1}^{T} y_t(x_{t-k:t})^\T y_t(x_{t-k:t})
\st
 \forall_t:~ g_t(x_{t-k:t}) \le 0\comma h_t(x_{t-k:t}) = 0 ~,
\end{align}
where the feature with same $t_k=t$ have been collected in different
vector-value functions $y_t, g_t, h_t$.
\begin{shaded}
\begin{Verbatim}
struct KOMO_Problem {
  virtual uint get_k() = 0;
  virtual void getStructure(uintA& variableDimensions, uintA& featureTimes, TermTypeA& featureTypes)=0;
  virtual void phi(arr& phi, arrA& J, arrA& H, TermTypeA& tt, const arr& x) = 0;
};
\end{Verbatim}
\end{shaded}
Here, the structure function returns $N$, $d_{1,..,N}$, $J$, $t_j$,
@tt(j)@.


%% /** NOTE: Why do I define the functions with many arguments to return f, J, and constraints all at once,
%%  * instead of having nice individual methods to return those?
%%  *
%%  * 1) Because I want these problem definitions (classes) to be STATE-LESS. That is, there should not be a set_x(x); before a get_f();
%%  *    I really have bad experience with non-stateless problem definitions.
%%  *
%%  * 2) Because the computation of quantities is expensive and it is usually most efficient to compute all needed quantities together
%%  *    (Instead of calling get_f(x); and then get_J(x); separately)
%%  *
%%  * The methods allow some returns to be optional: use NoArr
%%  *
%%  */

%%%%%%%%%%%%%%%%%%%%%%%%%%%%%%%%%%%%%%%%%%%%%%%%%%%%%%%%%%%%%%%%%%%%%%%%%%%%%%%%

\section{KOMO}

I do not introduce the KOMO concepts here. Read this\footnote{
M. Toussaint: A tutorial on Newton methods for constrained trajectory
optimization and relations to SLAM, Gaussian Process smoothing,
optimal control, and probabilistic inference. In Geometric and
Numerical Foundations of Movements, Springer, 2016.}
\url{http://ipvs.informatik.uni-stuttgart.de/mlr/papers/16-toussaint-Newton.pdf} !

The goal of the implementation is the separation between the code of
optimizers and code to specify motion problems. The problem
form \refeq{eqKOMO} provides the abstraction for that interface. The
optimization methods all assume the general form
\begin{align}\label{eqOpt}
\min_x f(x) \st g(x)\le 0 \comma h(x) = 0
\end{align}
of a non-linear constrained optimization problem, with the additional
assumption that the (approximate) Hessian $\he f(x)$ can be provided
and is semi-pos-def. Therefore, the KOMO code essentially does the
following
\begin{itemize}
\item Provide interfaces to define sets of $k$-order task spaces and
costs/constraints in these task spaces at various time slices; which
constitutes a MotionProblem. Such a MotionProblem definition is very
semantic, referring to the kinematics of the robot.
\item Abstracts and converts a MotionProblem definition into the general
form \refeq{eqKOMO} using a kinematics engine. The resulting
MotionProblemFunction is not semantic anymore and provides the
interface to the generic optimization code.
\item Converts the problem definition \refeq{eqKOMO} into the general
forms \refeq{eqConstrained} and \refeq{eqOpt} using appropriate matrix packings
to exploit the chain structure of the problem. This code does not
refer to any robotics or kinematics anymore.
\item Applies various optimizers. This is generic code.
\end{itemize}

The code introduces specialized matrix packings to exploit the
structure of $J$ and to efficiently compute the banded matrix $J^\T
J$. Note that the rows of $J$ have at most $(k+1)n$ non-zero elements
since a row refers to exactly one task and depends only on one
specific tuple $(x_{t-k},..,x_t)$. Therefore, although $J$ is
generally a $D\times (T+1)n$ matrix (with $D=\sum_t \dim(f_t)$),
each row can be packed to store only $(k+1)n$ non-zero elements. We
introduced a \emph{row-shifted} matrix packing representation for
this. Using specialized methods to compute $J^\T J$ and $J^\T x$ for
any vector $x$ for the row-shifted packing, we can efficiently compute
the banded Hessian and any other terms we need in Gauss-Newton
methods.


\subsection{Formal problem representation}

The following definitions also document the API of the code.
\begin{description}
\item[KinematicEngine] is a mapping $\G:~ x \mapsto \G(x)$
that maps a joint configuration to a data structure $\G(x)$ which
allows to efficiently evaluate task maps. Typically $\G(x)$ stores
the frames of all bodies/shapes/objects and collision
information. More abstractly, $\G(x)$ is any data structure that is
sufficient to define the task maps below.

Note: In the code there is yet no abstraction KinematicEngine. Only
one specific engine (KinematicWorld) is used. It would be
straight-forward to introduce an abstraction for kinematic engines
pin-pointing exactly their role for defining task maps.

\item[TaskMap] is a mapping $\phi:~ (\G_{-k},..,\G_0) \mapsto
(y,J)$ which gets $k+1$ kinematic data structures as input and returns
some vector $y\in\RRR^d$ and (on request) its Jacobian $J\in\RRR(d\times n)$.

\item[Task] is a tuple $c=(\phi, \r_{1:T},
 y^*_{1:T},\textsf{tt})$ where $\phi$ is a TaskMap and the
parameters $\r_{1:T},y^*_{1:T} \in\RRR^{T\times d}$ allow for an
additional linear transformation in each time slice. Here,
$d=\dim(\phi)$ is the dimensionality of the task map. This defines the
transformed task map
\begin{align}
\hat\phi_t(x_{t-k},..,x_t)
& = \diag(\r_t) (\phi(\G(x_{t-k}),..,\G(x_t)) - y^*_t) ~,
\end{align}
which depending on $\textsf{tt}\in\{\textsf{fTT, sumOfSqrTT, ineqT, eqT}\}$ is
interpreted as cost or constraint feature. Note that, in the cost case,
$y^*_{1:T}$ has the semantics of a reference target for the task
variable, and $\r^*_{1:T}$ of a precision. In the code,
$\r_{1:T},y^*_{1:T}$ may optionally be given as $1\times 1$, $1\times
T\po$, $d\times 1$, or $d\times T\po$ matrices---and are interpreted
constant along the missing dimensions.

\item[MotionProblem] is a tuple $(T,\CC,x_{-k+1:0})$ which gives
the number of time steps, a list $\CC=\{c_i\}$ of Tasks, and
a \emph{prefix} $x_{-k:-1} \in\RRR^{k\times n}$. The prefix allows to
evaluate tasks also for time $t\le k$, where the prefix defines the
kinematic configurations $\G(x_{-k+1}),..,\G(x_0)$ at negative
times. This defines the KOMO problem.

\end{description}




\subsection{User Interfaces}

\subsubsection{Easy}

For convenience there is a single high-level method to call the
optimization, defined in @<Motion/komo.h>@
\begin{code}
\begin{verbatim}
/// Return a trajectory that moves the endeffector to a desired target position
arr moveTo(ors::KinematicWorld& world, //in initial state
           ors::Shape& endeff,         //endeffector to be moved
           ors::Shape& target,         //target shape
           byte whichAxesToAlign=0,    //bit coded options to align axes
           uint iterate=1);            //usually the optimization methods may be called just
                                       //once; multiple calls -> safety
\end{verbatim}
\end{code}
The method returns an optimized joint space trajectory so that the
endeff reaches the target. Optionally the optimizer additionaly
aligns some axes between the coordinate frames. This is just one
typical use case; others would include constraining vector-alignments
to zero (orthogonal) instead of +1 (parallel), or directly specifying
quaternions, or using many other existing task maps. See expert
interface.

This interface specifies the relevant coordinate frames by referring
to Shapes. Shapes (@ors::Shape@) are rigidly attached to bodies
(``links'') and usually represent a (convex) collision
mesh/primitive. However, a Shape can also just be a marker frame
(@ShapeType markerST=5@), in which case it is just a convenience to
define reference frames attached to bodies. So, the best way to
determine the geometric parameters of the endeffector and target
(offsets, relative orientations etc) is by transforming the respective
shape frames (@Shape::rel@).

The method uses implicit parameters (grabbed from cfg file or command line or default):
\begin{code}
\begin{verbatim}
  double posPrec = MT::getParameter<double>("KOMO/moveTo/precision", 1e3);
  double colPrec = MT::getParameter<double>("KOMO/moveTo/collisionPrecision", -1e0);
  double margin = MT::getParameter<double>("KOMO/moveTo/collisionMargin", .1);
  double zeroVelPrec = MT::getParameter<double>("KOMO/moveTo/finalVelocityZeroPrecision", 1e1);
  double alignPrec = MT::getParameter<double>("KOMO/moveTo/alignPrecision", 1e3);
\end{verbatim}
\end{code}


\subsubsection{Using a specs file}

Example:
\begin{code}
\begin{verbatim}
KOMO{
  T = 100
  duration = 5
}

Task sqrAccelerations{
  map={ type=qItself }
  order=2    # accelerations (default is 0)
  time=[0 1] # from start to end (default is [0 1])
  type=cost  # squared costs (default is 'cost')
  scale=1    # factor of the map (default is [1])
  target=[0] # offset of the map (default is [0])
}

Task finalHandPosition{
  map={ type=pos ref1=hand ref2=obj vec1=[0 0 .1] }
  time=[1 1] # only final
  type=equal # hard equality constraint
}

Task finalAlignmentPosition{
  map={ type=vecAlign ref1=hand vec1=[1 0 0] vec2=[0 1 0]}
  time=[1 1] # only final
  type=equal # hard equality constraint
  target=[1] # scalar product between vec1@hand and vec2@world shall be 1
}

Task collisions{
  map={ type=collisionIneq margin=0.05 }
  type=inEq # hard inequality constraint
}
\end{verbatim}
\end{code}



\subsubsection{Expert using the included kinematics engine}

See the implementation of @moveTo@! This really is the core guide to
build your own cost functions.

More generically, if the user would like to implement new TaskMaps or
use some of the existing ones:
\begin{itemize}
\item The user can define new $k$-order task maps by instantiating the
abstraction. There exist a number of predefined task maps. The
specification of a task map usually has only a few parameters like
``which endeffector shape(s) are you referring to''. Typically, a good
convention is to define task maps in a way such that \emph{zero} is a
desired state or the constraint boundary, such as relative
coordinates, alignments or orientation. (But that is not necessary,
see the linear transformation below.)

\item To define an optimization problem, the user creates a list of
tasks, where each task is defined by a task map and parameters that
define how the map is interpreted as a) a cost term or b) an inequality
constraint. This interpretation allows: a linear
transformation separately for each $t$ (=setting a reference/target
and precision); how maps imply a constraint. This interpretation has a
significant number of parameters: for each time slice different
targets/precisions could be defined.
\end{itemize}


\subsubsection{Expert with own kinematics engine}

The code needs a data structure $\G(q_t)$ to represent the
(kinematic) state $q_t$, where coordinate frames of all
bodies/shapes/objects have been precomputed so that evaluation of task
maps is fast. Currently this is @KinematicWorld@.

Users that prefer using the own kinematics engine can instantiate the
abstraction. Note that the engine needs to fulfill two roles: it must
have a @setJointState@ method that also precomputes all frames of all
bodies/shapes/objects. And it must be siffucient as argument of your
task map instantiations.

\subsubsection{Optimizers}

The user can also only use the optimizers, directly instantiating the
$k$-order Markov problem abstraction; or, yet a level below, directly
instantiating the @ConstrainedProblem@ abstraction. Examples are given
in @examples/Optim/kOrderMarkov@ and
@examples/Optim/constrained@. Have a look at the specific
implementations of the benchmark problems, esp.\ the
@ParticleAroundWalls@ problem.

\subsubsection{Parameters \& Reporting}

Every run of the code generates a MT.log file, which tells about every
parameter that was internally used. You can overwrite any of these
parameters on command line or in an MT.cfg file.

Inspecting the cost report after an optimization is
important. Currently, the code goes through the task list $\CC$ and
reports for each the costs associated to it. There are also methods to
display the cost arising in the different tasks over time.


%% \subsection{Special Cases}

%% \subsubsection{``True'' dynamics for fully articulated systems}

%% %% penalize $u = M\ddot q + C\dot q + G$

%% %% Our kinematics: no efficient implementation of C!! Approximate $C\dot q
%% %% + G = F$ indep of $\dot q$; and all terms indep of $q$!!

%% \subsubsection{Jerk optimization: 3-order}

%% \subsubsection{Gaussian Process priors: kernel regularization}\label{secKernel}

%% \subsubsection{Inverse Kinematics: 1-order 1-step}

%% \subsubsection{Operational Space Control: 2-order 1-step}

%% \subsubsection{Endpose Optimization: 2-order 1-step}

%% \subsubsection{Multipose Optimization}


%% \subsection{Time Optimization}

\subsection{Potential Improvements}

There is many places the code code be improved (beyond documenting it
better):
\begin{items}
\item The KinematicEngine should be abstracted to allow for easier
plugin of alternative engines.

\item Our kinematics engine uses SWIFT++ for proximity and penetration
computation. The methods would profit enormously from better (faster,
more accurate) proximity engines (signed distance functions, sphere-swept
primitives).
\end{items}

\subsection{Disclaimer}

This document by no means aims to document all aspects of the code,
esp.\ those relating to the used kinematics engine etc. It only tries
to introduce to the concepts and design decisions behind the KOMO
code.

More documentation of optimization and kinematics concepts used in the
code can be drawn from my teaching lectures on Optimization and
Robotics.

%\appendix

%\input{old2}
%\input{old}


\section{Control}


\subsection{Slow and fast control loop}

There are two nested control loops:

In the slow loop ($\sim 50$Hz, non-strict, non-real-time) the
controller has full access to the results of pre-computed
optimizations, full models of the robots kinematics (dynamics?) and
potentially delayed information on the robot (current pose, forces,
contacts, etc). The slow loop may realize computationally complex
things, e.g., operational space control, re-adaptation of a plan
(phase adaptation, recalibration of task maps), model predictive
control, online planning, etc.

The fast loop is 1kHz, strictly and real-time. It has direct access to
the current robot state $q$ (needs to compute $\dot q$ from filtered
differentiation of $q$) as well as the current readouts of the F/T
sensors $u_\ft$. \textbf{We constrain the fast controller to be a linear
regulator in these observables and their integral:}
\begin{align}
e
&\gets \g e + (f^* - J_\ft^\dag u_\ft)
  \quad\text{or}\quad \dot e = (f^*-J_\ft^\dag u_\ft) + (1-\g)e\\
u
&= u_0 + k_p^\text{base} \cdot K_p (q^* - q) + k_d^\text{base} \cdot K_d (\dot q^* - \dot q) + K_I e ~.
\end{align}
This a (redundant) parameterization of a regulator linear in $(q, \dot
q, e)$. We choose this parameterization because $q^*, \dot q^*, f^*$
can be interpreted as ``references''. But actually, we could just drop
them (absorb them in $u_0$) without loosing generality. In addition to
this, the fast loop respects control limits by clipping
$u \gets \texttt{clip}(u, -u_{max}, u_{max}$ element-wise. $u_{max}$
is a constant set in configuration files, not a
fluent. \todo{Additional mechanisms should also in the fast loop
guarantee velocity and joint limits.}

The parameter vectors $k_p^\text{base}$ and $k_d^\text{base}$ are
constants set in the PR2 configuration files. They are hand-tuned so
that setting $K_p=K_d=I$ leads to acceptable (rather low gain)
behavior. The $\cdot$ denotes an element-wise product.

About the integral term: $J_\ft^\dag$ allows us to linearly project the
sensor signals to any other space in which we have a target $f^*$ and
integrate the error.

$K_p, K_d, K_I, J_\ft^\dag$ are arbitrary matrices; $u_0$ an arbitrary
control bias. Therefore, the \textbf{control mode} of the fast loop is
determined by the tuple
\begin{align}
M = (q^*, \dot q^*, f^*, u_0, K_p, K_d, K_I, J_\ft^\dag, \g) ~.
\end{align}
This is the message that the slow loop needs to send to the fast loop
-- the slow loop can change the control mode at any time.

Inversely, the fast loop passes the message
\begin{align}
(q, \dot q, f, u)
\end{align}
to the slow loop, giving it information on the true current state
 $(q, \dot q)$, sensor readings $f$, and computed controls $u$.

The core question therefore is how the slow loop computes the message
$M$ to realize the desired control behaviors. The list of basic desired
control behaviors is:
\begin{enumerate}
\item Follow a pre-computed trajectory $(q_{0:T},\tau)$, where $\tau$
is the time resolution
\item Follow the position-reference that is online computed by a
operational space (or inverse kinematics) controller; the $K_p$ should
such that P-gains can be set/added/removed along \emph{endeffector}
spaces rather than only uniformly configuration space
\item Establish a contact
\item Stabalize a contact force
\item Limit F/T (to avoid breaking a handle)
\item Sliding (moving tangentially) on a surface (or along a DOF like
  an opening door) which is perceived via the F/T signal
\end{enumerate}


\subsection{Operational Space Control: Computing gains by projecting
operational space gains}

The appendix B derived the necessary equations in all generality. In
practise, it is sufficient to modify the $K_p$ only, using the
Jacobian of the desired task space. In equation \refeq{eqProjGains} we have
\begin{align}
\bar K_p = A^\1 J^\T C K_p J \comma A = H + J^\T C J ~,
\end{align}
where we assumed $M=\Id$ (quasi-dynamic model) and no other
tasks. Further, assuming $C=c$ and $K_p=k$ are scalars we have
\begin{align}
\bar K_p = k(H/c+J^\T J)^\1 J^\T J ~.
\end{align}
I actually tested just $k J^\T J$.

TODO: Let @FeedbackController@ really compute these projected PD
behaviors, instead of only $q^*,\dot q^*$! Then all of this is
automatic!



\subsection{Controlling the F/T signal---the \emph{sensed} force}

\subsubsection{Preliminaries: Understanding force transmission}

The following law of force propagation is well known,
\begin{align}\label{eqForce}
u = J^\T f
\end{align}
where $f$ is a force in the endeffector (e.g., the negative
of its gravity load), $J$ the position Jacobian of the endeffector,
and $u$ are the torques ``perceived'' in each of the joints due to the
force $f$. This law is correct only under the assumption that nothing
moves. Inversely, this law is typically used to compensate forces:
Assume you have a load on an endeffector, gravity pulls it down. The
gravity force pulling the load down propagates to torques $u$ in each
joint -- if you want to compensate this torque the motors need to
create the reative torque.

The same also holds for force-torque $f\in\RRR^6$, where the Jacobian
is the stacking of the position and the axial Jacobian.

Typically, $f$ is lower-dimensional than $u$. So, actually, there
should be many $u$ that generate a desired $f^*$? What is the optimal
one? Well, assume $f^*=0$ for a moment. Then, any choice of $u$ will
accelerate the robot (assuming gravity compensation). The only choice
to generate $f^*=0$ and not to accelerate the robot is $u=0$. Equally,
the only choice to generate any $f^*$ without accelerating the robot
is $u = J^\T f^*$.

When we include system dynamics in the equation, we have the general
\begin{align}\label{eqDyn}
u = M \ddot q + h + J^\T f ~.
\end{align}
where $M$ (the inertia matrix) and $h$ (the coreolis and gravity
forces) depend on $(q,\dot q)$. One way to read this equation is: the
torques you ``feel'' in the joints are the reactive torques of the
robot's inertia (that derive the acceleration) plus the torque you
feel from the endeff force $f$.


\subsubsection{Controlling the direct F/T signal with a fixated endeff}

Consider the following exercise: Fix the endeffector rigidly, e.g.\ to
a table with a clamp (Schraubzwinge). Write a controller that
generates any desired $f^*$ in the F/T sensor with the least effort,
and stably, and staying close to a desired homing posture.

If we unrealistically assume that our model is correct then the
solution simply is \refeq{eqDyn}; for $\ddot q =0$ and a
gravity-compensated robot just \refeq{eqForce}; where
\begin{align}
J = J_\ft ~,
\end{align}
which is the position and axial Jacobian of the F/T sensor w.r.t.\ q.

However, this equation \textbf{does not use any F/T sensor feedback}
to generate the desired F/T signal. This cannot work well in practise.
We can resolve this with an I-controller on the F/T signal error.
\begin{align}\label{ctrlInt}
e
&= \int_t dt [f^* - f] \\
u
&= J^\T \a e ~.
\end{align}

The $\a$ here has the meaning of an exponential decay of the signal
error---which we can show assuming the perfect model. Under perfect
model assumption, the F/T sensor measures
\begin{align}
f
&= J^\dag u \comma J^\dag J^\T \equiv \Id \comma J^\dag = (J J^\T)^\1 J \\
& \quad \text{Note: } J^\dag u = J^\dag J^\T f = (J J^\T)^\1 J J^\T f = f \\
\end{align}
Note that $J J^\T$ is a $d\times d$-matrix and invertible and $J^\dag$
the appropriate left-pseudo-inverse of $J^\T$. Inserting
this perfect-model measurement in the control law \refeq{ctrlInt} we
get
\begin{align}
\dot e
&= f^* - J^\dag u \\
\dot u
&= J^\T \a (f^* - J^\dag u) = \a J^\T f^* - \a \underbrace{J^\T
J^\dag}{=\Id} u = \a (J^\T f^* - u)  ~.
\end{align}
Here, $J^\T J^\dag = J^\T (J J^\T)^\1 J^\T$ is actually the
projection that projects any joint torques $u$ into the space that
directly relates to endeffector forces and not to
accelerations. However, if the $u$ was chosen by some law $J^\T f$,
then $u$ will always lie within this projection (will never lead to
accelerations of the robot), and therefore it actually is the identity
matrix.

Now, the above states that $\dot u = \a (J^\T f^* - u)$, which says
that $u$ exponentially approaches the perfect-model correct torque
$J^\T f^*$, which a decay rate $\a$. Therefore, $\a$ can be considered
a decay rate.

\textbf{Open:} What if we have a $\ddot q$ as well? Two possibilities: 1)
Reiterate the above reasoning with $\ddot q$. 2) Just add the signals.


\subsubsection{Control the indirectly sensed contact force of endeff}

Exercise: We have the F/T sensor, but attached to it a hand and a
contact point with some relative transformation to the F/T
sensor. This point is in contact with a table. What we want to control
is the force between point and table, which is just a 1D thing.

This is best addressed by thinking of the F/T sensor as if it was a 6D
joint (like a ball joint). If we have a force $f$ at some
endeffector then we ``feel'' this force in all joints of the robot as
$u = J^\T f$. This includes the F/T sensor joints! So the Jacobian of the
endeffector variable (be it 1D or 3D) w.r.t.\ the sensor pseudo-ball-joint
exactly gives the measurement equation. Let's denote this Jacobian as
\begin{align}
J_\ft \in \RRR^{d\times 6} ~,
\end{align}
where $d=1$ if it is only the distance to the table, or $d=3$ if it is
all forces. Further, let's denote by
\begin{align}
J \in \RRR^{d\times n}
\end{align}
the Jacobian w.r.t.\ all the real robot joints.

As above, the \emph{perceived} endeffector force (this time perceived
by the F/T sensor) is
\begin{align}
u_\ft = J^\T_\ft f \quad\To\quad f = J^\dag_\ft u_\ft ~,
\end{align}
where $u_\ft\in\RRR^6$ is the F/T signal. Again we may use an I-controller
to correct for the error between desired endeffector force $f^*$ and
perceived one:
\begin{align}
\dot e
&= f^* - f
 = f^* - J^\dag_\ft u_\ft \\
u
&= J^\T \a e ~.
\end{align}
Note that the last equation generates joint torques proportional to
the normal endeffector Jacobian $J$ because $e$ is an error in
endeffector force space (not F/T signal space).

This fits to our contoller setup by
\begin{align}
J^\dag_\ft &\gets J^\dag_\ft \comma
f^* \gets f^* \comma
\g  \gets 1 \comma
K_I \gets \a J^\T ~.
\end{align}
When force control is turned off, we need to remember to set $\g=0, e=0$ to
ensure that next time it is turned on again it doesn't blow.

\textbf{Open:} What happens for $\g<1$? Is this equivalent to $\a<1$? Perhaps not.
($\sum_{t=0}^\infty \g^t = \frac{1}{1-\g}$)


%% Given a force $f_e$ at the endeff point, the F/T sensor reads
%% (assuming no motions)
%% \begin{align}
%% f
%% &= (f_e; \tau) \\
%% \tau_i
%% &= r_i \times f_e \\
%% r_i
%% &= (\Id - a_i a_i^\T) (p_e - p_\ft)
%% \end{align}
%% where $\tau$ is the measured force, $a_i$ is the $i$th F/T
%% axis $a_i$ (the $i$th axis around which torque is measured), $r_i$ the lever for $a_i$, that is, $r_i$ is the part of $(p_e - p_\ft)$ that is
%% orthogonal to $a_i$, $p_e$ the position of the endeff point, and
%% $p_\ft$ the origin of the F/T sensor. All this is linear, so let's
%% write
%% \begin{align}
%% \tau = R f_e
%% \end{align}

%% Now, given that we want to control the contact force, we can set a
%% desired $f_e^*$, translate this to the desired $f^*$ and control
%% this. BUT, is this what we want? Because that would control a 6D F/T
%% signal although we originally wanted to control only a 1D contact
%% force???

%% \textbf{this is not good thinking}


\subsubsection{$q$-control under force constraints}

Assume we have a P(I)D controller on $q$---typically a PID in some
task space that has been projected to joint space. We would like to
execute that desired reference behavior but subject to constraints on
the sensed endeffector force
\begin{align}
f_{lo} \le J^\dag_\ft f \le f_{hi} ~.
\end{align}
These are $2 d$ constraints.

As with lagrange parameters, we can simply activate the constraints
when violated: When one of the components violates the constraint,
control the force to be exactly $f_{lo|hi}$. For the latter, use the
$f$-error-integral method as above. This should eventually have higher
priority to any other gains (keep other I-gains limited!).

\subsubsection{I-gains on position?}



\subsection{Technical Details and Issues}

\subsubsection{Ctrl-Message documentation}

One message type for setting the control mode AND feedback from the
controller.

Setting the control model: $(q^*, \dot q^*, f^*, u_0, K_p, K_d,
K_f)$. Can be set any time.

Feedback from the controller: $(q, \dot q, f, u)$. Published with 1kHz.


\subsubsection{Filtering of the differentiation of $q$}

[Peter: please fill in]

\subsection{Enforcing control, velocity, joint and force limits}

Enforcing control limits is really simple: Just clip the computed $u$.

Enforcing velocity limits turned out difficult: The velocity
signal is so noisy, a direct feedback coupling was bad. Also, the
IF-case of velocity-limit-violation turned off and on quickly and
introduced even more noise (rattling motors...)

I have ignored limits totally so far -- should be handled (as
collisions) in the slow loop.

FORCE LIMITS! Not idea how to handle this.

Maybe a route to a more principled approach to all of these: Take the
Augmented Lagrangian way to handle constraints as a template: First
associate only a soft squared penalty with margin penetration. Then
compute/update the respective dual parameters that push you out of the
margin.

%%%%%%%%%%%%%%%%%%%%%%%%%%%%%%%%%%%%%%%%%%%%%%%%%%%%%%%%%%%%%%%%%%%%%%%%%%%%%%%%
\appendix

\subsection{Reference: General OPEN-LOOP ideal contact force controller}

\subsubsection{General case}

This controller is sort-of open loop! It does not take into account any F/T
feedback. What is receives as a specification is $\ddot y^*$ (desired
endeff accel) and $\l^*$ (desired contact force); as well as models of
the maps $\phi, J_\phi, g, J_g$. We discuss below how this can properly
be made a feedback regulator. We write the problem as a general constraint problem
\begin{align}
\min_{u, \ddot q, \l}\quad
 & \norm{u-a}^2_H \\
\st
 & u = M \ddot q + h + J_g^\T \lambda \\
 & J_\phi\ddot q = c \\
 & \l = \l^* \\
 & \color{blue}J_g\ddot q = b
\end{align}
Notes:
\begin{items}
\item The role of $a$ becomes clearer when we treat the blue
constraint below
\item The 2nd constraint relates to an arbitrary task map
$\phi:~ q \mapsto y$ with Jacobian $J_\phi$, $c = \ddot y^* - \dot
J_\phi \dot q$ and some desired task space acceleration $\ddot y^*$.
\item We have a set of functions $g:~ q \to \RRR^m$ with Jacobian
$J_g$ which play the role of inequality constraints
\item The 3rd constraint captures desired contact forces with the constraints
\item The blue constraint expresses that a) assuming the contact is
active there must not be acceleration w.r.t.\ $g$ or b) if the contact
is not active we might want to control the acceleration towards it
(to make it active). (This constrains the dynamics. Without this
constraint the dynamics could assume that the contact forces $\l$ are
generated while the endeff is moving (e.g., by strange external
forces). This constraint makes it consistent.)
\end{items}

To derive closed form solutions, each of these equality constraints
can be handled in two ways: relax it to become a squared penalty
(and then potentially taking the infinite precision limit); or
resolve it.

We resolve the 1st and 3rd constraint, and relax the 2nd and 4th to
later take the limit. The solution is
\begin{align}
\bh
&:= h+J_g^\T\l^*  \\
f(\ddot q)
&=\norm{M\ddot q - (a-\bh)}^2_H
 + \norm{J_g \ddot q - b}^2_B
 + \norm{J_\phi \ddot q - c}^2_C \\
\ddot q^*
&= (M^\T H M + J_g^\T B J_g + J_\phi^\T C J_\phi)^\1 [ M^\T H (a-\bh) + J_g^\T B b + J_\phi^\T C c]
\end{align}
The limit $B\to\infty$:
\begin{align}
\ddot q^*
&= (X + J_g^\T B J_g)^\1 [ J_g^\T B b + x ] \\
&= J_g{}^\#_{XB} b + (\Id - J_g{}^\#_{XB} J_g)~ (X^\1 x)
\end{align}
And note that $(X^\1 x)$ is the solution to only having the other
terms.
% Because $J_g^\T \l^*$ is certainly not in the nullspace of
%$J_g$ one can neglect it (in the limit $B\to\infty$) ---
Given $\ddot q^*$, the optimal control is computed as $u=M\ddot q + h
+ J_g^\T \l^*$. We still did not take the limit of the $C$-term
(endeffector position control). We could using the hierarchical limit
case.

\subsection{Reference: Pullback of operational space linear controllers}

The above assumes that at any instance in time we want a certain
task-space acceleration $\ddot y^*$ and translates this to an optimal
joint control in that instant in time. If we want to implement a
certain feedback behavior in the task space, that is, we have a
desired feedback control law $\pi:~ y,\dot y \mapsto \ddot y$, we can
evaluate $\pi$ at every point in time and project to operational space
control.

\begin{align}
\ddot y
 &= \ddot \phi(q) = \ddot (J q) = \dot (\dot J q + J \dot q) = 2 \dot J \dot q + J \ddot q \\
\ddot y^*
 &= K_p y + K_d \dot y + k \\
J \ddot q
 &\overset{!}= c = \ddot y^* - 2 \dot J \dot q
  = K_p y + K_d \dot y + k - 2 \dot J \dot q \\
 &\approx K_p (J(q-q_0)+\phi(q_0)) + K_d J \dot q + k \\
 &= K_p J q + K_d J \dot q + k' \comma k' = k + K_p (\phi(q_0)-J q_0) \\
\ddot q^*
 &= A^\1 [... + J^\T C c] = A^\1 [... + J^\T C (K_p J q + K_d J \dot q + k')] \\
 &= A^\1 [...] + A^\1 J^\T C K_p J q + A^\1 J^\T C K_d J \dot q + A^\1 J^\T C k' \\
 &= \bar K_p q + \bar K_d \dot q + \bar k \comma
\bar k =  A^\1 [...] + A^\1 J^\T C k'\comma \bar K_p = A^\1 J^\T C K_p
 J\comma \bar K_d = A^\1 J^\T C K_d J \label{eqProjGains}
\end{align}


\subsection{How to make this FEEDBACK?}

W.r.t.\ $y$ (endeff pos) it is clear how to make this feedback: We can
impose a PD behavior on the endeffector
$$ \ddot y^* = k_p (y^* - y) + k_d (\dot y^*-\dot y) $$
and send this desired endeff accel to the general controller.

What about $\l^*$??

\subsection{What do we want?}

\begin{description}
\item[desired task space acceleration law]
\begin{align}
\ddot y^*
 &= \ddot y^\rf + K_p(y^\rf-y) + K_d (\dot y^\rf -\dot y) + K_{Ip} \int (y^\rf-y) + K_{Id} \int (\dot y^\rf -\dot y)
\end{align}
That defines a desired \emph{acceleration}. But if the system was
precise in enforcing this acceleration it would be
non-compliant. Note: strictly speaking, if this law says $\ddot y^*=0$
(e.g., because $K_p$ and $K_d$ are zero), then this is a strict
(non-compliant) statement.

\item[precision/compliance]
Given a desried $\ddot y^*$, the precision along some
dimensions may not that important. We may capture this with the
precision (or compliance) matrix $C$. As a convention, let the
$\eig(C)\in[0,1]$, and an eigenvalue of $1$ states full precision,
while an eigen value of $0$ states full compliance.

This implies an objective term
$$\norm{J_\phi \ddot q - \ddot y^*}^2_C$$

\item[Null cost reference] Typically one defines control costs
  $\norm{u}^2_H = \norm{M \ddot q + F}^2_H$. However, this becomes
  semantically tricky. When defining what is 'desired' I propose to
  stay on the level of accelerations. So we have a desired ('null')
  acceleration law
\begin{align}
\ddot q_0^*
 &= \ddot q_0^\rf + K^q_p (q_0^\rf-q) + K^q_d (\dot q_0^\rf -\dot q)
\end{align}
and consider costs
$$\norm{\ddot q - \ddot q_0^*}^2_H ~.$$ Note that, by appropriate
choices of parameters, the typical control cost can be
mimicked. However, the semantics is somewhat different. For instance,
setting $K^q_d$ and $\dot q_0^\rf$ implies that we want to damp
motion, and choosing controls $u$ that implement this daming are at
\emph{null costs}. Equally, for non-zero $F$ and $\ddot q_0^*=0$,
applying controls that ensure zero acceleration are at \emph{null
  costs}. That's we I call it \emph{null cost reference}. This is
rather different to generally penalize $\norm{u}^2_H$, which would
imply costs for any controls $u$, even if they just implement
counteracting $F$ of generating the null reference.

\item[Optimal acceleration law] We compute the optimal acceperation
  $\ddot q^*$ in its 1st order Taylor approximation w.r.t.\ $q$ and
  $\dot q$:
\begin{align}
\ddot y^*
 &= \ddot y^\rf + K_p(y^\rf-y) + K_d (\dot y^\rf -\dot y) \\
 &\approx \ddot y^\rf + K_p y^\rf - K_p (J(q-q_0)+y_0) + K_d \dot y^\rf - K_d J \dot q \\
 &= k - K_p J q - K_d J \dot q \comma k = \ddot y^\rf + K_p y^\rf+ K_d \dot y^\rf + K_p (J q_0 - y_0) \\
\ddot q_0^*
 &= k^q - K^q_p q - K^q_d \dot q
\comma k^q = \ddot q_0^\rf + K^q_p q_0^\rf+ K^q_d \dot q_0^\rf \\
\ddot q^*
 &= \argmin_{\ddot q} \norm{\ddot q - \ddot q_0^*}^2_H + \norm{J \ddot q - \ddot y^*}^2_C \\
 &= (H + J^\T C J)^\1 [H \ddot q_0^* + J^\T C \ddot y^*] \\
 &\approx
 \bar k - \bar K_p q - \bar K_d \dot q \\
 &= \bar K_p (q_\rf - q) - \bar K_d \dot q \comma q_\rf = \bar K_p^\1 \bar k  \label{eqProjGains}\\
\bar K_p
 &= (H + J^\T C J)^\1 [H K^q_p + J^\T C K_p J] \\
\bar K_d
 &= (H + J^\T C J)^\1 [H K^q_d + J^\T C K_d J] \\
\bar k
 &= ~ (H + J^\T C J)^\1 [H k^q + J^\T C k]
%% \ddot q^*
%%  &= \argmin_{\ddot q} \norm{J \ddot q - \ddot y^*}^2_C \\
%% \ddot q^*
%%  &= (J^\T C J)^\1 [J^\T C \ddot y^*] \\
%%  &\approx (J^\T C J)^\1 [J^\T C (k - K_p J q - K_d J \dot q)] \\
%%  &= \bar k - \bar K_p q - \bar K_d \dot q  \label{eqProjGains}\\
%% \bar K_p
%%  &= (J^\T C J)^\1 J^\T C K_p J \\
%% \bar K_d
%%  &= (J^\T C J)^\1 J^\T C K_d J \\
%% \bar k
%%  &= ~ (J^\T C J)^\1 J^\T C k
\end{align}

%% $$\mat{cc}{H & 0 \\ 0 & C} \gets C \comma \mat{c}{\Id \\ J} \gets
%% J ~.$$

\item[Transfer to controls] Given an optimal acceleration law and the
  system dynamics, we choose:
\begin{align}
\ddot q^*
 &= \bar K_p q + \bar K_d \dot q + \bar k \\
u^*
 &= M \ddot q^* + F \\
 &= M \bar K_p q + M \bar K_d \dot q + M \bar k + F
\end{align}
Note again, only here, the system dynamics enter. The specification of
the optimal acceleration law is independent of the dynamics. (Unless
$H$ and $\ddot q^\rf$ are chosen in relation to $M$ and $F$ to mimick
typical control costs---but we explicitly avoid this to make desired
system behavior somewhat less dependent on (possibly inaccurate)
dynamics models).


\item[Error correction of system dynamics]

The optimal acceleration law gives an explicit desired
acceleration. We may estimate the control error by a low pass filter
on the acceleration errors. There are two options:

Let $\<\ddot q^*\>$ be a low pass filter of the desired accellerations
$\ddot q^*(q,\dot q)$; and $\<\ddot q\>$ a low pass of the actual true
accelerations. We may define $g = \<\ddot q\> - \<\ddot q^*\>$ and
control
$$u=M (\ddot q^* + g) + F ~.$$

Or we may assume system dynamics
\begin{align}
u=M \ddot q + F + g
\end{align}
for some unknown and variable (slowly changing) $g\in\RRR^n$ which
reflects constant loads on the joints. We may estimate $g$ as a
low-pass filter,
\begin{align}
g
&= \< u - M \ddot q - F \>_\text{low pass}
 = \< u \> - M \<\ddot q\> - F \\
&= \<M \ddot q^* + F + g\> - M \<\ddot q\> - F \\
&= M~ [ \< \ddot q^* + g\> - \<\ddot q\>]
\end{align}
which is puzzlingly different to the above. ($M$ is obvious, but the rest?)

\item[Compliant error correction of system dynamics]

The above describes a scheme that corrects any errors from the desired
acceleration. However, in the case of contact, and desired compliance,
we do not \emph{want} to enforce exact reference following along
certain dimensions. E.g., in the case of a contact we systematically
do not accelerate towards, leading to a systematic error in the system
equations, an adaptation of $g$, and perhaps divergence.

\item[Error correction on task space level]

Let $\<\ddot y^*\>$ be a low pass filter of the desired task space
accellerations and $\<\ddot y\>$ a low pass of the actual true
accelerations. We may define $g=\<\ddot y\> - \<\ddot y^*\>$ and add
$g$ to the $\ddot y^\rf$.

Alternatively, we may define $g$ as integral error in the task space
and add it to $\ddot y^\rf$.

In both cases, $g$ adds to $k$, showing that it adds to $\ddot q^*$
as $(H + J^\T C J)^\1 J^\T C g$.

The matrix $C$ controls compliance.



\item[Measured-force limits]

\item[Limit Energy]

\end{description}





\end{document}


\end{document}

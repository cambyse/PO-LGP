\documentclass{article}

\author{Johannes Kulick}
\title{Joint Dependency}

\usepackage[minnames=1, maxnames=2, maxbibnames=99, giveninits=true, backend=biber, style=authoryear-comp, dashed=false, uniquename=false, doi=false, isbn=false, url=false]{biblatex}
\addbibresource{abbrv.bib}
\addbibresource{references.bib}
\usepackage{hyperref}
\usepackage{fontspec}

\begin{document}
\maketitle

\section{Intro}
This is part of the documentation of my software artifacts developed during my
PhD time at the MLR lab. I hope it might help others to use my software for
their research.

The joint dependency code was used for several experiments to learn about the
dependency structure of joints, most notably the experiments from
\textcite{kulick_active_2015}, but also \textcite{kulick_robots_2015,
bernstein_opening_2017}. It uses the maximum cross entropy method developed by
\textcite{kulick_advantage_2015}.

\section{Package}

The main repository is at my github account:
\url{https://github.com/hildensia/joint_dependency}, a mirror is
  at our own gitlab:
  \url{https://animal.informatik.uni-stuttgart.de/johannes.kulick/joint_dependency}
  (but be aware that I might update the github version but not the gitlab
  version). The data from the ICRA paper \parencite{kulick_active_2015} are at
  \url{https://github.com/hildensia/ICRA2015_Experimental_Data}.

\subsection{Requirements}
The package needs python, scipy ($>=0.15$), pandas, numpy, blessings, progressbar,
cython, and my own \verb+bayesian_changepoint_detection+ (see accompanying
documentation in this folder).

\section{HowTo}
To get the code up and running use \verb+pip install -e .+ and then run the
\verb+experiment.py+ in \verb+joint_dependencies+.

To understand, what's going on start in the \verb+run_experiment()+ function in that
same file. It sets up a simulation (or the connection to the robot using ros) al
the belief and runs the necessary exploration algorithms iteratively.

If you want to change the world the code is run look at the
\verb+create_lockbox()+
function. That creates a set of joints corresponding to the joints on the
physical lockbox.

\verb+get_best_point()+ does the actual Monte-Carlo optimization. To change the
sampling of the actions (which can indeed have a huge impact on the overall
performance) look into the \verb+action_sampling_fnc+ here.

The exploration measures are in \verb+inference_cy.pyx+ (and use cython to speed
it up, so feel free to spice up your code with cython...). The actual interface
to those methods changed quite a bit over time, so feel free to add parameters
to it, but be sure that you change the signature of all those functions.
\printbibliography[heading=bibintoc]
\end{document}

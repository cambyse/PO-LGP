\documentclass{article}

\author{Johannes Kulick}
\title{Bayesian Changepoint Detection}

\usepackage[minnames=1, maxnames=2, maxbibnames=99, giveninits=true, backend=biber, style=authoryear-comp, dashed=false, uniquename=false, doi=false, isbn=false, url=false]{biblatex}
\addbibresource{abbrv.bib}
\addbibresource{references.bib}
\usepackage{hyperref}

\begin{document}
\maketitle

\section{Intro}
This is part of the documentation of my software artifacts developed during my
PhD time at the MLR lab. I hope it might help others to use my software for
their research.

The Bayesian changepoint detection code is useful to calculate probabilities of
a change of the underlying model in time series data. It implements the
algorithms from \textcite{fearnhead_exact_2006, adams_bayesian_2007}. E.g., in my joint
dependency \parencite{kulick_active_2015} experiments I used it to get change points in the
friction of joints to find possible unlocking states of e.g. keys.

\section{Package}

The main repository is at my github account:
\url{https://github.com/hildensia/bayesian_changepoint_detection}, a mirror is
  at our own gitlab:
  \url{https://animal.informatik.uni-stuttgart.de/johannes.kulick/bayesian_changepoint_detection}
  (but be aware that I might update the github version but not the gitlab
  version).

\subsection{Requirements}
The package needs python, scipy and numpy. 

\section{HowTo}
The best way to get started is to have a look at the example jupyter notebook at
\url{http://nbviewer.ipython.org/urls/raw.githubusercontent.com/hildensia/bayesian_changepoint_detection/master/Example%20Code.ipynb?create=1}


\printbibliography[heading=bibintoc]
\end{document}

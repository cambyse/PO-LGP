\documentclass[10pt,fleqn,twoside]{article}
\usepackage{palatino}
\usepackage{amsmath}
\usepackage{amssymb}
\usepackage{amsfonts}
\usepackage{amsthm}
\usepackage{eucal}
\usepackage{graphicx}
\usepackage{color}

\usepackage[round]{natbib}
\bibliographystyle{abbrvnat}
%\usepackage[german]{babel}
%\usepackage[utf8]{inputenc}

\graphicspath{{pics/}{figs/}{~/write/tex/pics/}{~/write/tex/figs/}{~/teaching/pics-all/}}
\usepackage{geometry}
\geometry{a4paper,hdivide={35mm,*,35mm},vdivide={35mm,*,35mm}}
\renewcommand{\baselinestretch}{1.1}

\newenvironment{items}{
\par\small
\begin{list}{--}{\leftmargin4ex \rightmargin0ex \labelsep1ex \labelwidth2ex
\topsep0pt \parsep0ex \itemsep3pt}
}{
\end{list}
}


  \renewcommand{\a}{\ensuremath\alpha}
  \renewcommand{\b}{\beta}
  \renewcommand{\c}{\gamma}
  \renewcommand{\d}{\delta}
    \newcommand{\D}{\Delta}
    \newcommand{\e}{\epsilon}
    \newcommand{\g}{\gamma}
    \newcommand{\G}{\Gamma}
  \renewcommand{\l}{\lambda}
  \renewcommand{\L}{\Lambda}
    \newcommand{\m}{\mu}
    \newcommand{\n}{\nu}
    \newcommand{\N}{\nabla}
  \renewcommand{\k}{\kappa}
  \renewcommand{\o}{\omega}
  \renewcommand{\O}{\Omega}
    \newcommand{\p}{\phi}
    \newcommand{\ph}{\varphi}
  \renewcommand{\P}{\Phi}
  \renewcommand{\r}{\varrho}
    \newcommand{\s}{\sigma}
  \renewcommand{\S}{\Sigma}
  \renewcommand{\t}{\theta}
    \newcommand{\T}{\Theta}
  %\renewcommand{\v}{\vartheta}
    \newcommand{\x}{\xi}
    \newcommand{\X}{\Xi}
    \newcommand{\Y}{\Upsilon}

  \renewcommand{\AA}{{\cal A}}
    \newcommand{\BB}{{\cal B}}
    \newcommand{\CC}{{\cal C}}
    \newcommand{\EE}{{\cal E}}
    \newcommand{\FF}{{\cal F}}
    \newcommand{\GG}{{\cal G}}
    \newcommand{\HH}{{\cal H}}
    \newcommand{\II}{{\cal I}}
    \newcommand{\KK}{{\cal K}}
    \newcommand{\LL}{{\cal L}}
    \newcommand{\MM}{{\cal M}}
    \newcommand{\NN}{{\cal N}}
    \newcommand{\oNN}{\overline\NN}
    \newcommand{\OO}{{\cal O}}
    \newcommand{\PP}{{\cal P}}
    \newcommand{\QQ}{{\cal Q}}
    \newcommand{\RR}{{\cal R}}
  \renewcommand{\SS}{{\cal S}}
    \newcommand{\TT}{{\cal T}}
    \newcommand{\uu}{{\cal u}}
    \newcommand{\UU}{{\cal U}}
    \newcommand{\VV}{{\cal V}}
    \newcommand{\XX}{{\cal X}}
    \newcommand{\YY}{{\cal Y}}
    \newcommand{\SOSO}{{\cal SO}}
    \newcommand{\GLGL}{{\cal GL}}

    \newcommand{\Ee}{{\rm E}}

  \newcommand{\NNN}{{\mathbb{N}}}
  \newcommand{\ZZZ}{{\mathbb{Z}}}
  %\newcommand{\RRR}{{\mathrm{I\!R}}}
  \newcommand{\RRR}{{\mathbb{R}}}
  \newcommand{\CCC}{{\mathbb{C}}}
  \newcommand{\one}{{{\bf 1}}}
  \newcommand{\eee}{\text{e}}

  \newcommand{\NNNN}{{\overline{\cal N}}}

  \renewcommand{\[}{\Big[}
  \renewcommand{\]}{\Big]}
  \renewcommand{\(}{\Big(}
  \renewcommand{\)}{\Big)}
  \renewcommand{\|}{\,|\,}
  \renewcommand{\=}{\!=\!}
    \newcommand{\<}{{\ensuremath\langle}}
  \renewcommand{\>}{{\ensuremath\rangle}}

  \newcommand{\Prob}{{\rm Prob}}
  \newcommand{\Dir}{{\rm Dir}}
  \newcommand{\Aut}{{\rm Aut}}
  \newcommand{\cor}{{\rm cor}}
  \newcommand{\corr}{{\rm corr}}
  \newcommand{\cov}{{\rm cov}}
  \newcommand{\sd}{{\rm sd}}
  \newcommand{\tr}{{\rm tr}}
  \newcommand{\Tr}{{\rm Tr}}
  \newcommand{\rank}{{\rm rank}}
  \newcommand{\diag}{{\rm diag}}
  \newcommand{\id}{{\rm id}}
  \newcommand{\Id}{{\rm\bf I}}
  \newcommand{\Gl}{{\rm Gl}}
  \renewcommand{\th}{\ensuremath{{}^\text{th}} }
  \newcommand{\lag}{\mathcal{L}}
  \newcommand{\inn}{\rfloor}
  \newcommand{\lie}{\pounds}
  \newcommand{\longto}{\longrightarrow}
  \newcommand{\speer}{\parbox{0.4ex}{\raisebox{0.8ex}{$\nearrow$}}}
  \renewcommand{\dag}{ {}^\dagger }
  \newcommand{\Ji}{J^\sharp}
  \newcommand{\h}{{}^\star}
  \newcommand{\w}{\wedge}
  \newcommand{\too}{\longrightarrow}
  \newcommand{\oot}{\longleftarrow}
  \newcommand{\To}{\Rightarrow}
  \newcommand{\oT}{\Leftarrow}
  \newcommand{\oTo}{\Leftrightarrow}
  \newcommand{\Too}{\;\Longrightarrow\;}
  \newcommand{\oto}{\leftrightarrow}
  \newcommand{\ot}{\leftarrow}
  \newcommand{\ootoo}{\longleftrightarrow}
  \newcommand{\ow}{\stackrel{\circ}\wedge}
  \newcommand{\feed}{\nonumber \\}
  \newcommand{\comma}{~,\quad}
  \newcommand{\period}{~.\quad}
  \newcommand{\del}{\partial}
%  \newcommand{\quabla}{\Delta}
  \newcommand{\point}{$\bullet~~$}
  \newcommand{\doubletilde}{ ~ \raisebox{0.3ex}{$\widetilde {}$} \raisebox{0.6ex}{$\widetilde {}$} \!\! }
  \newcommand{\topcirc}{\parbox{0ex}{~\raisebox{2.5ex}{${}^\circ$}}}
  \newcommand{\topdot} {\parbox{0ex}{~\raisebox{2.5ex}{$\cdot$}}}
  \newcommand{\topddot} {\parbox{0ex}{~\raisebox{1.3ex}{$\ddot{~}$}}}
  \newcommand{\sym}{\topcirc}
  \newcommand{\tsum}{\textstyle\sum}

  \newcommand{\half}{\ensuremath{\frac{1}{2}}}
  \newcommand{\third}{\ensuremath{\frac{1}{3}}}
  \newcommand{\fourth}{\ensuremath{\frac{1}{4}}}

  \newcommand{\ubar}{\underline}
  %\renewcommand{\vec}{\underline}
  \renewcommand{\vec}{\boldsymbol}
  %\renewcommand{\_}{\underset}
  %\renewcommand{\^}{\overset}
  %\renewcommand{\*}{{\rm\raisebox{-.6ex}{\text{*}}{}}}
  \renewcommand{\*}{\text{\footnotesize\raisebox{-.4ex}{*}{}}}

  \newcommand{\gto}{{\raisebox{.5ex}{${}_\rightarrow$}}}
  \newcommand{\gfrom}{{\raisebox{.5ex}{${}_\leftarrow$}}}
  \newcommand{\gnto}{{\raisebox{.5ex}{${}_\nrightarrow$}}}
  \newcommand{\gnfrom}{{\raisebox{.5ex}{${}_\nleftarrow$}}}

  %\newcommand{\RND}{{\SS}}
  %\newcommand{\IF}{\text{if }}
  %\newcommand{\AND}{\textsc{and }}
  %\newcommand{\OR}{\textsc{or }}
  %\newcommand{\XOR}{\textsc{xor }}
  %\newcommand{\NOT}{\textsc{not }}

  %\newcommand{\argmax}[1]{{\rm arg}\!\max_{#1}}
  %\newcommand{\argmin}[1]{{\rm arg}\!\min_{#1}}
  \DeclareMathOperator*{\argmax}{argmax}
  \DeclareMathOperator*{\argmin}{argmin}
  \DeclareMathOperator{\sign}{sign}
  \DeclareMathOperator{\acos}{acos}
  %\newcommand{\argmax}[1]{\underset{~#1}{\text{argmax}}\;}
  %\newcommand{\argmin}[1]{\underset{~#1}{\text{argmin}}\;}
  \newcommand{\ee}[1]{\ensuremath{\cdot10^{#1}}}
  \newcommand{\sub}[1]{\ensuremath{_{\text{#1}}}}
  \newcommand{\up}[1]{\ensuremath{^{\text{#1}}}}
  \newcommand{\kld}[3][{}]{D_{#1}\big(#2\,\big|\!\big|\,#3\big)}
  %\newcommand{\kld}[2]{D\big(#1:#2\big)}
  \newcommand{\sprod}[2]{\big<#1\,,\,#2\big>}
  \newcommand{\End}{\text{End}}
  \newcommand{\txt}[1]{\quad\text{#1}\quad}
  \newcommand{\Over}[2]{\genfrac{}{}{0pt}{0}{#1}{#2}}
  %\newcommand{\mat}[1]{{\bf #1}}
  \newcommand{\arr}[2]{\hspace*{-.5ex}\begin{array}{#1}#2\end{array}\hspace*{-.5ex}}
  \newcommand{\mat}[3][.9]{
    \renewcommand{\arraystretch}{#1}{\scriptscriptstyle{\left(
      \hspace*{-1ex}\begin{array}{#2}#3\end{array}\hspace*{-1ex}
    \right)}}\renewcommand{\arraystretch}{1.2}
  }
  %\newcommand{\case}[1]{\left\{\arr{ll}{#1}\right.}
  \newcommand{\seq}[1]{\textsf{\<#1\>}}
  \newcommand{\seqq}[1]{\textsf{#1}}
  \newcommand{\floor}[1]{\lfloor#1\rfloor}
  \newcommand{\Exp}[2]{{\rm E}_{#1}\{#2\}} \newcommand{\E}{{\rm E}}

  \newcommand{\href}[1]{}
  \newcommand{\url}[2]{\href{#1}{{\color{blue}#2}}}
  \newcommand{\anchor}[2]{\begin{picture}(0,0)\put(#1){#2}\end{picture}}
  \newcommand{\pagebox}{\begin{picture}(0,0)\put(-3,-23){
    \textcolor[rgb]{.5,1,.5}{\framebox[\textwidth]{\rule[-\textheight]{0pt}{0pt}}}}
    \end{picture}}

  \newcommand{\hide}[1]{
    \begin{list}{}{\leftmargin0ex \rightmargin0ex \topsep0ex \parsep0ex}
       \helvetica{5}{1}{m}{n}
       \renewcommand{\section}{\par SECTION: }
       \renewcommand{\subsection}{\par SUBSECTION: }
       \item[$~~\blacktriangleright$]
       #1%$\blacktriangleleft~~$
       \message{^^JHIDE--Warning!^^J}
    \end{list}
  }
  %\newcommand{\hide}[1]{{\tt[hide:~}{\footnotesize\sf #1}{\tt]}\message{^^JHIDE--Warning!^^J}}
  \newcommand{\Hide}{\renewcommand{\hide}[1]{\message{^^JHIDE--Warning (hidden)!^^J}}}
  \newcommand{\HIDE}{\renewcommand{\hide}[1]{}}
  \newcommand{\fullhide}[1]{}
  \newcommand{\todo}[1]{{\tt[TODO: #1]}\message{^^JTODO--Warning: #1^^J}}
  \newcommand{\Todo}{\renewcommand{\todo}[1]{\message{^^JTODO--Warning (hidden)!^^J}}}
  %\renewcommand{\title}[1]{\renewcommand{\thetitle}{#1}}
  \newcommand{\header}{\begin{document}\mytitle\cleardefs}
  \newcommand{\contents}{{\tableofcontents}\renewcommand{\contents}{}}
  \newcommand{\footer}{\small\bibliography{bibs}\end{document}}
  \newcommand{\widepaper}{\usepackage{geometry}\geometry{a4paper,hdivide={25mm,*,25mm},vdivide={25mm,*,25mm}}}
  \newcommand{\moviex}[2]{\movie[externalviewer]{#1}{#2}} %\pdflatex\usepackage{multimedia}
  \newcommand{\rbox}[1]{\fboxrule2mm\fcolorbox[rgb]{1,.85,.85}{1,.85,.85}{#1}}
  \newcommand{\redbox}[2]{\fboxrule2mm\fcolorbox[rgb]{1,.7,.7}{1,.7,.7}{\begin{minipage}{#1}\center#2\end{minipage}}}
  \newcommand{\twocol}[5][c]{\begin{minipage}[#1]{#2\columnwidth}#4\end{minipage}\begin{minipage}[#1]{#3\columnwidth}#5\end{minipage}}
  \newcommand{\threecoltext}[7][c]{\begin{minipage}[#1]{#2\textwidth}#5\end{minipage}\begin{minipage}[#1]{#3\textwidth}#6\end{minipage}\begin{minipage}[#1]{#4\textwidth}#7\end{minipage}}
  \newcommand{\threecol}[7][c]{\begin{minipage}[#1]{#2\columnwidth}#5\end{minipage}\begin{minipage}[#1]{#3\columnwidth}#6\end{minipage}\begin{minipage}[#1]{#4\columnwidth}#7\end{minipage}}
  \newcommand{\helvetica}[4]{\setlength{\unitlength}{1pt}\fontsize{#1}{#1}\linespread{#2}\usefont{OT1}{phv}{#3}{#4}}
  \newcommand{\helve}[1]{\helvetica{#1}{1.5}{m}{n}}
  \newcommand{\german}{\usepackage[german]{babel}\usepackage[latin1]{inputenc}}

\newcommand{\norm}[2]{|\!|#1|\!|_{{}_{#2}}}
\newcommand{\expr}[1]{[\hspace{-.2ex}[#1]\hspace{-.2ex}]}

\newcommand{\Jwi}{J^\sharp_W}
\newcommand{\Twi}{T^\sharp_W}
\newcommand{\Jci}{J^\natural_C}
\newcommand{\hJi}{{\bar J}^\sharp}
\renewcommand{\|}{\,|\,}
\renewcommand{\=}{\!=\!}
\newcommand{\myminus}{{\hspace*{-.3pt}\text{\rm -}\hspace*{-.5pt}}}
\newcommand{\myplus}{{\hspace*{-.3pt}\text{\rm +}\hspace*{-.5pt}}}
\newcommand{\1}{{\myminus1}}
\newcommand{\2}{{\myminus2}}
\newcommand{\3}{{\myminus3}}
\newcommand{\mT}{{\myminus{}\top}}
\newcommand{\po}{{\myplus1}}
\newcommand{\pt}{{\myplus2}}
%\renewcommand{\-}{\myminus}
%\newcommand{\+}{\myplus}
\renewcommand{\T}{{\!\top\!}}
\renewcommand{\mT}{{\myminus{}\top}}

\newenvironment{centy}{
\vspace{15mm}
\large
\hspace*{5mm}
\begin{minipage}{8cm}\it\color{blue}
}{
\end{minipage}
}

\newcommand{\old}{{\text{old}}}
\newcommand{\new}{{\text{new}}}


%%%%%%%%%%%%%%%%%%%%%%%%%%%%%%%%%%%%%%%%%%%%%%%%%%%%%%%%%%%%%%%%%%%%%%%%%%%%%%%%
\pdflatex
\usepackage{framed}
  \definecolor{shadecolor}{gray}{0.9}
  \setlength{\FrameSep}{3pt}
\usepackage{fancyvrb}

%\DefineShortVerb{\@}

\newcommand{\go}{{\texttt{go}}}

\title{A Simple Concurrent Action Formalism}
\author{Marc Toussaint}

\begin{document}
\maketitle

\section{Introduction}

There are many existing formalisms and approaches to 1) describe
concurrent (multi-agent) action domains and problems (including PDDL
level 4); 2) to reduce such domains to normal Markovian (or
semi-Markov) domains; and 3) solve the resulting temporal \emph{planning}
problems. \cite{*}

I see the situation as follows: Basic research was already done more
than a decade ago. Now, the issues are rather:
\begin{itemize}
\item What is an efficient formalism that allows us
  to integrate it with our existing relational RL and robot control
  frameworks?
\item What is an approach to also compute \emph{posteriors} (e.g., for
  anticipation) rather than ``only'' solve (usually deterministic)
  planning problems
\item What is a formulation that allows us to leverage Monte-Carlo
  Tree Search methods (UCT)?
\end{itemize}

In the following I propose a simple approach. Questions w.r.t.\ this
approach are:
\begin{itemize}
\item Is it equally general (can cope with the same range of domains)
as other formalisms?
\item What notions of optimality are there, also in the stochastic
duration case?
\item More thoroughly formalism also the stochastic case.
\item Optimality and theory on MCTS within this framework.
\end{itemize}



\section{From Actions to Initiation \& Termination Decisions}

The classical formulations still think of the ``actions''
as the decision variables. In the concurrent case, they lift this to
joint actions, i.e., making decisions about action tuples. But this
raises ugly isues when actions have variable or even stochastic
durations. All that led to the ideas of temporal planning, etc.

Much easier is the following process definition. We
first describe \emph{deterministic domains}:

\subsection{Initiation \& termination operators}

Assume we have $n$ action symbols (activity would be a better word,
but too long). For each action symbol $a$ we have \emph{two} operators
(instead of one, as with standard action operators):
\begin{itemize}
\item The \emph{initiation operator}
$$a_I(X):~ \text{precond}(X) \to \go(a,X)=d_a, \text{other\_effects}$$
\item The \emph{termination operator}
$$a_T(X):~ \quad  \to \neg\go(a,X),  \text{other\_effects}$$
\end{itemize}

Both of these operators change the current relational
state \emph{instantly}.

The \go-predicate is the core: A $\go(a,X)$-predicate in the current
state indicates that the action $a$ with arguments $X$ is currently
active. Further, the \go-predicate is \emph{real-valued}! Its current
value is the \textbf{time-to-go} until this activity will terminate
and change the relational state. The initiation operator initialized
the value of the \go-predicate to the (here deterministic) duration
$d_a$ of that action.

If multiple initiation operators initiate multiple actions instantly
at the same point in time, they're all concurrently active, perhaps
with different times-to-go.

\subsection{Initiation \& wait decisions}

The resulting \emph{decision} process is Markovian: The process is a
series of two types of decisions: \emph{initiation decisions}
and \emph{wait decisions}.

Each initiation decision might refer to a different agent (indicated
by the argment $X$ to an action $a$). Initiation decisions are
instantaneous, have no real-time duration. Therefore, although they
``increase the step-counter'' of the Markov Decision Process---or the
depth of the search tree---real-time does not progress.

The wait decision [[refer to previous literature on these within
sMDP]] is a unique thing, has no arguments $X$ or $a$, and does not
really refer to a particular agent---it rather expresses
that \emph{all agents decide not to initiate anything further in the
current relational state}. Therefore, the wait decision progresses
real-time \emph{until the relational state changes}. In real-world,
there may be different ``reasons'' the relational state changes:
\begin{items}
\item A currently active activity may return feedback, e.g. on its
convergence, failure, termination, change-in-success-probability,
anything. In all of these cases the activity just adds a predicate to
the relational state encoding the novel information.
\item A sensor signal might add and observation predicate to the
relational state. (Actually, if one thinks of the activity of sensors
also as an action, then this case is no different to the first.)
\end{items}

However, in model-based planning we assume to have a model of the
world. Therefore we assume to know (at least in the deterministic
case) at what time an activity will change the
relational state. The wait decision is therefore realized by a very
particular operator on the relational state, which is not expressed by
a rule with preconditions and effects, but by the following little
``procedure'':
\begin{enumerate}
\item Find the \go-predicate with the minimal time-to-go value $d_{min}$.
\item Decrement all \go-predicate-values by $d_{min}$.
\item For all $\go(a,X)$ predicates with $\go(a,X)=0$ call
the \emph{termination operator} $a_T(X)$
\end{enumerate}

\subsection{Redundancies in the decision tree}

Consider the decision tree for this process. There is undesirable
redundancy in this tree: When multiple initiation decisions are made,
their order should not patter. (At least in the classical formalisms they
don't---in my formalisms there could be interesting preconditions to
allow/disallow certain orders.) If the order does not matter, we can
identify the resulting nodes as equal.

\section{Example}

Here is an example domain: the box assembly. The initial part simply
declares a number of symbols which can later be used to formulate
literals (nothing but tuples of symbols or variables, in this
representation).

The initial state is a set (conjunction) of literals.

The terminal condition is a set (conjunction) of literals.

For every action symbol (activity) an initiation and termination
operator is defined. Each operator first declares variables (arguments
of the operator), the precondition as a set of literals, and the
effect as a set of literals.

Note that for this example we had to introduce symbols to that allow
us to indicate mutexes between activities, like busy (that hand is
busy and can't do something else), etc. We played with this a lot and
think that also the classical formulations are by no means better or
easier: They have to introduce a lot of extra formalism to indicate
various types of mutexes (eternal, etc). Here, all mutexing is
represented in the preconditions of initiation operators.

[The file syntax is actually a hierarchical hypergraph syntax, with
tuples being hypernodes/edges. We interally implement all first order
logic operations (esp.\ computing feasible variable substitutions) on
such a generalized graph representation.]

\begin{shaded}
\begin{Verbatim}[fontfamily=courier,fontsize=\tiny]
## Syntactic keywords
Terminate
#Rule

## activities
pickingup
positioning
screwing
releasing
busy
inhandNil

## basic predicates
table
on
wall
screw
ground
object
material
humR
humL
rob
hand
used
inPosition
inhand
fixed

## constants
Constant 67
Constant 71
Constant 72
Constant 76
Constant 77
Constant 78

## initial state
(screw 67)
(object 67)
(ground 71)
(material 71)
(object 71)
(wall 72)
(material 72)
(object 72)
(humR 76)
(hand 76)
(humL 77)
(hand 77)
(rob 78)
(hand 78)
(inhandNil 76)
(inhandNil 77)
(inhandNil 78)

### terminal state

terminal { (used 67) (inPosition 71) (inPosition 72) }

### RULES

Rule activate_pickingup {
     X, Y
     (pickingup X Y)! (hand X) (object Y) (inhandNil X) (busy X)! (busy Y)!
     effect { (go pickingup X Y)=2.1 (busy X) (busy Y) }
}

Term (Terminate pickingup) {
     X, Y
     effect { (go pickingup X Y)! (inhand X Y) (inhandNil X)! (busy X)! }
}

Rule activate_positioning {
     X, Y
     (positioning X Y)! (hand X) (object Y) (inhand X Y) (material Y) (inPosition Y)! (busy X)!
     effect { (go positioning X Y)=3.5 (busy X) }
}

Term (Terminate positioning) {
     X, Y
     effect { (go positioning X Y)! (inPosition Y)  (busy X)! }
}

Rule activate_releasing {
     X, Y
     (releasing X Y)! (hand X) (object Y) (inhand X Y) (material Y) (busy X)!
     effect { (go releasing X Y)=1.0 (busy X) }
}

Term (Terminate releasing){
     X, Y
     effect { (go releasing X Y)! (inhand X Y)! (inhandNil X) (busy X)! (busy Y)! }
}

Rule activate_screwing {
     X, Y, Z, U, V, W
     (screwing X Y Z U V W)! (hand X) (screw Y) (inhand X Y) (wall Z) (ground W) (inPosition Z) (inPosition W) (hand U) (inhand U Z) (hand V) (inhand V W) (fixed Z W)! (busy X)! (busy U)! (busy V)!  (used Y)!
     effect { (go screwing X Y Z U V W)=8.0 (busy X) (busy U) (busy V) (inPosition Z) (inPosition W) }
}

Term (Terminate screwing) {
     X, Y, Z, U, V, W
     effect { (go screwing X Y Z U V W)! (fixed Z W) (used Y) (inPosition Z) (inPosition W) (busy X)! (busy U)! (busy V)! (inhand X Y)! (inhandNil X) }
}

\end{Verbatim}
\end{shaded}

\bibliography{conAct.bib}

\end{document}

%% \subsection{Deterministic effect and duration}

%% TODO: compare to PDDL level 4!

%% We assume we are in relational setting. The current state is described
%% by 1st-order logic sentence.

%% The transitino dynamics of the domain are desribed by a set of
%% (ConAct-STRIPS) rules, where each rule is a 5 tupe
%% \begin{itemize}
%% \item The action predicate $a(\{X_i\})$ with logical variables
%%   $\{X_i\}$ as arguments
%% \item The precondition $\phi(\{X_i\},\{Z_i\})$ (a logic sentence) that
%%   may depend on deictic variables $\{Z_i\}$
%% \item The effect $\O(\{X_i\},\{Z_i\})$, that defines a state
%%   transition under the frame assumption
%% \item A mutex list of action predicates $\{b(\{X_i\})\}$ which
%%   indicates that execution of $a(\{X_i\})$ is not allowed if any of
%%   $b(\{X_i\})$ are currently active
%% \item A deterministic duration $d\in\RRR$
%% \end{itemize}


%% \section{Overview of reductions}

%% \begin{description}
%% \item[Activation decisions] We think of the current activity $A$ (the
%%   set of all currently active actions) as part of the state. As the
%%   ``new action space'' we introduce the decision to activate a certain
%%   action, i.e., to add a certain action to $A$. Executing concurrent
%%   actions then is a sequence of decision. (details below)

%% \item[Combinatorial actionset decisions] We chop the everything in
%%   separate episodes. In each episode we need to make a decision about
%%   the combinatorial concurrent action set $A$.
%% \end{description}


%% \section{Activation decisions}

%% The original state $s$ is a sentence in a relational language. For
%% every action predicate $a(\{X_i\})$ we introduce a new function
%% $timeToGo(a,\{X_i\})$ which evaluates to $0$ if the action is
%% currently inactive and non-zero otherwise.

%% Also, for every action predicate $a(\{X_i\})$ we introduce activation
%% decision predicates $activate(a,\{X_i\})$. Each (grounded) decision
%% predicate uniquely defines a state transition in the extended state
%% space:
%% \begin{verbatim}
%% activate(a,X):
%% PRE: timeToGo(a,X)=0, \forall_b in MUTEX(a,X): timeToGo(b,X)=0
%% POST: timeToGo(a,X)=a.duration
%% \end{verbatim}

%% \textbf{These activation decision predicates to not progress time}

%% Additionally, we introduce one more decision, which is the
%% \texttt{waitForNextHappening}, which uniquely defines a state
%% transition in the extended state space by searching for the smallest
%% non-zero time-to-go and decrementing \emph{all} time-to-gos by that
%% amount.

%% \textbf{This single decision is the only one to progress time}

%% Now everything should become a problem of finding a sequence of
%% decisions in the extended state space, right?












\documentclass[10pt,fleqn,twoside]{article}
\usepackage{palatino}
\usepackage{amsmath}
\usepackage{amssymb}
\usepackage{amsfonts}
\usepackage{amsthm}
\usepackage{eucal}
\usepackage{graphicx}
\usepackage{color}

\usepackage[round]{natbib}
\bibliographystyle{abbrvnat}
%\usepackage[german]{babel}
%\usepackage[utf8]{inputenc}

\graphicspath{{pics/}{figs/}{~/write/tex/pics/}{~/write/tex/figs/}{~/teaching/pics-all/}}
\usepackage{geometry}
\geometry{a4paper,hdivide={35mm,*,35mm},vdivide={35mm,*,35mm}}
\renewcommand{\baselinestretch}{1.1}

\newenvironment{items}{
\par\small
\begin{list}{--}{\leftmargin4ex \rightmargin0ex \labelsep1ex \labelwidth2ex
\topsep0pt \parsep0ex \itemsep3pt}
}{
\end{list}
}


  \renewcommand{\a}{\ensuremath\alpha}
  \renewcommand{\b}{\beta}
  \renewcommand{\c}{\gamma}
  \renewcommand{\d}{\delta}
    \newcommand{\D}{\Delta}
    \newcommand{\e}{\epsilon}
    \newcommand{\g}{\gamma}
    \newcommand{\G}{\Gamma}
  \renewcommand{\l}{\lambda}
  \renewcommand{\L}{\Lambda}
    \newcommand{\m}{\mu}
    \newcommand{\n}{\nu}
    \newcommand{\N}{\nabla}
  \renewcommand{\k}{\kappa}
  \renewcommand{\o}{\omega}
  \renewcommand{\O}{\Omega}
    \newcommand{\p}{\phi}
    \newcommand{\ph}{\varphi}
  \renewcommand{\P}{\Phi}
  \renewcommand{\r}{\varrho}
    \newcommand{\s}{\sigma}
  \renewcommand{\S}{\Sigma}
  \renewcommand{\t}{\theta}
    \newcommand{\T}{\Theta}
  %\renewcommand{\v}{\vartheta}
    \newcommand{\x}{\xi}
    \newcommand{\X}{\Xi}
    \newcommand{\Y}{\Upsilon}

  \renewcommand{\AA}{{\cal A}}
    \newcommand{\BB}{{\cal B}}
    \newcommand{\CC}{{\cal C}}
    \newcommand{\EE}{{\cal E}}
    \newcommand{\FF}{{\cal F}}
    \newcommand{\GG}{{\cal G}}
    \newcommand{\HH}{{\cal H}}
    \newcommand{\II}{{\cal I}}
    \newcommand{\KK}{{\cal K}}
    \newcommand{\LL}{{\cal L}}
    \newcommand{\MM}{{\cal M}}
    \newcommand{\NN}{{\cal N}}
    \newcommand{\oNN}{\overline\NN}
    \newcommand{\OO}{{\cal O}}
    \newcommand{\PP}{{\cal P}}
    \newcommand{\QQ}{{\cal Q}}
    \newcommand{\RR}{{\cal R}}
  \renewcommand{\SS}{{\cal S}}
    \newcommand{\TT}{{\cal T}}
    \newcommand{\uu}{{\cal u}}
    \newcommand{\UU}{{\cal U}}
    \newcommand{\VV}{{\cal V}}
    \newcommand{\XX}{{\cal X}}
    \newcommand{\YY}{{\cal Y}}
    \newcommand{\SOSO}{{\cal SO}}
    \newcommand{\GLGL}{{\cal GL}}

    \newcommand{\Ee}{{\rm E}}

  \newcommand{\NNN}{{\mathbb{N}}}
  \newcommand{\ZZZ}{{\mathbb{Z}}}
  %\newcommand{\RRR}{{\mathrm{I\!R}}}
  \newcommand{\RRR}{{\mathbb{R}}}
  \newcommand{\CCC}{{\mathbb{C}}}
  \newcommand{\one}{{{\bf 1}}}
  \newcommand{\eee}{\text{e}}

  \newcommand{\NNNN}{{\overline{\cal N}}}

  \renewcommand{\[}{\Big[}
  \renewcommand{\]}{\Big]}
  \renewcommand{\(}{\Big(}
  \renewcommand{\)}{\Big)}
  \renewcommand{\|}{\,|\,}
  \renewcommand{\=}{\!=\!}
    \newcommand{\<}{{\ensuremath\langle}}
  \renewcommand{\>}{{\ensuremath\rangle}}

  \newcommand{\Prob}{{\rm Prob}}
  \newcommand{\Dir}{{\rm Dir}}
  \newcommand{\Aut}{{\rm Aut}}
  \newcommand{\cor}{{\rm cor}}
  \newcommand{\corr}{{\rm corr}}
  \newcommand{\cov}{{\rm cov}}
  \newcommand{\sd}{{\rm sd}}
  \newcommand{\tr}{{\rm tr}}
  \newcommand{\Tr}{{\rm Tr}}
  \newcommand{\rank}{{\rm rank}}
  \newcommand{\diag}{{\rm diag}}
  \newcommand{\id}{{\rm id}}
  \newcommand{\Id}{{\rm\bf I}}
  \newcommand{\Gl}{{\rm Gl}}
  \renewcommand{\th}{\ensuremath{{}^\text{th}} }
  \newcommand{\lag}{\mathcal{L}}
  \newcommand{\inn}{\rfloor}
  \newcommand{\lie}{\pounds}
  \newcommand{\longto}{\longrightarrow}
  \newcommand{\speer}{\parbox{0.4ex}{\raisebox{0.8ex}{$\nearrow$}}}
  \renewcommand{\dag}{ {}^\dagger }
  \newcommand{\Ji}{J^\sharp}
  \newcommand{\h}{{}^\star}
  \newcommand{\w}{\wedge}
  \newcommand{\too}{\longrightarrow}
  \newcommand{\oot}{\longleftarrow}
  \newcommand{\To}{\Rightarrow}
  \newcommand{\oT}{\Leftarrow}
  \newcommand{\oTo}{\Leftrightarrow}
  \newcommand{\Too}{\;\Longrightarrow\;}
  \newcommand{\oto}{\leftrightarrow}
  \newcommand{\ot}{\leftarrow}
  \newcommand{\ootoo}{\longleftrightarrow}
  \newcommand{\ow}{\stackrel{\circ}\wedge}
  \newcommand{\feed}{\nonumber \\}
  \newcommand{\comma}{~,\quad}
  \newcommand{\period}{~.\quad}
  \newcommand{\del}{\partial}
%  \newcommand{\quabla}{\Delta}
  \newcommand{\point}{$\bullet~~$}
  \newcommand{\doubletilde}{ ~ \raisebox{0.3ex}{$\widetilde {}$} \raisebox{0.6ex}{$\widetilde {}$} \!\! }
  \newcommand{\topcirc}{\parbox{0ex}{~\raisebox{2.5ex}{${}^\circ$}}}
  \newcommand{\topdot} {\parbox{0ex}{~\raisebox{2.5ex}{$\cdot$}}}
  \newcommand{\topddot} {\parbox{0ex}{~\raisebox{1.3ex}{$\ddot{~}$}}}
  \newcommand{\sym}{\topcirc}
  \newcommand{\tsum}{\textstyle\sum}

  \newcommand{\half}{\ensuremath{\frac{1}{2}}}
  \newcommand{\third}{\ensuremath{\frac{1}{3}}}
  \newcommand{\fourth}{\ensuremath{\frac{1}{4}}}

  \newcommand{\ubar}{\underline}
  %\renewcommand{\vec}{\underline}
  \renewcommand{\vec}{\boldsymbol}
  %\renewcommand{\_}{\underset}
  %\renewcommand{\^}{\overset}
  %\renewcommand{\*}{{\rm\raisebox{-.6ex}{\text{*}}{}}}
  \renewcommand{\*}{\text{\footnotesize\raisebox{-.4ex}{*}{}}}

  \newcommand{\gto}{{\raisebox{.5ex}{${}_\rightarrow$}}}
  \newcommand{\gfrom}{{\raisebox{.5ex}{${}_\leftarrow$}}}
  \newcommand{\gnto}{{\raisebox{.5ex}{${}_\nrightarrow$}}}
  \newcommand{\gnfrom}{{\raisebox{.5ex}{${}_\nleftarrow$}}}

  %\newcommand{\RND}{{\SS}}
  %\newcommand{\IF}{\text{if }}
  %\newcommand{\AND}{\textsc{and }}
  %\newcommand{\OR}{\textsc{or }}
  %\newcommand{\XOR}{\textsc{xor }}
  %\newcommand{\NOT}{\textsc{not }}

  %\newcommand{\argmax}[1]{{\rm arg}\!\max_{#1}}
  %\newcommand{\argmin}[1]{{\rm arg}\!\min_{#1}}
  \DeclareMathOperator*{\argmax}{argmax}
  \DeclareMathOperator*{\argmin}{argmin}
  \DeclareMathOperator{\sign}{sign}
  \DeclareMathOperator{\acos}{acos}
  %\newcommand{\argmax}[1]{\underset{~#1}{\text{argmax}}\;}
  %\newcommand{\argmin}[1]{\underset{~#1}{\text{argmin}}\;}
  \newcommand{\ee}[1]{\ensuremath{\cdot10^{#1}}}
  \newcommand{\sub}[1]{\ensuremath{_{\text{#1}}}}
  \newcommand{\up}[1]{\ensuremath{^{\text{#1}}}}
  \newcommand{\kld}[3][{}]{D_{#1}\big(#2\,\big|\!\big|\,#3\big)}
  %\newcommand{\kld}[2]{D\big(#1:#2\big)}
  \newcommand{\sprod}[2]{\big<#1\,,\,#2\big>}
  \newcommand{\End}{\text{End}}
  \newcommand{\txt}[1]{\quad\text{#1}\quad}
  \newcommand{\Over}[2]{\genfrac{}{}{0pt}{0}{#1}{#2}}
  %\newcommand{\mat}[1]{{\bf #1}}
  \newcommand{\arr}[2]{\hspace*{-.5ex}\begin{array}{#1}#2\end{array}\hspace*{-.5ex}}
  \newcommand{\mat}[3][.9]{
    \renewcommand{\arraystretch}{#1}{\scriptscriptstyle{\left(
      \hspace*{-1ex}\begin{array}{#2}#3\end{array}\hspace*{-1ex}
    \right)}}\renewcommand{\arraystretch}{1.2}
  }
  %\newcommand{\case}[1]{\left\{\arr{ll}{#1}\right.}
  \newcommand{\seq}[1]{\textsf{\<#1\>}}
  \newcommand{\seqq}[1]{\textsf{#1}}
  \newcommand{\floor}[1]{\lfloor#1\rfloor}
  \newcommand{\Exp}[2]{{\rm E}_{#1}\{#2\}} \newcommand{\E}{{\rm E}}

  \newcommand{\href}[1]{}
  \newcommand{\url}[2]{\href{#1}{{\color{blue}#2}}}
  \newcommand{\anchor}[2]{\begin{picture}(0,0)\put(#1){#2}\end{picture}}
  \newcommand{\pagebox}{\begin{picture}(0,0)\put(-3,-23){
    \textcolor[rgb]{.5,1,.5}{\framebox[\textwidth]{\rule[-\textheight]{0pt}{0pt}}}}
    \end{picture}}

  \newcommand{\hide}[1]{
    \begin{list}{}{\leftmargin0ex \rightmargin0ex \topsep0ex \parsep0ex}
       \helvetica{5}{1}{m}{n}
       \renewcommand{\section}{\par SECTION: }
       \renewcommand{\subsection}{\par SUBSECTION: }
       \item[$~~\blacktriangleright$]
       #1%$\blacktriangleleft~~$
       \message{^^JHIDE--Warning!^^J}
    \end{list}
  }
  %\newcommand{\hide}[1]{{\tt[hide:~}{\footnotesize\sf #1}{\tt]}\message{^^JHIDE--Warning!^^J}}
  \newcommand{\Hide}{\renewcommand{\hide}[1]{\message{^^JHIDE--Warning (hidden)!^^J}}}
  \newcommand{\HIDE}{\renewcommand{\hide}[1]{}}
  \newcommand{\fullhide}[1]{}
  \newcommand{\todo}[1]{{\tt[TODO: #1]}\message{^^JTODO--Warning: #1^^J}}
  \newcommand{\Todo}{\renewcommand{\todo}[1]{\message{^^JTODO--Warning (hidden)!^^J}}}
  %\renewcommand{\title}[1]{\renewcommand{\thetitle}{#1}}
  \newcommand{\header}{\begin{document}\mytitle\cleardefs}
  \newcommand{\contents}{{\tableofcontents}\renewcommand{\contents}{}}
  \newcommand{\footer}{\small\bibliography{bibs}\end{document}}
  \newcommand{\widepaper}{\usepackage{geometry}\geometry{a4paper,hdivide={25mm,*,25mm},vdivide={25mm,*,25mm}}}
  \newcommand{\moviex}[2]{\movie[externalviewer]{#1}{#2}} %\pdflatex\usepackage{multimedia}
  \newcommand{\rbox}[1]{\fboxrule2mm\fcolorbox[rgb]{1,.85,.85}{1,.85,.85}{#1}}
  \newcommand{\redbox}[2]{\fboxrule2mm\fcolorbox[rgb]{1,.7,.7}{1,.7,.7}{\begin{minipage}{#1}\center#2\end{minipage}}}
  \newcommand{\twocol}[5][c]{\begin{minipage}[#1]{#2\columnwidth}#4\end{minipage}\begin{minipage}[#1]{#3\columnwidth}#5\end{minipage}}
  \newcommand{\threecoltext}[7][c]{\begin{minipage}[#1]{#2\textwidth}#5\end{minipage}\begin{minipage}[#1]{#3\textwidth}#6\end{minipage}\begin{minipage}[#1]{#4\textwidth}#7\end{minipage}}
  \newcommand{\threecol}[7][c]{\begin{minipage}[#1]{#2\columnwidth}#5\end{minipage}\begin{minipage}[#1]{#3\columnwidth}#6\end{minipage}\begin{minipage}[#1]{#4\columnwidth}#7\end{minipage}}
  \newcommand{\helvetica}[4]{\setlength{\unitlength}{1pt}\fontsize{#1}{#1}\linespread{#2}\usefont{OT1}{phv}{#3}{#4}}
  \newcommand{\helve}[1]{\helvetica{#1}{1.5}{m}{n}}
  \newcommand{\german}{\usepackage[german]{babel}\usepackage[latin1]{inputenc}}

\newcommand{\norm}[2]{|\!|#1|\!|_{{}_{#2}}}
\newcommand{\expr}[1]{[\hspace{-.2ex}[#1]\hspace{-.2ex}]}

\newcommand{\Jwi}{J^\sharp_W}
\newcommand{\Twi}{T^\sharp_W}
\newcommand{\Jci}{J^\natural_C}
\newcommand{\hJi}{{\bar J}^\sharp}
\renewcommand{\|}{\,|\,}
\renewcommand{\=}{\!=\!}
\newcommand{\myminus}{{\hspace*{-.3pt}\text{\rm -}\hspace*{-.5pt}}}
\newcommand{\myplus}{{\hspace*{-.3pt}\text{\rm +}\hspace*{-.5pt}}}
\newcommand{\1}{{\myminus1}}
\newcommand{\2}{{\myminus2}}
\newcommand{\3}{{\myminus3}}
\newcommand{\mT}{{\myminus{}\top}}
\newcommand{\po}{{\myplus1}}
\newcommand{\pt}{{\myplus2}}
%\renewcommand{\-}{\myminus}
%\newcommand{\+}{\myplus}
\renewcommand{\T}{{\!\top\!}}
\renewcommand{\mT}{{\myminus{}\top}}

\newenvironment{centy}{
\vspace{15mm}
\large
\hspace*{5mm}
\begin{minipage}{8cm}\it\color{blue}
}{
\end{minipage}
}

\newcommand{\old}{{\text{old}}}
\newcommand{\new}{{\text{new}}}


%%%%%%%%%%%%%%%%%%%%%%%%%%%%%%%%%%%%%%%%%%%%%%%%%%%%%%%%%%%%%%%%%%%%%%%%%%%%%%%%
\pdflatex
\usepackage{framed}
  \definecolor{shadecolor}{gray}{0.9}
  \setlength{\FrameSep}{3pt}
\usepackage{fancyvrb}
\DefineShortVerb{\@}

\title{Relational Machine\\Interfacing between robot activity control
  and relational learning and planning methods}
\author{M Toussaint}

\begin{document}
\maketitle

\section{Purpose}

What is described here plays a similar role to finite-state machines
for action sequencing in robotics (like SMASH, RBO's hierarchical
automaton), or also hierarchical RL frameworks (options, etc). The
motivation for the specific \emph{Relational Machine} approach is to
combine two aims:
\begin{itemize}
\item Represent sequencing and cuncurrency of activities (i.e., options,
  durative actions) in a way that matches directly with formalisms
  for relational reinforcement learning, MCTS planning and other
  relational AI algorithms.
\item At the same time provide a simple framework that flexibly allows
  us to design (also by hand and direct coding \& scripting) general
  concurrent activity policies, including the specification of the
  activities (control modes) themselves.
\end{itemize}

\section{The Relational Machine itself}

\textbf{Centralism:} We assume a centralized representation of the
relational state (or belief state). That is, even when describing
multi-agent systems, we do not consider a decentralized state as in
DEC-POMDPs, but assume that all sufficient knowledge for future
decisions (on the AI level) are accumulated in a central state
representation.

More specifically, we represent three things in a central ``knowledge
base'' (see notes on FOL):
\begin{itemize}
\item The fluent\footnote{changable, time-varying} \textbf{relational
  state} is a set of grounded literals. These literals represent
  ongoing activities, current termination criteria, sensor
  information, belief predicates, etc in a relational manner. These
  literals are mostly symbolic, but can be annotated with arbitrary
  dictionaries.
\item  The \textbf{set of symbols} describes which objects are
  currently in the domain. Typically this is static, but might be also
  dynamic or subject to uncertainty (see our
  uncertainty-over-existance work). Also this set of symbols describes
  which predicates are available, which relate to activities (like
  control modes, sensor activities/services, etc) as well as
  descriptive state symbols, termination condition symbols,
  uncertainty symbols etc.
\item The \textbf{set of rules} may comprise general first-order
  clauses or (probabilistic) relational rules. These rules
  generally describe state transitions---however, as 'activity
  indicators' are part of the state they can play also play the role
  of a policy: For instance, to represent ``$P(a|s)$'' a rule may,
  conditional to the state, add an activity predicate $a$ to the
  state. But rules can also model $P(s'|s,a)$ (as action operators)
  and thereby define the set of feasible decisions in $s$; or model
  $P(s'|s)$, i.e., state transitions that are usually triggered by the
  environment instead of the agent. Rules may also model the
  (expected) real-time duration of such transitions (see concurrent
  actions notes).
\end{itemize}


The Relational Machine needs to play two roles
\begin{itemize}
\item during real-time operation in interaction with real perceptional
  and control activities, where it controls how the activity state
  progresses, i.e., where it implements the logic of what activities
  are triggered/terminated when, etc.
\item as a forward simulation model on the relational level, to enable
  planning (esp.\ MCTS planning). In this case there is no interaction
  with real sensors and motors. Instead the knowledge base needs to
  include rules that simulate how these activities would interact with
  the relational state.
\end{itemize}

\subsection{Real-time operation}

In real-time, the Relational Machine interacts only with
\textbf{activities}. All activities always have access to the full
relational state (esp.\ they receive changes) and may trigger changes
to the relational state. Planning (or decision making in general) is
just special activity detailed in the next section.

A \textbf{change of the relational state} can be caused by any of the
following events:
\begin{itemize}
\item An activity explicitly modifies the relational state. For instance, an
  sensor activity indicates a new termination condition. Or a motor
  activity indicates convergence or time-out.
\item The set of rules auto-progresses the state. For instance, the
  rules represent a policy of what to do given a novel termination
  condition: a respective activity is cancelled (deleted from the
  relational state) and a new activity added
\end{itemize}
That's basically it.

The relational state determines the ``currently active activities'' as
follows: Every ``active activity'' must be identified with exactly one
fact (=grounded literal in the relational state). For instance
@(positionControl hand obj1){ rel=[0 0 0.2] PD=[.5, .9, .1, 10.] }@
will acivate a position controller that aims to align the position of
the @hand@ frame with the relative position $(0,0,0.2)$ in the @obj1@
frame following a desired PD approach behavior with time scale $0.5$,
daming ratio $0.9$ and max vel/acc $(0.1, 10)$.

Whenever such a fact is part of the state, such a position controller
is active. Whenever this fact is deleted from the state, the position
controller becomes inactive (or is removed from the task list of the
underlying operational space controller).

Note that the relational/symbolic aspect of such activities is captured
by the literal itself (the FOL tuple), whereas additional parameters
of the activity are \textbf{annotated with a dictionary} (technically,
a subgraph).

\subsection{Example activities}

One might distinguish three types of activities:
\begin{itemize}
\item \textbf{Control activities} impose a control task, typically
  between two objects (hand-object). ``Adding a control task''
  typically means to add a cost term in an operational space
  controller.

  Control activities may report back on their progress:
  e.g.\ convergence in the case of a PD-type feedback controller,
  time-out in the case of a trajectory following task,
  failure/stalling.

\item \textbf{Sensor \& perceptual activities} add/delete facts, e.g.,
  contact with an object, perception/recognition of an object, etc.

\item \textbf{Planning, reasoning, sensor processing activities} which
  are computational processes that, as a result, add information or a
  decision to the relation state.
\end{itemize}

However, it may be natural to define activities that are jointly
control \& sensor activities (that heavily interweave control and
force feedback, for instance).


\subsection{Off-line operation}

For stochastic forward simulation of the state, as needed, e.g., in MCTS
planning, the Relational Machine needs to simulate 'itself' as well as
the state changes that would otherwise be triggered by real motor \&
sensor activities. Here we would not aim at realistic (physical)
siulations of the motor \& sensor activities\footnote{this is also an
  interesting option}, but rather leverage approximate, symbolic models
of the state progression. Such models can be learned from previous
real-world executions (as in model-based Relational RL), or hand-coded
as a domain abstraction.

For offline operation we assume that the knowledge base includes rules
that forward predict the state changes (including their real-time
durations) that would otherwise be triggered by real activities. 





\end{document}

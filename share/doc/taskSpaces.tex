\documentclass[10pt,fleqn,twoside]{article}
\usepackage{palatino}
\usepackage{amsmath}
\usepackage{amssymb}
\usepackage{amsfonts}
\usepackage{amsthm}
\usepackage{eucal}
\usepackage{graphicx}
\usepackage{color}

\usepackage[round]{natbib}
\bibliographystyle{abbrvnat}
%\usepackage[german]{babel}
%\usepackage[utf8]{inputenc}

\graphicspath{{pics/}{figs/}{~/write/tex/pics/}{~/write/tex/figs/}{~/teaching/pics-all/}}
\usepackage{geometry}
\geometry{a4paper,hdivide={35mm,*,35mm},vdivide={35mm,*,35mm}}
\renewcommand{\baselinestretch}{1.1}

\newenvironment{items}{
\par\small
\begin{list}{--}{\leftmargin4ex \rightmargin0ex \labelsep1ex \labelwidth2ex
\topsep0pt \parsep0ex \itemsep3pt}
}{
\end{list}
}


  \renewcommand{\a}{\ensuremath\alpha}
  \renewcommand{\b}{\beta}
  \renewcommand{\c}{\gamma}
  \renewcommand{\d}{\delta}
    \newcommand{\D}{\Delta}
    \newcommand{\e}{\epsilon}
    \newcommand{\g}{\gamma}
    \newcommand{\G}{\Gamma}
  \renewcommand{\l}{\lambda}
  \renewcommand{\L}{\Lambda}
    \newcommand{\m}{\mu}
    \newcommand{\n}{\nu}
    \newcommand{\N}{\nabla}
  \renewcommand{\k}{\kappa}
  \renewcommand{\o}{\omega}
  \renewcommand{\O}{\Omega}
    \newcommand{\p}{\phi}
    \newcommand{\ph}{\varphi}
  \renewcommand{\P}{\Phi}
  \renewcommand{\r}{\varrho}
    \newcommand{\s}{\sigma}
  \renewcommand{\S}{\Sigma}
  \renewcommand{\t}{\theta}
    \newcommand{\T}{\Theta}
  %\renewcommand{\v}{\vartheta}
    \newcommand{\x}{\xi}
    \newcommand{\X}{\Xi}
    \newcommand{\Y}{\Upsilon}

  \renewcommand{\AA}{{\cal A}}
    \newcommand{\BB}{{\cal B}}
    \newcommand{\CC}{{\cal C}}
    \newcommand{\EE}{{\cal E}}
    \newcommand{\FF}{{\cal F}}
    \newcommand{\GG}{{\cal G}}
    \newcommand{\HH}{{\cal H}}
    \newcommand{\II}{{\cal I}}
    \newcommand{\KK}{{\cal K}}
    \newcommand{\LL}{{\cal L}}
    \newcommand{\MM}{{\cal M}}
    \newcommand{\NN}{{\cal N}}
    \newcommand{\oNN}{\overline\NN}
    \newcommand{\OO}{{\cal O}}
    \newcommand{\PP}{{\cal P}}
    \newcommand{\QQ}{{\cal Q}}
    \newcommand{\RR}{{\cal R}}
  \renewcommand{\SS}{{\cal S}}
    \newcommand{\TT}{{\cal T}}
    \newcommand{\uu}{{\cal u}}
    \newcommand{\UU}{{\cal U}}
    \newcommand{\VV}{{\cal V}}
    \newcommand{\XX}{{\cal X}}
    \newcommand{\YY}{{\cal Y}}
    \newcommand{\SOSO}{{\cal SO}}
    \newcommand{\GLGL}{{\cal GL}}

    \newcommand{\Ee}{{\rm E}}

  \newcommand{\NNN}{{\mathbb{N}}}
  \newcommand{\ZZZ}{{\mathbb{Z}}}
  %\newcommand{\RRR}{{\mathrm{I\!R}}}
  \newcommand{\RRR}{{\mathbb{R}}}
  \newcommand{\CCC}{{\mathbb{C}}}
  \newcommand{\one}{{{\bf 1}}}
  \newcommand{\eee}{\text{e}}

  \newcommand{\NNNN}{{\overline{\cal N}}}

  \renewcommand{\[}{\Big[}
  \renewcommand{\]}{\Big]}
  \renewcommand{\(}{\Big(}
  \renewcommand{\)}{\Big)}
  \renewcommand{\|}{\,|\,}
  \renewcommand{\=}{\!=\!}
    \newcommand{\<}{{\ensuremath\langle}}
  \renewcommand{\>}{{\ensuremath\rangle}}

  \newcommand{\Prob}{{\rm Prob}}
  \newcommand{\Dir}{{\rm Dir}}
  \newcommand{\Aut}{{\rm Aut}}
  \newcommand{\cor}{{\rm cor}}
  \newcommand{\corr}{{\rm corr}}
  \newcommand{\cov}{{\rm cov}}
  \newcommand{\sd}{{\rm sd}}
  \newcommand{\tr}{{\rm tr}}
  \newcommand{\Tr}{{\rm Tr}}
  \newcommand{\rank}{{\rm rank}}
  \newcommand{\diag}{{\rm diag}}
  \newcommand{\id}{{\rm id}}
  \newcommand{\Id}{{\rm\bf I}}
  \newcommand{\Gl}{{\rm Gl}}
  \renewcommand{\th}{\ensuremath{{}^\text{th}} }
  \newcommand{\lag}{\mathcal{L}}
  \newcommand{\inn}{\rfloor}
  \newcommand{\lie}{\pounds}
  \newcommand{\longto}{\longrightarrow}
  \newcommand{\speer}{\parbox{0.4ex}{\raisebox{0.8ex}{$\nearrow$}}}
  \renewcommand{\dag}{ {}^\dagger }
  \newcommand{\Ji}{J^\sharp}
  \newcommand{\h}{{}^\star}
  \newcommand{\w}{\wedge}
  \newcommand{\too}{\longrightarrow}
  \newcommand{\oot}{\longleftarrow}
  \newcommand{\To}{\Rightarrow}
  \newcommand{\oT}{\Leftarrow}
  \newcommand{\oTo}{\Leftrightarrow}
  \newcommand{\Too}{\;\Longrightarrow\;}
  \newcommand{\oto}{\leftrightarrow}
  \newcommand{\ot}{\leftarrow}
  \newcommand{\ootoo}{\longleftrightarrow}
  \newcommand{\ow}{\stackrel{\circ}\wedge}
  \newcommand{\feed}{\nonumber \\}
  \newcommand{\comma}{~,\quad}
  \newcommand{\period}{~.\quad}
  \newcommand{\del}{\partial}
%  \newcommand{\quabla}{\Delta}
  \newcommand{\point}{$\bullet~~$}
  \newcommand{\doubletilde}{ ~ \raisebox{0.3ex}{$\widetilde {}$} \raisebox{0.6ex}{$\widetilde {}$} \!\! }
  \newcommand{\topcirc}{\parbox{0ex}{~\raisebox{2.5ex}{${}^\circ$}}}
  \newcommand{\topdot} {\parbox{0ex}{~\raisebox{2.5ex}{$\cdot$}}}
  \newcommand{\topddot} {\parbox{0ex}{~\raisebox{1.3ex}{$\ddot{~}$}}}
  \newcommand{\sym}{\topcirc}
  \newcommand{\tsum}{\textstyle\sum}

  \newcommand{\half}{\ensuremath{\frac{1}{2}}}
  \newcommand{\third}{\ensuremath{\frac{1}{3}}}
  \newcommand{\fourth}{\ensuremath{\frac{1}{4}}}

  \newcommand{\ubar}{\underline}
  %\renewcommand{\vec}{\underline}
  \renewcommand{\vec}{\boldsymbol}
  %\renewcommand{\_}{\underset}
  %\renewcommand{\^}{\overset}
  %\renewcommand{\*}{{\rm\raisebox{-.6ex}{\text{*}}{}}}
  \renewcommand{\*}{\text{\footnotesize\raisebox{-.4ex}{*}{}}}

  \newcommand{\gto}{{\raisebox{.5ex}{${}_\rightarrow$}}}
  \newcommand{\gfrom}{{\raisebox{.5ex}{${}_\leftarrow$}}}
  \newcommand{\gnto}{{\raisebox{.5ex}{${}_\nrightarrow$}}}
  \newcommand{\gnfrom}{{\raisebox{.5ex}{${}_\nleftarrow$}}}

  %\newcommand{\RND}{{\SS}}
  %\newcommand{\IF}{\text{if }}
  %\newcommand{\AND}{\textsc{and }}
  %\newcommand{\OR}{\textsc{or }}
  %\newcommand{\XOR}{\textsc{xor }}
  %\newcommand{\NOT}{\textsc{not }}

  %\newcommand{\argmax}[1]{{\rm arg}\!\max_{#1}}
  %\newcommand{\argmin}[1]{{\rm arg}\!\min_{#1}}
  \DeclareMathOperator*{\argmax}{argmax}
  \DeclareMathOperator*{\argmin}{argmin}
  \DeclareMathOperator{\sign}{sign}
  \DeclareMathOperator{\acos}{acos}
  %\newcommand{\argmax}[1]{\underset{~#1}{\text{argmax}}\;}
  %\newcommand{\argmin}[1]{\underset{~#1}{\text{argmin}}\;}
  \newcommand{\ee}[1]{\ensuremath{\cdot10^{#1}}}
  \newcommand{\sub}[1]{\ensuremath{_{\text{#1}}}}
  \newcommand{\up}[1]{\ensuremath{^{\text{#1}}}}
  \newcommand{\kld}[3][{}]{D_{#1}\big(#2\,\big|\!\big|\,#3\big)}
  %\newcommand{\kld}[2]{D\big(#1:#2\big)}
  \newcommand{\sprod}[2]{\big<#1\,,\,#2\big>}
  \newcommand{\End}{\text{End}}
  \newcommand{\txt}[1]{\quad\text{#1}\quad}
  \newcommand{\Over}[2]{\genfrac{}{}{0pt}{0}{#1}{#2}}
  %\newcommand{\mat}[1]{{\bf #1}}
  \newcommand{\arr}[2]{\hspace*{-.5ex}\begin{array}{#1}#2\end{array}\hspace*{-.5ex}}
  \newcommand{\mat}[3][.9]{
    \renewcommand{\arraystretch}{#1}{\scriptscriptstyle{\left(
      \hspace*{-1ex}\begin{array}{#2}#3\end{array}\hspace*{-1ex}
    \right)}}\renewcommand{\arraystretch}{1.2}
  }
  %\newcommand{\case}[1]{\left\{\arr{ll}{#1}\right.}
  \newcommand{\seq}[1]{\textsf{\<#1\>}}
  \newcommand{\seqq}[1]{\textsf{#1}}
  \newcommand{\floor}[1]{\lfloor#1\rfloor}
  \newcommand{\Exp}[2]{{\rm E}_{#1}\{#2\}} \newcommand{\E}{{\rm E}}

  \newcommand{\href}[1]{}
  \newcommand{\url}[2]{\href{#1}{{\color{blue}#2}}}
  \newcommand{\anchor}[2]{\begin{picture}(0,0)\put(#1){#2}\end{picture}}
  \newcommand{\pagebox}{\begin{picture}(0,0)\put(-3,-23){
    \textcolor[rgb]{.5,1,.5}{\framebox[\textwidth]{\rule[-\textheight]{0pt}{0pt}}}}
    \end{picture}}

  \newcommand{\hide}[1]{
    \begin{list}{}{\leftmargin0ex \rightmargin0ex \topsep0ex \parsep0ex}
       \helvetica{5}{1}{m}{n}
       \renewcommand{\section}{\par SECTION: }
       \renewcommand{\subsection}{\par SUBSECTION: }
       \item[$~~\blacktriangleright$]
       #1%$\blacktriangleleft~~$
       \message{^^JHIDE--Warning!^^J}
    \end{list}
  }
  %\newcommand{\hide}[1]{{\tt[hide:~}{\footnotesize\sf #1}{\tt]}\message{^^JHIDE--Warning!^^J}}
  \newcommand{\Hide}{\renewcommand{\hide}[1]{\message{^^JHIDE--Warning (hidden)!^^J}}}
  \newcommand{\HIDE}{\renewcommand{\hide}[1]{}}
  \newcommand{\fullhide}[1]{}
  \newcommand{\todo}[1]{{\tt[TODO: #1]}\message{^^JTODO--Warning: #1^^J}}
  \newcommand{\Todo}{\renewcommand{\todo}[1]{\message{^^JTODO--Warning (hidden)!^^J}}}
  %\renewcommand{\title}[1]{\renewcommand{\thetitle}{#1}}
  \newcommand{\header}{\begin{document}\mytitle\cleardefs}
  \newcommand{\contents}{{\tableofcontents}\renewcommand{\contents}{}}
  \newcommand{\footer}{\small\bibliography{bibs}\end{document}}
  \newcommand{\widepaper}{\usepackage{geometry}\geometry{a4paper,hdivide={25mm,*,25mm},vdivide={25mm,*,25mm}}}
  \newcommand{\moviex}[2]{\movie[externalviewer]{#1}{#2}} %\pdflatex\usepackage{multimedia}
  \newcommand{\rbox}[1]{\fboxrule2mm\fcolorbox[rgb]{1,.85,.85}{1,.85,.85}{#1}}
  \newcommand{\redbox}[2]{\fboxrule2mm\fcolorbox[rgb]{1,.7,.7}{1,.7,.7}{\begin{minipage}{#1}\center#2\end{minipage}}}
  \newcommand{\twocol}[5][c]{\begin{minipage}[#1]{#2\columnwidth}#4\end{minipage}\begin{minipage}[#1]{#3\columnwidth}#5\end{minipage}}
  \newcommand{\threecoltext}[7][c]{\begin{minipage}[#1]{#2\textwidth}#5\end{minipage}\begin{minipage}[#1]{#3\textwidth}#6\end{minipage}\begin{minipage}[#1]{#4\textwidth}#7\end{minipage}}
  \newcommand{\threecol}[7][c]{\begin{minipage}[#1]{#2\columnwidth}#5\end{minipage}\begin{minipage}[#1]{#3\columnwidth}#6\end{minipage}\begin{minipage}[#1]{#4\columnwidth}#7\end{minipage}}
  \newcommand{\helvetica}[4]{\setlength{\unitlength}{1pt}\fontsize{#1}{#1}\linespread{#2}\usefont{OT1}{phv}{#3}{#4}}
  \newcommand{\helve}[1]{\helvetica{#1}{1.5}{m}{n}}
  \newcommand{\german}{\usepackage[german]{babel}\usepackage[latin1]{inputenc}}

\newcommand{\norm}[2]{|\!|#1|\!|_{{}_{#2}}}
\newcommand{\expr}[1]{[\hspace{-.2ex}[#1]\hspace{-.2ex}]}

\newcommand{\Jwi}{J^\sharp_W}
\newcommand{\Twi}{T^\sharp_W}
\newcommand{\Jci}{J^\natural_C}
\newcommand{\hJi}{{\bar J}^\sharp}
\renewcommand{\|}{\,|\,}
\renewcommand{\=}{\!=\!}
\newcommand{\myminus}{{\hspace*{-.3pt}\text{\rm -}\hspace*{-.5pt}}}
\newcommand{\myplus}{{\hspace*{-.3pt}\text{\rm +}\hspace*{-.5pt}}}
\newcommand{\1}{{\myminus1}}
\newcommand{\2}{{\myminus2}}
\newcommand{\3}{{\myminus3}}
\newcommand{\mT}{{\myminus{}\top}}
\newcommand{\po}{{\myplus1}}
\newcommand{\pt}{{\myplus2}}
%\renewcommand{\-}{\myminus}
%\newcommand{\+}{\myplus}
\renewcommand{\T}{{\!\top\!}}
\renewcommand{\mT}{{\myminus{}\top}}

\newenvironment{centy}{
\vspace{15mm}
\large
\hspace*{5mm}
\begin{minipage}{8cm}\it\color{blue}
}{
\end{minipage}
}

\newcommand{\old}{{\text{old}}}
\newcommand{\new}{{\text{new}}}


%%%%%%%%%%%%%%%%%%%%%%%%%%%%%%%%%%%%%%%%%%%%%%%%%%%%%%%%%%%%%%%%%%%%%%%%%%%%%%%%
\pdflatex
\usepackage{framed}
  \definecolor{shadecolor}{gray}{0.9}
  \setlength{\FrameSep}{3pt}
\usepackage{fancyvrb}
\DefineShortVerb{\@}

\newcommand{\pos}{{\textsf{p}}}
\newcommand{\eff}{{\textsf{eff}}}
\newcommand{\rotvec}{{\textsf{w}}}
\newcommand{\veC}{{\textsf{v}}}
\newcommand{\quat}{{\textsf{q}}}
\newcommand{\col}{{\textsf{col}}}
\newcommand{\TR}[2]{T_{{#1}\rightarrow{#2}}}
\newcommand{\RO}[2]{R_{{#1}\rightarrow{#2}}}

%%%%%%%%%%%%%%%%%%%%%%%%%%%%%%%%%%%%%%%%%%%%%%%%%%%%%%%%%%%%%%%%%%%%%%%%%%%%%%%

\title{Task Spaces}
\author{M Toussaint}

\begin{document}
\maketitle

\section{Purpose}

Task spaces are defined by a mapping $\phi:~ q\to y$ from the joint
state $q\in\RRR^n$ to a task space $y\in\RRR^m$. They are central in
designing motion and manipulation, both, in the context of trajectory
optimization as well as in specifying position/force/impedance
controllers:

For \textbf{trajectory optimization}, cost functions are defined by
costs or constraints in task spaces. Given a single task space $\phi$,
we may define
\begin{items}
\item costs $\norm{\phi(q)}^2$,
\item an inequality constraint $\phi(q)\le 0$ (element-wise inequality),
\item an equality constraint $\phi(q) = 0$.
\end{items}
All three assume that the `target' is at zero. For convenience, the
code allows to specify an additional linear transform $\tilde\phi(q)
\gets \rho(\phi(q)-y_\text{ref})$, defined by a target reference
$y_\text{ref}$ and a scaling $\rho$. In KOMO, costs and constraints
can also be defined on $k+1$-tuples of consecutive states in task
space, allowing to have cost and constraints on task space velocities
or accelerations. Trajectory optimization problems are defined by many
such costs/constraints in various task spaces at various time steps.

For simple \textbf{feedback control}, in each task space we may have
\begin{items}
\item a desired linear acceleration behavior in the task space
\item a desired force or force constraint (upper bound) in the task
  space
\item a desired impedance around a reference
\end{items}
All of these can be fused to a joint-level force-feedback controller
(details, see controller docu). On the higher level, the control mode
is specified by defining multiple task spaces and the desired
behaviors in these. (The activity of such tasks (on the symbolic
level) is controlled by the RelationalMachine, see its docu.)

In both cases, defining task spaces is the core.

\section{Basic notation}

We follow the notation in the robotics lecture slides. We enumerate
all bodies by $i\in\BB$. We typically use $v,w\in\RRR^3$ to denote
vectors attached to bodies. $\TR{W}{i}$ is the 4-by-4 homogeneous
transform from world frame to the frame of shape $i$, and $\RO{W}{i}$
is its rotation matrix only. In the text (not in equations) we
sometimes write
$$(i+ v)$$
where $i$ denotes a body and $v\in\RRR^3$ a relative 3D-vector. The
rigorous notation for this would be $\TR{W}{i} v$, which is the
position of $i$ plus the relative displacement $v$.

To numerically evaluate kinematics we assume that, for a certain joint
configuration $q$, the positions $p_k$ and axes $a_k$ of all joints
$k\in JJ$ have been precomputed. The boolean expression
$$[k\prec i] $$ iff joint $k$ is ``below'' body $i$ in the kinematic
tree, that is, joint $k$ is between root and body $i$ and therefore
moves it.

Sometimes we write $J_{\cdot k}=...$, which means that the $k$th
column of $J$ is defined as given.

Let's first define
\begin{align}
(A_i)_{\cdot k}
&= [k \prec i] [\text{$i$ rotational}]~ a_k && \text{axes matrix below $i$}
\end{align}
This matrix contains all rotational axes below $i$ as columns and will
turn out convenient, because it captures all axes that make $i$
move. Many Jacobians can easily be described using $A_i$. Analogously
we define
\begin{align}
(T_i)_{\cdot k}
&= [k \prec i] [\text{$i$ prismatic}]~ a_k && \text{prism matrix below $i$}
\end{align}
This captures all prismatic joints. Note the following relation to
Featherstone's notation: In his notation, $h_k\in\RRR^6$ denotes, for
very joint $k$, how much the joints contributes to rotation and translation,
expressed in the link frame. The two matrices $A_i$ and $T_i$ together
express the same, but in world coordinates. While typically axes have
unit length (and entries of $h$ are zeros or ones), this is not necessary
in general, allowing for arbitrary scaling of joint configurations $q$
with these axis (e.g., using degree instead of radial units).

For convenience, for a matrix of 3D columns $A\in\RRR^{n\times 3}$, we write
$$B = A\times p \iff B_{\cdot k} = A_{\cdot k} \times p$$
which is the column-wise cross product. Also, we define
\begin{align}
(\hat A_i)_{\cdot k}
&= [k \prec i] [\text{$i$ rotational}]~ a_k \times p_k && \text{axes
    position matrix below $i$}
\end{align}
which could also be written as $\hat A_i = A_i \times P$ if $P$
contains all axes positions.

%%%%%%%%%%%%%%%%%%%%%%%%%%%%%%%%%%%%%%%%%%%%%%%%%%%%%%%%%%%%%%%%%%%%%%%%%%%%%%%%
\section{Task Spaces}

\paragraph{Position}

\begin{align}
\phi^\pos_{iv}(q)
 &= \TR{W}{i}~ v
 && \text{position of $(i+v)$} \\
J^\pos_{iv}(q)_{\cdot k}
 &= [k\prec i]~ a_k \times(\phi^\pos_{iv}(q) - p_k) \\
J^\pos_{iv}(q)
 &= A_i \times \phi^\pos_{iv}(q) - \hat A_i \\
\phi^\pos_{iv-jw}(q)
 &= \phi^\pos_{iv} - \phi^\pos_{jw}
 && \text{position difference} \\
J^\pos_{iv-jw}(q)
 & = J^\pos_{iv} - J^\pos_{jw} \\
\phi^\pos_{iv|jw}(q)
 & = R_j^\1 (\phi^\pos_{iv} - \phi^\pos_{jw})
 && \text{relative position} \\
J^\pos_{iv|jw}(q)
 & = R_j^\1 [J^\pos_{iv}-J^\pos_{jw} - A_j \times (\phi^\pos_{iv} - \phi^\pos_{jw})]
\end{align}

Derivation: For $y=R p$ the derivative w.r.t.\ a rotation around axis
$a$ is $y' = R p' + R' p = R p' + a \times R p$. For $y=R^\1 p$ the
derivative is $y' = R^\1 p' - R^\1 (R') R^\1 p = R^\1 (p' - a \times
p)$.  (For details see
\url{http://ipvs.informatik.uni-stuttgart.de/mlr/marc/notes/3d-geometry.pdf})

%%%%%%%%%%%%%%%%%%%%%%%%%%%%%%%%%%%%%%%%%%%%%%%%%%%%%%%%%%%%%%%%%%%%%%%%%%%%%%%%
\paragraph{Vector}

\begin{align}
\phi^\veC_{iv}(q)
 &= \RO{W}{i}~ v
 && \text{vector} \\
J^\veC_{iv}(q)
 &= A_i\times \phi^\veC_{iv}(q) \\
\phi^\veC_{iv-jw}(q)
 &= \phi^\veC_{iv} - \phi^\veC_{jw}
 && \text{vector difference} \\
J^\veC_{iv-jw}(q)
 &= J^\veC_{iv} - J^\veC_{jw} \\
\phi^\veC_{iv|j}(q)
 &= R_j^\1 \phi^\veC_{iv}
 && \text{relative vector} \\
J^\veC_{iv|j}(q)
 &= R_j^\1 [J^\veC_{iv} - A_j \times \phi^\veC_{iv}]
\end{align}

%%%%%%%%%%%%%%%%%%%%%%%%%%%%%%%%%%%%%%%%%%%%%%%%%%%%%%%%%%%%%%%%%%%%%%%%%%%%%%%%

\subsection{Quaternion}

See equation (15) in the geometry notes for explaining the jacobian.
\begin{align}
\phi^\quat_{i}(q)
 &= \text{quaternion}(\RO{W}{i})
 && \text{quaternion $\in\RRR^4$} \\
J^\quat_{i}(q)_{\cdot k}
 &= \mat{c}{0 \\ \half (A_i)_{\cdot k}} \circ \phi^\quat_{i}(q)
 && J^\quat_{i}(q)\in\RRR^{4 \times n} \\
\phi^\quat_{i-j}(q)
 &= \phi^\quat_{i} - \phi^\quat_{j}
 && \text{difference $\in\RRR^4$} \\
J^\quat_{i-j}(q)
 &= J^\quat_{i} - J^\quat_{j} \\
\phi^\quat_{i|j}(q)
 &= (\phi^\quat_{j})^\1 \circ \phi^\quat_{i}
 && \text{relative} \\
J^\quat_{i|j}(q)
 &= \text{not implemented}
\end{align}

A relative rotation can also be measured in terms of the 3D rotation
vector. Lets define
$$w(r) = \frac{2 \p}{\sin(\p)} \bar r \comma
\p=\acos(r_0)$$
as the rotation for a quaternion. We have
\begin{align}
\phi^\rotvec_{i|j}(q)
 &= w(\phi^\quat_{j})^\1 \circ \phi^\quat_{i})
 && \text{relative rotation vector $\in\RRR^3$} \\
J^\rotvec_{i|j}(q)
 &= AJ^\quat_{i} - J^\quat_{j} \\
\end{align}



%%%%%%%%%%%%%%%%%%%%%%%%%%%%%%%%%%%%%%%%%%%%%%%%%%%%%%%%%%%%%%%%%%%%%%%%%%%%%%%%

\subsection{Alignment}

parameters: shape indices $i,j$, attached vectors $v,w$

$\phi^\text{align}_{iv|jw}(q) = (\phi^\veC_{jw})^\T~ \phi^\veC_{iv}$

$J^\text{align}_{iv|jw}(q) = (\phi^\veC_{jw})^\T~ J^\veC_{iv} +\phi^\veC_{iv}{}^\T~ J^\veC_{jw}$

Note: \quad $\phi^\text{align}=1 \oto $ align \quad $\phi^\text{align}=-1 \oto $ anti-align \quad $\phi^\text{align}=0 \oto $ orthog.

%%%%%%%%%%%%%%%%%%%%%%%%%%%%%%%%%%%%%%%%%%%%%%%%%%%%%%%%%%%%%%%%%%%%%%%%%%%%%%%%

\subsection{Gaze}

2D orthogonality measure of object relative to camera plane

parameters: eye index $i$ with offset $v$; target index $j$ with
offset $w$

\begin{align}
\phi^\text{gaze}_{iv,jw}(q)
 &= \mat{c}{
\phi^\veC_{i,e_x}{}^\T (\phi^\pos_{jw} - \phi^\pos_{iv}) \\
\phi^\veC_{i,e_y}{}^\T (\phi^\pos_{jw} - \phi^\pos_{iv}) } \quad\in\RRR^2
\end{align}
Here $e_x=(1,0,0)$ and $e_y=(0,1,0)$ are the camera plane axes.

Jacobians straight-forward

%%%%%%%%%%%%%%%%%%%%%%%%%%%%%%%%%%%%%%%%%%%%%%%%%%%%%%%%%%%%%%%%%%%%%%%%%%%%%%%%

\subsection{qItself}

$\phi^\text{qItself}_{iv,jw}(q) = q$

%%%%%%%%%%%%%%%%%%%%%%%%%%%%%%%%%%%%%%%%%%%%%%%%%%%%%%%%%%%%%%%%%%%%%%%%%%%%%%%%

\subsection{Joint limits measure}

parameters: joint limits $q_{\text{low}}, q_{\text{hi}}$, margin $m$

$\phi_{\text{limits}}(q) = \frac{1}{m}\sum_{i=1}^n [m-q_i+q_{\text{low}}]^+ + [m+q_i-q_{\text{hi}}]^+$

$J_{\text{limits}}(q)_{1,i} = - \frac{1}{m}[m-q_i+q_{\text{low}}>0] + \frac{1}{m}[m+q_i-q_{\text{hi}}>0]$

$[x]^+ = x>0\text{?}x:0$ \qquad $[\cdots]$: indicator function

%%%%%%%%%%%%%%%%%%%%%%%%%%%%%%%%%%%%%%%%%%%%%%%%%%%%%%%%%%%%%%%%%%%%%%%%%%%%%%%%

\subsection{Collision limits measure}

parameters: margin $m$

$\phi_{\text{col}}(q) = \frac{1}{m}\sum_{k=1}^K [m-|p^a_k - p^b_k|]^+$

$J_{\text{col}}(q) = \frac{1}{m} \sum_{k=1}^K [m-|p^a_k - p^b_k|>0]
(- J^\pos_{p^a_k} + J^\pos_{p^b_k})^\T \frac{p^a_k - p^b_k}{|p^a_k - p^b_k|}$ 

A collision detection engine returns a set $\{
(a,b,p^a,p^b)_{k=1}^K \}$ of potential collisions between shape $a_k$
and $b_k$, with nearest points $p^a_k$ on $a$ and $p^b_k$ on $b$.

%%%%%%%%%%%%%%%%%%%%%%%%%%%%%%%%%%%%%%%%%%%%%%%%%%%%%%%%%%%%%%%%%%%%%%%%%%%%%%%%

\subsection{Shape distance measures (using SWIFT)}

\begin{items}
\item  allPTMT, //phi=sum over all proxies (as is standard)
\item  listedVsListedPTMT, //phi=sum over all proxies between listed shapes
\item  allVsListedPTMT, //phi=sum over all proxies against listed shapes
\item  allExceptListedPTMT, //as above, but excluding listed shapes
\item  bipartitePTMT, //sum over proxies between the two sets of shapes (shapes, shapes2)
\item  pairsPTMT, //sum over proxies of explicitly listed pairs (shapes is n-times-2)
\item  allExceptPairsPTMT, //sum excluding these pairs
\item  vectorPTMT //vector of all pair proxies (this is the only case
  where dim(phi)>1)
\end{items}

\subsection{GJK pairwise shape distance (including negative)}

\subsection{Plane distance}

\section{Application}

Just get a glimpse on how task space definitions are used to script
motions, here is a script of a little PR2 dance. (The 'logic' below
the script implements kind of macros -- that's part of the RAP.)
(\texttt{wheels} is the same as qItself, but refers only to the 3 base coordinates)

\begin{shaded}
\begin{Verbatim}[fontfamily=courier,fontsize=\tiny]
cleanAll #this only declares a novel symbol...

Script {
  (FollowReferenceActivity wheels){ type=wheels, target=[0, .3, .2], PD=[.5, .9, .5, 10.]}
  (MyTask endeffR){ type=pos, ref2=base_footprint, target=[.2, -.5, 1.3], PD=[.5, .9, .5, 10.]}
  (MyTask endeffL){ type=pos, ref2=base_footprint, target=[.2, +.5, 1.3], PD=[.5, .9, .5, 10.]}
  { (conv FollowReferenceActivity wheels)  (conv MyTask endeffR) }  #this waits for convergence of activities
  (cleanAll) #this switches off the current activities
  (cleanAll)! #this switches off the switching-off

  (FollowReferenceActivity wheels){ type=wheels, target=[0, -.3, -.2], PD=[.5, .9, .5, 10.]}
  (MyTask endeffR){ type=pos, ref2=base_footprint, target=[.7, -.2, .7], PD=[.5, .9, .5, 10.]}
  (MyTask endeffL){ type=pos, ref2=base_footprint, target=[.7, +.2, .7], PD=[.5, .9, .5, 10.]}
  { (conv FollowReferenceActivity wheels)  (conv MyTask endeffL) }
  (cleanAll)
  (cleanAll)!

  (FollowReferenceActivity wheels){ type=wheels, target=[0, .3, .2], PD=[.5, .9, .5, 10.]}
  (MyTask endeffR){ type=pos, ref2=base_footprint, target=[.2, -.5, 1.3], PD=[.5, .9, .5, 10.]}
  (MyTask endeffL){ type=pos, ref2=base_footprint, target=[.2, +.5, 1.3], PD=[.5, .9, .5, 10.]}
  { (conv MyTask endeffL) }
  (cleanAll)
  (cleanAll)!

  (FollowReferenceActivity wheels){ type=wheels, target=[0, 0, 0], PD=[.5, .9, .5, 10.]}
  (HomingActivity)
  { (conv HomingActivity) (conv FollowReferenceActivity wheels) }
}


Rule {
    X, Y, 
    { (cleanAll) (conv X Y) }
    { (conv X Y)! }
}

Rule {
    X, 
    { (cleanAll) (MyTask X) }
    { (MyTask X)! }
}

Rule {
    X, 
    { (cleanAll) (FollowReferenceActivity X) }
    { (FollowReferenceActivity X)! }
}
\end{Verbatim}
\end{shaded}



\end{document}

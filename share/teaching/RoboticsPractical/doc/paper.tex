\documentclass[10pt,fleqn,twoside]{article}
\usepackage{palatino}
\usepackage{amsmath}
\usepackage{amssymb}
\usepackage{amsfonts}
\usepackage{amsthm}
\usepackage{eucal}
\usepackage{graphicx}
\usepackage{color}

\usepackage[round]{natbib}
\bibliographystyle{abbrvnat}
%\usepackage[german]{babel}
%\usepackage[utf8]{inputenc}

\usepackage{fancyvrb}
\DefineShortVerb{\@}
\fvset{numbers=left,xleftmargin=5ex}

\graphicspath{{pics/}{figs/}{~/write/tex/pics/}{~/write/tex/figs/}{~/teaching/pics-all/}}
\usepackage{geometry}
\geometry{a4paper,hdivide={35mm,*,35mm},vdivide={35mm,*,35mm}}
\renewcommand{\baselinestretch}{1.1}

\newcommand{\rf}{{\text{ref}}}
\newcommand{\eig}{{\text{eig}}}

%% \newenvironment{items}{
%% \par\small
%% \begin{list}{--}{\leftmargin4ex \rightmargin0ex \labelsep1ex \labelwidth2ex
%% \topsep0pt \parsep0ex \itemsep3pt}
%% }{
%% \end{list}
%% }

%
  \renewcommand{\a}{\ensuremath\alpha}
  \renewcommand{\b}{\beta}
  \renewcommand{\c}{\gamma}
  \renewcommand{\d}{\delta}
    \newcommand{\D}{\Delta}
    \newcommand{\e}{\epsilon}
    \newcommand{\g}{\gamma}
    \newcommand{\G}{\Gamma}
  \renewcommand{\l}{\lambda}
  \renewcommand{\L}{\Lambda}
    \newcommand{\m}{\mu}
    \newcommand{\n}{\nu}
    \newcommand{\N}{\nabla}
  \renewcommand{\k}{\kappa}
  \renewcommand{\o}{\omega}
  \renewcommand{\O}{\Omega}
    \newcommand{\p}{\phi}
    \newcommand{\ph}{\varphi}
  \renewcommand{\P}{\Phi}
  \renewcommand{\r}{\varrho}
    \newcommand{\s}{\sigma}
  \renewcommand{\S}{\Sigma}
  \renewcommand{\t}{\theta}
    \newcommand{\T}{\Theta}
  %\renewcommand{\v}{\vartheta}
    \newcommand{\x}{\xi}
    \newcommand{\X}{\Xi}
    \newcommand{\Y}{\Upsilon}

  \renewcommand{\AA}{{\cal A}}
    \newcommand{\BB}{{\cal B}}
    \newcommand{\CC}{{\cal C}}
    \newcommand{\EE}{{\cal E}}
    \newcommand{\FF}{{\cal F}}
    \newcommand{\GG}{{\cal G}}
    \newcommand{\HH}{{\cal H}}
    \newcommand{\II}{{\cal I}}
    \newcommand{\KK}{{\cal K}}
    \newcommand{\LL}{{\cal L}}
    \newcommand{\MM}{{\cal M}}
    \newcommand{\NN}{{\cal N}}
    \newcommand{\oNN}{\overline\NN}
    \newcommand{\OO}{{\cal O}}
    \newcommand{\PP}{{\cal P}}
    \newcommand{\QQ}{{\cal Q}}
    \newcommand{\RR}{{\cal R}}
  \renewcommand{\SS}{{\cal S}}
    \newcommand{\TT}{{\cal T}}
    \newcommand{\uu}{{\cal u}}
    \newcommand{\UU}{{\cal U}}
    \newcommand{\VV}{{\cal V}}
    \newcommand{\XX}{{\cal X}}
    \newcommand{\YY}{{\cal Y}}
    \newcommand{\SOSO}{{\cal SO}}
    \newcommand{\GLGL}{{\cal GL}}

    \newcommand{\Ee}{{\rm E}}

  \newcommand{\NNN}{{\mathbb{N}}}
  \newcommand{\ZZZ}{{\mathbb{Z}}}
  %\newcommand{\RRR}{{\mathrm{I\!R}}}
  \newcommand{\RRR}{{\mathbb{R}}}
  \newcommand{\CCC}{{\mathbb{C}}}
  \newcommand{\one}{{{\bf 1}}}
  \newcommand{\eee}{\text{e}}

  \newcommand{\NNNN}{{\overline{\cal N}}}

  \renewcommand{\[}{\Big[}
  \renewcommand{\]}{\Big]}
  \renewcommand{\(}{\Big(}
  \renewcommand{\)}{\Big)}
  \renewcommand{\|}{\,|\,}
  \renewcommand{\=}{\!=\!}
    \newcommand{\<}{{\ensuremath\langle}}
  \renewcommand{\>}{{\ensuremath\rangle}}

  \newcommand{\Prob}{{\rm Prob}}
  \newcommand{\Dir}{{\rm Dir}}
  \newcommand{\Aut}{{\rm Aut}}
  \newcommand{\cor}{{\rm cor}}
  \newcommand{\corr}{{\rm corr}}
  \newcommand{\cov}{{\rm cov}}
  \newcommand{\sd}{{\rm sd}}
  \newcommand{\tr}{{\rm tr}}
  \newcommand{\Tr}{{\rm Tr}}
  \newcommand{\rank}{{\rm rank}}
  \newcommand{\diag}{{\rm diag}}
  \newcommand{\id}{{\rm id}}
  \newcommand{\Id}{{\rm\bf I}}
  \newcommand{\Gl}{{\rm Gl}}
  \renewcommand{\th}{\ensuremath{{}^\text{th}} }
  \newcommand{\lag}{\mathcal{L}}
  \newcommand{\inn}{\rfloor}
  \newcommand{\lie}{\pounds}
  \newcommand{\longto}{\longrightarrow}
  \newcommand{\speer}{\parbox{0.4ex}{\raisebox{0.8ex}{$\nearrow$}}}
  \renewcommand{\dag}{ {}^\dagger }
  \newcommand{\Ji}{J^\sharp}
  \newcommand{\h}{{}^\star}
  \newcommand{\w}{\wedge}
  \newcommand{\too}{\longrightarrow}
  \newcommand{\oot}{\longleftarrow}
  \newcommand{\To}{\Rightarrow}
  \newcommand{\oT}{\Leftarrow}
  \newcommand{\oTo}{\Leftrightarrow}
  \newcommand{\Too}{\;\Longrightarrow\;}
  \newcommand{\oto}{\leftrightarrow}
  \newcommand{\ot}{\leftarrow}
  \newcommand{\ootoo}{\longleftrightarrow}
  \newcommand{\ow}{\stackrel{\circ}\wedge}
  \newcommand{\feed}{\nonumber \\}
  \newcommand{\comma}{~,\quad}
  \newcommand{\period}{~.\quad}
  \newcommand{\del}{\partial}
%  \newcommand{\quabla}{\Delta}
  \newcommand{\point}{$\bullet~~$}
  \newcommand{\doubletilde}{ ~ \raisebox{0.3ex}{$\widetilde {}$} \raisebox{0.6ex}{$\widetilde {}$} \!\! }
  \newcommand{\topcirc}{\parbox{0ex}{~\raisebox{2.5ex}{${}^\circ$}}}
  \newcommand{\topdot} {\parbox{0ex}{~\raisebox{2.5ex}{$\cdot$}}}
  \newcommand{\topddot} {\parbox{0ex}{~\raisebox{1.3ex}{$\ddot{~}$}}}
  \newcommand{\sym}{\topcirc}
  \newcommand{\tsum}{\textstyle\sum}

  \newcommand{\half}{\ensuremath{\frac{1}{2}}}
  \newcommand{\third}{\ensuremath{\frac{1}{3}}}
  \newcommand{\fourth}{\ensuremath{\frac{1}{4}}}

  \newcommand{\ubar}{\underline}
  %\renewcommand{\vec}{\underline}
  \renewcommand{\vec}{\boldsymbol}
  %\renewcommand{\_}{\underset}
  %\renewcommand{\^}{\overset}
  %\renewcommand{\*}{{\rm\raisebox{-.6ex}{\text{*}}{}}}
  \renewcommand{\*}{\text{\footnotesize\raisebox{-.4ex}{*}{}}}

  \newcommand{\gto}{{\raisebox{.5ex}{${}_\rightarrow$}}}
  \newcommand{\gfrom}{{\raisebox{.5ex}{${}_\leftarrow$}}}
  \newcommand{\gnto}{{\raisebox{.5ex}{${}_\nrightarrow$}}}
  \newcommand{\gnfrom}{{\raisebox{.5ex}{${}_\nleftarrow$}}}

  %\newcommand{\RND}{{\SS}}
  %\newcommand{\IF}{\text{if }}
  %\newcommand{\AND}{\textsc{and }}
  %\newcommand{\OR}{\textsc{or }}
  %\newcommand{\XOR}{\textsc{xor }}
  %\newcommand{\NOT}{\textsc{not }}

  %\newcommand{\argmax}[1]{{\rm arg}\!\max_{#1}}
  %\newcommand{\argmin}[1]{{\rm arg}\!\min_{#1}}
  \DeclareMathOperator*{\argmax}{argmax}
  \DeclareMathOperator*{\argmin}{argmin}
  \DeclareMathOperator{\sign}{sign}
  \DeclareMathOperator{\acos}{acos}
  %\newcommand{\argmax}[1]{\underset{~#1}{\text{argmax}}\;}
  %\newcommand{\argmin}[1]{\underset{~#1}{\text{argmin}}\;}
  \newcommand{\ee}[1]{\ensuremath{\cdot10^{#1}}}
  \newcommand{\sub}[1]{\ensuremath{_{\text{#1}}}}
  \newcommand{\up}[1]{\ensuremath{^{\text{#1}}}}
  \newcommand{\kld}[3][{}]{D_{#1}\big(#2\,\big|\!\big|\,#3\big)}
  %\newcommand{\kld}[2]{D\big(#1:#2\big)}
  \newcommand{\sprod}[2]{\big<#1\,,\,#2\big>}
  \newcommand{\End}{\text{End}}
  \newcommand{\txt}[1]{\quad\text{#1}\quad}
  \newcommand{\Over}[2]{\genfrac{}{}{0pt}{0}{#1}{#2}}
  %\newcommand{\mat}[1]{{\bf #1}}
  \newcommand{\arr}[2]{\hspace*{-.5ex}\begin{array}{#1}#2\end{array}\hspace*{-.5ex}}
  \newcommand{\mat}[3][.9]{
    \renewcommand{\arraystretch}{#1}{\scriptscriptstyle{\left(
      \hspace*{-1ex}\begin{array}{#2}#3\end{array}\hspace*{-1ex}
    \right)}}\renewcommand{\arraystretch}{1.2}
  }
  %\newcommand{\case}[1]{\left\{\arr{ll}{#1}\right.}
  \newcommand{\seq}[1]{\textsf{\<#1\>}}
  \newcommand{\seqq}[1]{\textsf{#1}}
  \newcommand{\floor}[1]{\lfloor#1\rfloor}
  \newcommand{\Exp}[2]{{\rm E}_{#1}\{#2\}} \newcommand{\E}{{\rm E}}

  \newcommand{\href}[1]{}
  \newcommand{\url}[2]{\href{#1}{{\color{blue}#2}}}
  \newcommand{\anchor}[2]{\begin{picture}(0,0)\put(#1){#2}\end{picture}}
  \newcommand{\pagebox}{\begin{picture}(0,0)\put(-3,-23){
    \textcolor[rgb]{.5,1,.5}{\framebox[\textwidth]{\rule[-\textheight]{0pt}{0pt}}}}
    \end{picture}}

  \newcommand{\hide}[1]{
    \begin{list}{}{\leftmargin0ex \rightmargin0ex \topsep0ex \parsep0ex}
       \helvetica{5}{1}{m}{n}
       \renewcommand{\section}{\par SECTION: }
       \renewcommand{\subsection}{\par SUBSECTION: }
       \item[$~~\blacktriangleright$]
       #1%$\blacktriangleleft~~$
       \message{^^JHIDE--Warning!^^J}
    \end{list}
  }
  %\newcommand{\hide}[1]{{\tt[hide:~}{\footnotesize\sf #1}{\tt]}\message{^^JHIDE--Warning!^^J}}
  \newcommand{\Hide}{\renewcommand{\hide}[1]{\message{^^JHIDE--Warning (hidden)!^^J}}}
  \newcommand{\HIDE}{\renewcommand{\hide}[1]{}}
  \newcommand{\fullhide}[1]{}
  \newcommand{\todo}[1]{{\tt[TODO: #1]}\message{^^JTODO--Warning: #1^^J}}
  \newcommand{\Todo}{\renewcommand{\todo}[1]{\message{^^JTODO--Warning (hidden)!^^J}}}
  %\renewcommand{\title}[1]{\renewcommand{\thetitle}{#1}}
  \newcommand{\header}{\begin{document}\mytitle\cleardefs}
  \newcommand{\contents}{{\tableofcontents}\renewcommand{\contents}{}}
  \newcommand{\footer}{\small\bibliography{bibs}\end{document}}
  \newcommand{\widepaper}{\usepackage{geometry}\geometry{a4paper,hdivide={25mm,*,25mm},vdivide={25mm,*,25mm}}}
  \newcommand{\moviex}[2]{\movie[externalviewer]{#1}{#2}} %\pdflatex\usepackage{multimedia}
  \newcommand{\rbox}[1]{\fboxrule2mm\fcolorbox[rgb]{1,.85,.85}{1,.85,.85}{#1}}
  \newcommand{\redbox}[2]{\fboxrule2mm\fcolorbox[rgb]{1,.7,.7}{1,.7,.7}{\begin{minipage}{#1}\center#2\end{minipage}}}
  \newcommand{\twocol}[5][c]{\begin{minipage}[#1]{#2\columnwidth}#4\end{minipage}\begin{minipage}[#1]{#3\columnwidth}#5\end{minipage}}
  \newcommand{\threecoltext}[7][c]{\begin{minipage}[#1]{#2\textwidth}#5\end{minipage}\begin{minipage}[#1]{#3\textwidth}#6\end{minipage}\begin{minipage}[#1]{#4\textwidth}#7\end{minipage}}
  \newcommand{\threecol}[7][c]{\begin{minipage}[#1]{#2\columnwidth}#5\end{minipage}\begin{minipage}[#1]{#3\columnwidth}#6\end{minipage}\begin{minipage}[#1]{#4\columnwidth}#7\end{minipage}}
  \newcommand{\helvetica}[4]{\setlength{\unitlength}{1pt}\fontsize{#1}{#1}\linespread{#2}\usefont{OT1}{phv}{#3}{#4}}
  \newcommand{\helve}[1]{\helvetica{#1}{1.5}{m}{n}}
  \newcommand{\german}{\usepackage[german]{babel}\usepackage[latin1]{inputenc}}

\newcommand{\norm}[2]{|\!|#1|\!|_{{}_{#2}}}
\newcommand{\expr}[1]{[\hspace{-.2ex}[#1]\hspace{-.2ex}]}

\newcommand{\Jwi}{J^\sharp_W}
\newcommand{\Twi}{T^\sharp_W}
\newcommand{\Jci}{J^\natural_C}
\newcommand{\hJi}{{\bar J}^\sharp}
\renewcommand{\|}{\,|\,}
\renewcommand{\=}{\!=\!}
\newcommand{\myminus}{{\hspace*{-.3pt}\text{\rm -}\hspace*{-.5pt}}}
\newcommand{\myplus}{{\hspace*{-.3pt}\text{\rm +}\hspace*{-.5pt}}}
\newcommand{\1}{{\myminus1}}
\newcommand{\2}{{\myminus2}}
\newcommand{\3}{{\myminus3}}
\newcommand{\mT}{{\myminus{}\top}}
\newcommand{\po}{{\myplus1}}
\newcommand{\pt}{{\myplus2}}
%\renewcommand{\-}{\myminus}
%\newcommand{\+}{\myplus}
\renewcommand{\T}{{\!\top\!}}
\renewcommand{\mT}{{\myminus{}\top}}

\newenvironment{centy}{
\vspace{15mm}
\large
\hspace*{5mm}
\begin{minipage}{8cm}\it\color{blue}
}{
\end{minipage}
}

\newcommand{\old}{{\text{old}}}
\newcommand{\new}{{\text{new}}}


%%%%%%%%%%%%%%%%%%%%%%%%%%%%%%%%%%%%%%%%%%%%%%%%%%%%%%%%%%%%%%%%%%%%%%%%%%%%%%%%
\title{Practical Course Robotics}
\author{Marc Toussaint}

\begin{document}
\maketitle

{\small\tableofcontents }

\section{Introduction}

\section{Setting up your work environment}

\paragraph{Prelimiminaries}
\begin{itemize}
\item You need a gitlab account; access to @mlr_students@
\item Connect to the local mlr-robolab WIFI
\end{itemize}

\paragraph{Install from a fresh Ubuntu}
\begin{itemize}
\item install fresh Ubuntu 14.04.4 LTS
\item google 'ros install indigo'; copy\&paste steps; install package
  ros-indigo-desktop
\item install packages: synaptic, git, qtcreator,
  ros-indigo-alvar-msgs, ros-baxter-...
\item create ssh key:
\begin{Verbatim}
cd
ssh-keygen
cat .ssh/id_rsa.pub
\end{Verbatim}
\item enter ssh key in gitlab: gitlab start page; profile settings;
  ssh keys; copy\&paste the key (without linebreaks!!!); 'Add key'
\item in gitlab go to the project page; see the ssh URL ending with ...git
\item checkout our code
\begin{Verbatim}
cd
mkdir git
cd git
git clone <SSH-GIT-URL>
\end{Verbatim}
\item Install the code dependency ubuntu packages: 
\begin{Verbatim}
cd ~/git/mlr/install
./INSTALL_ALL_UBUNTU_PACKAGES.sh
\end{Verbatim}
Trouble shooting: read the README.md in ~/git/mlr
\item configure code and test make:
\begin{Verbatim}
cd ~/git/mlr/share/
git checkout baxter
cp gofMake/config.mk.default gofMake/config.mk
bin/createMakefileLinks.sh
cd src/Ors
make
\end{Verbatim}
\item goto project page and test make
\begin{Verbatim}
cd ~/git/mlr/share/teaching/RoboticsPractical/01-...
make
\end{Verbatim}
Test starting to run @./x.exe@
\end{itemize}

\paragraph{Make the baxter move}
\begin{itemize}
\item setup the WIFI connection to the baxters ros server

@source ~/git/mlr/share/bin/baxterwifisetup@

\item In a project folder, try to run @./x.exe -useRos 1@
\end{itemize}


\paragraph{Get comfortable}
\begin{itemize}
\item put all extra documentation useful for others in text files in
  ./doc
\item use qtcreater; learn create 'new project' (using 'import
  existing project' for a path with makefile); learn to set 'include paths'
\item create own folder @groupX@, maybe own branch
\end{itemize}



\section{Plan}

\subsection{Milestone 1: Pick-and-place}

Target: The robot perceives objects on the table (= segment,
localize). The robot grasps them and puts them into a bin.

\subsubsection{Subproblem: Basic Motion}

Learn how to use our code to generate targets in various task
spaces. Learn how create @CtrlTask@s directly in C++. Optionally, have
a look at the much mroe abstract RAP interface.

\subsubsection{Lecture: Basic Motion revisited}
\begin{itemize}
\item Task spaces, general problem
\item linear acceleration laws in task spaces
\item maths to project them down to configuration space
\item Discuss (practial is later): impedance, stiffness
\end{itemize}

\subsubsection{Subproblem: Segmenting \& tracking objects}

Understand how the @tabletop@ ROS packages can extract planes (the
table) and point cloud clusters on top of the plane. Learn how the
objects are imported in our system.

\subsubsection{Lecture: Basic perception}
\begin{itemize}
\item The pain of computer vision...
\item Keep it simple: point clouds, planes, clusters, markers
\item Practical packages
\end{itemize}

\subsubsection{Subproblem: Pick \& Place}

Realize the whole pick-and-place scenario. Core issues are
\begin{itemize}
\item Designing the motion tasks
\item Sequening, ideally failure detection \& reaction
\end{itemize}




\subsection{Milestone 2: System Identification, Machine Learning \&
  Compliant Optimal Control}

Target: The robot is controlled on the lowest level, sending direct
'torques' (or alike). Using system identification (ML) we learnt a
perfect model of both, the dynamics and the observations. Using
Bayesian filtering we can perfectly track the state---giving nice and
smooth velocity estimates. The robot 'intelligently' explores its
state-space to collect data for the previous tasks.

\subsubsection{Lecture: Dynamics Basics; and motivation}

\begin{itemize}
\item Dynamics \& optimal control revisited
\item Compliance, impedance control, manipulation \& teleoperation
\item (Do we have F/T sensors?)
\item caveats of real robots: 'non-Markovian', sticktion, time lag,
  gear clearance
\end{itemize}

\subsubsection{Subproblem: Collect data, formulate model, ML}

Think about motion patterns to collect data. Formulate models for the
robot dynamics as well as observation model. Apply ML.

\subsubsection{Subproblem: Use the model for (extended/unscented)
  Kalman filtering of the state}


\subsubsection{Subproblem: Use the model to translate desired
  $q$-accelerations directly to torques}



\subsection{Define your own project!}

\end{document}
